%% Copyright (c) 2022 Martín E. Zahnd
%%
%% This code is licensed under MIT license (see LICENSE.txt for details)
%%
\chapter{Lenguaje S}
%\graphicspath{ {./teoria/resources/lenguaje-s/} }

\section{Variables, etiquetas e instrucciones}
\begin{definicion}{Variables}{}
    Existen tres tipos de variables, las cuales no se declaran:
   \begin{enumerate}
       \item \textbf{Variables de entrada:} $X_1, X_2, X_3, \dotsc$

       \item \textbf{Variables temporales:} $Z_1, Z_2, Z_3, \dotsc$

       \item \textbf{Variable de salida:} $Y$
   \end{enumerate} 

   El tipo de datos es $\mathbb{N}$ y las variables auxiliares y la de salida
   comienzan inicializadas en 0.
\end{definicion}

\begin{definicion}{Etiquetas}{}
    Las etiquetas se representan como:
    \begin{gather*}
        A_1, B_1, C_1, D_1, E_1, A_2, \dotsc
    \end{gather*}
\end{definicion}

\begin{definicion}{Instrucciones}{}
    Existen tres tipos de instrucciones:
    \begin{enumerate}
        \item Sea $V$ una variable.
            \begin{gather*}
                V \gets V + 1
            \end{gather*}

            Si la variable $V$ tiene el valor $n \in \mathbb{N}$, luego de
            ejecutarse la instrucción va a tener el valor $n+1$.
        \item Sea $V$ una variable.
            \begin{gather*}
                V \gets V - 1
            \end{gather*}
            Si la variable $V$ tiene el valor $n \in \mathbb{N}_{\geq 1}$, 
            luego de ejecutarse la instrucción va a tener el valor $n-1$.
            Si $V$ tiene el valor 0, luego de ejecutarse queda con el mismo
            valor.
        \item Sea $V$ una variable y $L$ una etiqueta.
            \begin{gather*}
                IF ~ V \neq 0 \quad GOTO ~ L
            \end{gather*}

            Si la variable $V$ tiene el valor 0, se pasa a la próxima 
            instrucción; si tiene un valor distinto de cero, se pasa a la 
            primer instrucción del programa a la cual la antecede la etiqueta
            $L$.
    \end{enumerate}

    A cada instrucción la puede anteceder una etiqueta, la cual se escribe
    entre corchetes:
    \begin{gather*}
        [L] \quad \text{Instrucción}
    \end{gather*}
\end{definicion}

\subsection{Programa}

\begin{definicion}{Programa}{}
   Un programa es una lista finita de instrucciones $I_1, I_2, \dotsc, I_n$
   que se escriben una debajo de la otra.
\end{definicion}

\subsubsection{Ejemplos}

\begin{itemize}
    \item Dado el siguiente programa:
        \begin{align*}
            [A_1] \quad &X_1 \gets X_1 - 1 \\
                        &Y \gets Y + 1 \\
                        &IF ~ X \neq 0 \quad GOTO ~ A_1
        \end{align*}

        Veamos qué ocurre si entramos con diferentes valores:
        
        \begin{center}
        \begin{tabular}{c c c}
            \begin{tabular}{c | c}
                $X$ & $Y$ \\
                \hline
                2 & 0 \\
                1 & 1 \\
                0 & 2 \\
            \end{tabular}

            &

            \begin{tabular}{c | c}
                $X$ & $Y$ \\
                \hline
                1 & 0 \\
                0 & 1 \\
                \phantom{0} & \phantom{0} \\
            \end{tabular}

            &

            \begin{tabular}{c | c}
                $X$ & $Y$ \\
                \hline
                0 & 0 \\
                0 & 1 \\
                \phantom{0} & \phantom{0} \\
            \end{tabular}
        \end{tabular}
        \end{center}

        Este programa implementa $f: \mathbb{N} \to \mathbb{N} / f(n) =
        \begin{cases}
            n & n \neq 0 \\
            1 & n = 0
        \end{cases}$

    \item Supongamos que el programa tiene una única instrucción:
        \begin{gather*}
            Y \gets Y - 1
        \end{gather*}
        
        Este programa computa la función $f: \mathbb{N} \to \mathbb{N}/f(n)=0$.

        Es decir, el programa presentado es una forma de implementar la 
        función constante cero.


        \bigskip
        \textit{Observación:}
        Notemos que
        \begin{align*}
            &X_1 \gets X_1 - 1 \\
            &Y \gets Y - 1
        \end{align*}
        es un programa que computa la misma función.

        Entonces dos programas distintos pueden computar la misma función.

    \item Dado el siguiente programa:
        \begin{gather*}
           \left.\begin{aligned}
                   Y &\gets Y + 1 \\
                Y &\gets Y + 1 \\
                  &\vdots \\
                Y &\gets Y + 1
               \end{aligned}\right\} k \text{ instrucciones}
        \end{gather*}

        Este programa computa la función $f: \mathbb{N} \to \mathbb{N}/f(n)=k$

    \item Dado el siguiente programa:
        \begin{gather*}
            [A_1] \quad IF ~ X_1 \neq 0 \quad GOTO ~ A_1
        \end{gather*}

        \begin{center}
        \begin{tabular}{c c}
            \begin{tabular}{c | c}
                $X_1$ & $Y$ \\
                \hline
                2 & 0 \\
                2 & 0 \\
                2 & 0 \\
                \vdots & 0 \\
            \end{tabular}

            &

            \begin{tabular}{c | c}
                $X_1$ & $Y$ \\
                \hline
                0 & 0 \\
            \end{tabular}
        \end{tabular}
        \end{center}

        Notemos que el programa nunca termina si entramos con una constante
        distinta de cero.

        Entonces este programa computa la función 
        \nota{$Dom(f) = \{ 0 \}$}%
        $f: \mathbb{N} \to \mathbb{N} /
        f(x) = \begin{cases}
            \uparrow & x \neq 0 \\
            0 & x = 0
        \end{cases}$

        Donde $\uparrow$ significa que no está definida la función, es decir,
        el programa nunca termina.

    \item \textbf{Función identidad}

        Dado el siguiente programa
        \begin{align*}
            [A] \quad &IF ~ X_1 \neq 0 \quad GOTO ~ B \\
                        &Z_1 \gets Z_1 + 1 \\
                        &IF ~ Z_1 \neq 0 \quad GOTO ~ E \\
            [B] \quad &X_1 \gets X_1 - 1 \\
                        &Y \gets Y + 1 \\
                        &Z_1 \gets Z_1 + 1 \\
                        &IF ~ Z_1 \neq 0 \quad GOTO ~ A
        \end{align*}

        Como la etiqueta $E_1$ no antecede a ninguna instrucción, cuando el
        programa busca dicha etiqueta termina.

        Este programa computa la función identidad:
        $f: \mathbb{N} \to \mathbb{N} / f(x) = x$
\end{itemize}


\section{Macros}

\begin{definicion}{Macro}{}
    Una macro es una pseudoinstrucción que representa un segmento de 
    un programa.
\end{definicion}

\medskip

\begin{definicion}{Expansión de una macro}{}
    Cada vez que en un programa $\mathcal{P}$ aparece una macro, la misma se 
    debe reemplazar por el segmento del programa que representa.
    Este proceso de denomina \textit{expansión de la macro}.
\end{definicion}

Debemos ser cuidadosos de que las variables auxiliares y las etiquetas que
se utilicen para expandir una macro no aparezcan en otras instrucciones del
programa.

\subsubsection{Ejemplos de macros}

\begin{itemize}
    \item La macro \textit{salto incondicional} se representa como:
        \begin{gather*}
            GOTO ~ L
        \end{gather*}
        siendo $L$ una etiqueta.

        Esta macro se expande de la siguiente manera:
        \begin{align*}
            &V \gets V + 1 \\
            &IF ~ V \neq 0 \quad GOTO ~ L
        \end{align*}

        Donde $V$ es una variable temporal que no aparece en el resto del 
        código.

    \item La macro \textit{asignación de cero} se representa como:
        \begin{gather*}
            V \gets 0
        \end{gather*}

        Y se expande como
        \begin{align*}
            [L] \quad &V \gets V - 1 \\
                      &IF ~ V \neq 0 \quad GOTO ~ L
        \end{align*}

        Donde $L$ es una etiqueta que no aparece en el resto del programa.

    \item La macro \textit{asignación de variable} se representa como
        \begin{gather*}
            V_1 \gets V_2
        \end{gather*}

        Y se expande:
        \nota{Recordar que las etiquetas que aparecen en esta macro
        \underline{no} deben ser utilizadas en el programa donde se utiliza la
        macro. Lo mismo para la variable auxiliar.}%
        \begin{align*}
                        &V_1 \gets 0 \\
            [L_1] \quad &IF ~ V_2 \neq 0 \quad GOTO ~ L_2 \\
                        &GOTO ~ L_3 \\
            [L_2] \quad &V_2 \gets V_2 -1 \\
                        &V_1 \gets V_1 + 1 \\
                        &Z_1 \gets Z_1 + 1 \\
                        &GOTO ~ L_1 \\
            [L_3] \quad &IF ~ Z_1 \neq 0 \quad GOTO ~ L_4 \\
                        &GOTO ~ L_5 \\
            [L_4] \quad &Z_1 \gets Z_1 - 1 \\
                        &V_2 \gets V_2 + 1 \\
                        &GOTO ~ L_3
        \end{align*}

        \bigskip
        \textit{Observación:}
        Notemos que si cambiamos $V_1$ por $Y$ y $V_2$ por $X_1$ y vemos la
        macro como un programa completo este computaría la función identidad.
        A diferencia del ejemplo en la sección anterior, este programa 
        resguardaría el valor de entrada.
\end{itemize}

\subsubsection{Ejemplos de programas con macros}

\begin{itemize}
    \item \textbf{Función suma} \label{ej:funcion-suma}

        El siguiente programa computa la función 
        $f: \mathbb{N} \times \mathbb{N} \to \mathbb{N}/f(x_1, x_2) = x_1+x_2$

        \begin{align*}
                        &Y \gets X_1 \\
                        &Z_1 \gets X_2 \\
            [B_1] \quad &IF ~ Z_1 \neq 0 \quad GOTO ~ A_1 \\
                        &GOTO ~ E_1 \\
            [A_1] \quad &Z_1 \gets Z_1 - 1 \\
                        &Y \gets Y + 1 \\
                        &GOTO ~ B_1
        \end{align*}

        A partir de ahora podemos usar el macro
        \begin{gather*}
            V_1 \gets V_2 + V_3
        \end{gather*}

    \item \textbf{Función producto} \label{ej:funcion-producto}

        El siguiente programa computa la función 
        $f: \mathbb{N} \times \mathbb{N} \to \mathbb{N}/f(x_1, x_2) = x_1 x_2$

        \begin{align*}
                        &Z_2 \gets X_2 \\
            [B_1] \quad &IF ~ Z_2 \neq 0 \quad GOTO ~ A_1 \\
                        &GOTO ~ E_1 \\
            [A_1] \quad &Z_2 \gets Z_2 + 1 \\
                        &Z_1 \gets X_1 + Y \\
                        &Y \gets Z_1 \\
                        &GOTO ~ B_1
        \end{align*}

    \item \textbf{Función sucesor} \label{ej:funcion-sucesor}

        El siguiente programa computa la función
        $suc \mathbb{N} \to \mathbb{N} / suc(x) = x + 1$

        \begin{align*}
            &Y \gets X_1 \\
            &Y \gets Y + 1
        \end{align*}

    \item \textbf{Función proyección} \label{ej:funcion-proyeccion}

        \nota{$1 \leq i \leq k$}%
        $\Pi^k_i: {\mathbb{N}}^k \to \mathbb{N} / 
        \Pi_i (x_1, \dotsc, x_i, \dotsc, x_k) = x_i$

        Basta con cambiarle al programa que computa la función identidad su 
        variable $X_1$ por la variable $X_i$, 
        donde $X_i$ es la i-ésima variable que se quiere proyectar.
        \begin{align*}
            &Y \gets X_i
        \end{align*}
\end{itemize}

\bigskip
\textit{Observación:}
Si podemos programar la función $f: \mathbb{N}^k \to \mathbb{N}$, entonces
podemos implementar el macro
\begin{gather*}
    Z \gets f(V_1, \dotsc, V_k)
\end{gather*}

Si $\mathcal{P}$ es el programa  de $f$, cambio las apariciones de $Y$ por la
variable $Z$ y después las apariciones de $X_1, \dotsc, X_k$ por 
$V_1, \dotsc, V_k$, respectivamente, y el macro se expande como
\begin{gather*}
   \begin{aligned}
        Z_1 &\gets V_1 \\
        Z_2 &\gets V_2 \\
          &\vdots \\
        Z_k &\gets V_k \\
        \mathcal{P} \\
        V_n &\gets Z_1 \\
          &\vdots \\
        V_k &\gets Z_k \\
    \end{aligned}
    \notamath{$Z_1, \dotsc, Z_k$ no aparecen en $\mathcal{P}$ ni en el resto 
    del programa}
\end{gather*}

\section{Descripciones de programas}

\subsection{Estado de un programa}

\begin{definicion}{Estado de un programa}{}
    El estado de un programa $\mathcal{P}$ es una lista finita de ecuaciones 
    de la forma
    \begin{gather*}
        V = m
    \end{gather*}

    Donde $V$ es una variable y $m \in \mathbb{N}$.

    \medskip

    Hay una única ecuación para cada variable que aparece en $\mathcal{P}$.
\end{definicion}

\subsubsection{Ejemplo}

Dado el siguiente programa $\mathcal{P}$:
\begin{align*}
    [A_1] \quad &X_1 \gets X_1 - 1 \\
                &Y \gets Y + 1 \\
                &IF ~ X_1 \neq 0 \quad GOTO ~ A_1
\end{align*}

Entonces los estados de $\mathcal{P}$ son:
\begin{align*}
    X_1 &= 2, Y = 1 \\
    X_1 &= 3, Y = 1, Z_1 = 0 \\
    X_1 &= 5, Y = 1, Z_1 = 6
\end{align*}

Notemos que no es necesario que los estados sean alcanzados.

\medskip

Por otra parte, las siguientes secuencias \underline{no} son estados:
\begin{align*}
    X_1 = 2 \quad \text{Falta una ecuación asociada a la variable } Y \\
    X_1 = 2, Z_1 = 3 \quad \text{Falta una ecuación asociada a la 
    variable } Y \\
    Y = 1, X_1 = 6, Y = 2 \quad \text{Hay dos secuencias asociadas a la 
    variable } Y
\end{align*}

\subsection{Descripción instantánea de un programa o Snapshot}

\begin{definicion}{Descripción instantánea de un programa}{}
    Supongamos que un programa $\mathcal{P}$   tiene longitud $n$, es decir,
    consta de $n$ instrucciones $I_1, I_2, \dotsc, I_n$.

    \begin{itemize}
        \item Para un estado $\sigma$ de $\mathcal{P}$ y un 
            $i \in \{ 1,2,\dotsc,n+1\}$ tenemos que el par $(i, \sigma)$ es 
            una descripción instantánea de $\mathcal{P}$, la cual se llama 
            terminal si $i = n+1$.

            $i$ nos dice a qué instrucción apunta antes de ser ejecutada, y
            en $\sigma$ están los valores de las variables antes de ejecutar
            la instrucción $i$.

        \item Para una $(i, \sigma)$ no terminal se puede definir su sucesor 
        $(j, \tau)$:
        \begin{enumerate}
            \item La i-ésima instrucción de $\mathcal{P}$ es $V \gets V+1$.

            Entonces $j = i + 1$, y $\tau = \sigma$ salvo que la ecuación 
            $V=m$ en $\sigma$ va a ser $V=m+1$ en $\tau$

            \item La i-ésima instrucción de $\mathcal{P}$ es $V\gets V-1$.

                Entonces $j= i + 1$, y $\tau = \sigma$ salvo que
                $V=m$ en $\sigma$ será 

                $V = \begin{cases}
                    m - 1 & \text{si } m > 0 \\
                    0 & \text{si } m = 0
                \end{cases} ~$ en $\tau$.
                
            \item La i-ésima instrucción de $\mathcal{P}$ es 
                $IF ~ V \neq 0 \quad GOTO ~ L$

                Entonces 
                $\tau = \sigma$ y 
                \begin{itemize}
                    \item Si $V=0$ en $\sigma$
                        $\implies$
                        $j=i+1$
                    \item Si $V \neq 0$ en $\sigma$
                        $\implies$

                        $j = \begin{cases} 
                    \min{\{ k: I_k \text{ tiene etiqueta } L \}} & \text{si 
                    existe } [L] ~ I_k \\
                    n+1 & \text{si no existe } [L] ~ I_k
                \end{cases}$
                \end{itemize}
        \end{enumerate}
    \end{itemize}
\end{definicion}

\subsection{Cómputo}
\begin{definicion}{Cómputo}{}
    Un cómputo de un programa $\mathcal{P}$ a partir de una descripción 
    instantánea $d_1 = (i, \sigma)$ es una lista 
    $d_1, d_2, \dotsc, d_k$ de descripciones
    instantáneas de $\mathcal{P}$ tales que $d_{j+1}$ es el sucesor de $d_j$.

    Siendo $ 1 \leq j \leq k-1$, y $d_k$ la descripción 
    instantánea terminal.
\end{definicion}

\subsubsection{Ejemplo}

Para el siguiente programa $\mathcal{P}$:
\begin{align*}
    [A_1] \quad &X_1 \gets X_1 - 1 \\
                &Y \gets Y + 1 \\
                &IF ~ X_1 \neq 0 \quad GOTO ~ A_1
\end{align*}

El cómputo a partir de $d_1 = \{ 1, X_1 = 2, Y = 0 \}$ es:
\begin{align*}
    d_2 &= \{ 2, X_1 = 1, Y = 0 \} \\
    d_3 &= \{ 3, X_1 = 1, Y = 1 \} \\
    d_4 &= \{ 1, X_1 = 1, Y = 1 \} \\
    d_5 &= \{ 2, X_1 = 0, Y = 1 \} \\
    d_6 &= \{ 3, X_1 = 0, Y = 2 \} \\
    d_7 &= \{ 4, X_1 = 0, Y = 2 \}
\end{align*}

\subsection{Estado inicial}
\begin{definicion}{Estado inicial}{}
    Sea $\mathcal{P}$  un programa y sean $u_1, \dotsc, u_m$ números naturales.
    
    \medskip

    El estado inical de $\mathcal{P}$ para $u_1, \dotsc, u_m$ es el estado
    $\sigma_1$  definido como
    \begin{gather*}
        \notamath{$j>m$}
        \sigma_1 = (X_1 = u_1, X_2 = u_2, \dotsc, X_m = u_m, X_j = 0, Z_j = 0,
        Y = 0)
    \end{gather*}
    \nota{En cuyo caso no se escriben}%
    Donde $X_j$ y $Z_j$ pueden no aparecer en el programa.

    Por lo tanto la descripción inicial de $\mathcal{P}$ para $u_1,\dotsc,u_m$
    es $d_1 = (1, \sigma_1)$.
\end{definicion}

\medskip

\begin{definicion}{Cómputo a partir del estado inicial}{}
    Sea $\mathcal{P}$ un programa, sean $u_1, \dotsc, u_m$ números naturales
    y $\sigma_1$ el estado inicial para ellos.

    \medskip

    Entonces existen dos posibilidades:
    \begin{enumerate}
        \item Hay un cómputo de $\mathcal{P}$: $d_1, \dotsc, d_k$, siendo
            $d_1 = (1, \sigma_1)$.

            \bigskip
            \textbf{Notación:}
            $\Psi_{\mathcal{P}}^{k} (u_1, \dotsc, u_k)$ es el valor de $Y$
            en la descripción instantánea $d_k$.

        \item Existe una secuencia infinita $d_1, d_2, \dotsc$ siendo
            \nota{No hay tal cómputo}%
            $d_1 = (1, \sigma_1)$ y $d_{i+1}$ el sucesor de $d_i$.

            \bigskip
            \textbf{Notación:}
            $\Psi_{\mathcal{P}}^{k} (u_1, \dotsc, u_k) = \uparrow$ significa
            que $\Psi_{\mathcal{P}}^{k} (u_1, \dotsc, u_k)$ está indefinido.
    \end{enumerate}
\end{definicion}

\subsection{Función computable}

\begin{definicion}{Función parcialmente computable}{}
    Una función $f$ es parcialmente computable si existe un programa 
    $\mathcal{P}$ tal que $f = \Psi_{\mathcal{P}}^k$

    \medskip

    Donde si $x = (x_1, \dotsc, x_k) \in \mathbb{N}^k$ entonces
    \begin{itemize}
        \item Si $x \in Dom(f) \implies$
            \nota{Para todo \\$(u_1, \dotsc, u_m) \in \mathbb{N}^k$}%
            $f(u_1, \dotsc, u_k) = \Psi_{\mathcal{P}}^k (u_1, \dotsc, u_k)$
    \item Si $x \notin Dom(f) \implies$
            $\Psi_{\mathcal{P}}^k (x_1, \dotsc, x_k) = \uparrow$
    \end{itemize}
\end{definicion}

\medskip

\begin{definicion}{Función total}{}
    Sea $f$ una función.

    \medskip

    Decimos que $f$ es total si su dominio es $\mathbb{N}^m$.
\end{definicion}

\medskip

\nota{Noni la llama \\``computable'' a secas;
    Santiago usa el nombre completo: ``total computable''.}
\begin{definicion}{Función total computable}{}
    Una función $f: \mathbb{N}^k \to \mathbb{N}$ es total computable si es 
    parcialmente computable y total.
\end{definicion}

\bigskip
\textit{Observación:}
Notemos que el mismo programa puede servir para computar funciones de
$1, 2, 3, \dotsc$ variables.

Por lo tanto si en $\mathcal{P}$ aparece $X_n$ y no aparece $X_i$, con $i>n$,
hay dos opciones:
\begin{enumerate}
    \item Si sólo se especifican $m < n$ valores de entrada, entonces 
        $X_{m+1}, \dotsc, X_n$ toman el valor 0.
    \item Si se especifican $m>n$ valores de entrada, $\mathcal{P}$ ignorará
        $u_{n+1}, \dotsc, u_m$.
\end{enumerate}

\subsubsection{Ejemplo}

Sea el programa $\mathcal{P}$
\begin{gather*}
    [A_1] \quad IF ~ X_1 \neq 0 \quad GOTO ~ A_1
\end{gather*}

Notemos que $\Psi_{\mathcal{P}}^{1} (x) = \begin{cases}
    0 & x = 0\\
    \uparrow & x \neq 0
\end{cases} ~$ pues:

\begin{align*}
    d_1 &= (1, X_1 = 0, Y = 0) \\
    d_2 &= (2, X_1 = 0, Y = 0) \notamath{Terminal}\\
    \\
    d_1 &= (1, X_1 = n, Y = 0) \notamath{$n>0$} \\
    d_2 &= (1, X_1 = n, Y = 0)\\
        & \implies d_j = d_1 ~ \forall j \in \mathbb{N}
\end{align*}
\begin{gather*}
    \therefore ~ \nexists \text{ un cómputo de } d_1 \text{ de } \mathcal{P}
\end{gather*}

La función $f$ que computa es parcialmente computable pues su dominio es 
$\{0\}$, siendo $f: \{ 0 \} \to \mathbb{N} / f(0) = 0$.

\bigskip


\section{Teorema}

\begin{teorema}{Composición de funciones computables}{composicion-computables}
    Sea $f: \mathbb{N}^k \to \mathbb{N}$ (parcialmente) computable.

    Sean $g_1, g_2, \dotsc, g_k: \mathbb{N}^m \to \mathbb{N}$ (parcialmente) 
    computables.

    \medskip

    Entonces $h: \mathbb{N}^m \to \mathbb{N}$ tal que
    \begin{gather*}
        \notamath{Es decir: \\
        $h = f \circ (g_1 \times g_2 \times \dotsc \times g_k)$}%
        h(x_1, \dotsc, x_m) = f(g_1 (x_1, \dotsc, x_m), g_2 (x_1, \dotsc, x_m),
    \dotsc, g_k (x_1, \dotsc, x_m))
    \end{gather*}

    \smallskip

    es (parcialmente) computable.
\end{teorema}

\begin{proof} \phantom{.}

    Como $f$ es (parcialmente) computable 
    $\implies \exists \; \mathcal{P}_f$ que
    la computa $\implies$ podemos armar una macro.

    Como $g_j$ es (parcialmente) computable 
    $\implies \exists \; \mathcal{P}_{g_j}$ 
    que la computa $\implies$ podemos armar una macro.

    Armo el siguiente programa para probar que $h$ es (parcialmente) 
    computable.

    \begin{align*}
        Z_1 \gets& g_1 (x_1, \dotsc, x_m) \\
        Z_2 \gets& g_w (x_1, \dotsc, x_m) \\
        \vdots & \\
        Z_k \gets& g_k (x_1, \dotsc, x_m) \\
        Y \gets& f(z_1, \dotsc, z_k)
    \end{align*}
\end{proof}

\medskip

\noindent\rule{\textwidth}{0.25pt}

\medskip

\textbf{Bonus:} libro recomendado por María Laura Noni.

``\textit{Introduction to Automata Theory, Languages, and Computation by 
Hopcroft, J. and Ullman, J}.''


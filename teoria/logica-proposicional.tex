%% Copyright (c) 2022 Martín E. Zahnd
%%
%% This code is licensed under MIT license (see LICENSE.txt for details)
%%
\chapter{Lógica Proposicional}\label{chap:logica-proposicional}
% \graphicspath{ {./teoria/resources/logica-proposicional/} }

\section{Lenguajes}
\subsection{Alfabeto}

\begin{definicion}{Alfabeto}{}
    Sea $A$ un conjunto, con $A \neq \varnothing$.

    \medskip

    El conjunto $A$ se llama alfabeto.
\end{definicion}


\subsection{Expresión}

\begin{definicion}{Expresión}{}
    Una expresión es una sucesión finita de elementos de $A$ o la cadena 
    vacía, a la cual llamamos $\lambda$.
\end{definicion}

\subsubsection{Ejemplo}
\begin{gather*}
    A = \{ 1,2,3,5,4,6 \}
\end{gather*}

Las expresiones son: $121$, $4$, $\lambda$, $66666$


\subsection{Longitud}

\begin{definicion}{Longitud}{}
    Sea $E = e_0 e_1 \dots e_{n-1}$ una expresioń de elementos de $A$, 
    definimos:
    \begin{enumerate}
        \item $long(E) = n$
        \item $long(\lambda)=0$
    \end{enumerate}
\end{definicion}

\subsection{Conjunto A estrella}

\begin{definicion}{$A^{*}$}{}
    \begin{gather*}
        A^{*} = \bigcup_{n\in \mathbb{N}} A^n
    \end{gather*}
\end{definicion}

Entonces, en $A^{*}$ están:
\begin{gather*}
    A^0 = \{ \lambda \}, \; \notamath{$\lambda = \text{`` ''}$}
    A^1 = A, \; 
    A^j = \{ e \text{ expresión en } A / long(e) = j \}
\end{gather*}

\bigskip
\textit{Observación:}
$A^{*}$ es infinito.

\begin{proof} \phantom{.}

    Sea $A \neq \varnothing$.

    \begin{align*}
         \implies& \exists x \in A \\
         \implies& x, xx, xxx, xxxx, \dotsc \in A^{*}
    \end{align*}

    Por lo tanto
    \begin{gather*}
        f: \mathbb{N} \to A^{*} / f(n) =
        \begin{cases}
            \lambda & n = 0 \\
            \underbrace{x \, . \, x \, . \, \dotsc \, . \, x}_{n 
        \text{ veces}} & n \geq 1
        \end{cases}
    \end{gather*}

    Y como $f$ es inyectiva $\implies \# A^{*} \geq \aleph_0$
\end{proof}

\subsection{Lenguaje}

\begin{definicion}{}{}
    Dado $A$ un alfabeto.

    \medskip

    \nota{$\Sigma \subseteq A^{*}$}% 
    Un lenguaje $\Sigma$ sobre $A$, con $\Sigma \neq \varnothing$,
    es un subconjunto de $A^{*}$.

\end{definicion}

\subsubsection{Ejemplos}

\begin{enumerate}
    \item $A = \{ a, \dotsc, z \} \implies \# A = 27$

        Entonces definimos el lenguaje como:
        
        \begin{gather*}
            \Sigma \subseteq A^{*} / \Sigma = \{ e \in A^{*} / e 
                \text{ es una palabra que está en la última edición del } \\ 
            \text{diccionario de la Real Academia Española.} \}
        \end{gather*}

    \item Sea $A = \{ 0, 1, 2, 3, 4, 5, 6, 7, 8, 9 \}$.

        \begin{align*}
            \mathcal{L} &= \{ x \in A^{*} / x 
            \text{ representa un número natural} \} \\
              &= \{ x \in A^{*} / x = a_1, \dotsc, a_n, \text{ con }
              a_j \in A \text{ y } a_1 \neq 0\} \cup \{ 0 \}
        \end{align*}
\end{enumerate}

\subsection{Igualdad de expresiones}


\begin{definicion}{Igualdad de expresiones}{}
    Sean las expresiones $E$ y $F$ sobre $A$.

    \medskip

    Entonces $E = F \text{ si } long(E) = long(F)$ y
    \begin{center}
    \begin{enumerate}[%
                    labelindent=*,
                    style=multiline,
                    leftmargin=*,
                    align=left,
                    leftmargin=2\parindent,
                    label=Caso \arabic*)]
        \item $long(E)= 0 \implies E = F = \lambda$
        \item $long(E) = n \implies E = e_0 e_1 \dots e_{n}$ y
            \nota{$0 \leq i \leq n$}%
            $F = f_0 f_1 \dots f_j$, con $n = j$ y $e_i = f_i$.
    \end{enumerate}
    \end{center}
\end{definicion}

\subsection{Concatenación}

\begin{definicion}{Concatenación}{}
    \begin{center}
        \begin{enumerate}[%
                        labelindent=*,
                        style=multiline,
                        leftmargin=*,
                        align=left,
                        leftmargin=2\parindent,
                        label=Caso \arabic*)]
            \item Sean las expresiones $E = e_0 e_1 \dots e_{n-1}$ 
                y 
        $F = f_0 f_1 \dots f_{k-1}$, con $E, F \in A^{*}$

        \medskip

        Definimos $E$ concatenado con $F$ como:
        \begin{gather*}
            EF = e_0 e_1 \dots e_{n-1} f_0 f_1 \dots f_{k-1}
        \end{gather*}

            \item Sean $E = e_0 e_1 \dots e_{n-1}$, $F = \lambda$

                \medskip
                Definimos $E$ concatenado con $F$ como:
                \begin{gather*}
                    EF = E
                \end{gather*}
            \item Sean $E = e_0 e_1 \dots e_{n-1}$, $F = \lambda$
                
                \medskip
                Definimos $F$ concatenado con $E$ como:
                \begin{gather*}
                    FE = E
                \end{gather*}
            \item Sean $E = F = \lambda$

                \medskip
                Definimos $E$ concatenado con $F$ como:
                \begin{gather*}
                    EF = \lambda
                \end{gather*}
        \end{enumerate}
    \end{center}
\end{definicion}

\bigskip
\textit{Observación:}
Notemos que $long(EF) = long(E) + long(F) = n+k$

\bigskip
\textit{Observación:}
Notemos que 
$E F = e_0 e_1 \dots e_{n-1} f_0 f_1 \dots f_{k-1}$ mientras que

$F E = f_0 f_1 \dots f_{k-1} e_0 e_1 \dots e_{n-1}$. 

Por lo tanto
\begin{gather*}
    EF \neq FE \notamath{Excepto que $E = F$}
\end{gather*}

\medskip

\begin{teorema}{}{efgh-longe-geq-longg}
    Sean $A$ alfabeto, $E, F, G, H \in A^{*}$ 

    \medskip

    Si $EF = GH$ y $long(E) \geq long(G)$
    \begin{gather*}
        \implies \exists \text{ una expresión } H' \in A^{*}/E=GH'
    \end{gather*}
\end{teorema}

\begin{proof} \phantom{.}

   Por inducción en $long(E) = n$.

   Sea $\mathcal{P}(n) = A$, con $A$ alfabeto tal que $E, F, G, H \in A^{*}$
   y $long(E) = n \geq long(G)$ $\implies$ $\exists \; H' \in A^{*} / E = GH'$

   \begin{itemize}
       \item Caso base) $P(0)$ 
           \begin{align*}
                long(E) = 0 &\implies E = \lambda \\
                long(E) \geq long(G) \geq 0 &\implies G = \lambda
           \end{align*}

           Tomo $H' = \lambda \implies E = GH'$

    \item HI) $\mathcal{P}(n)$

    \item T) $\mathcal{P}(n+1)$

   \end{itemize}

   \bigskip 

   Sea $E, F, G, H \in A^{*} / EF = GH$
   \begin{gather*}
       long(E) = \overbrace{n+1}^{\geq 1} \geq long(G) = k+1 \\
       \therefore ~  E = e_1, \dotsc, e_{n+1}
   \end{gather*}

    \begin{enumerate}[%
        labelindent=*,
        style=multiline,
        leftmargin=*,
        align=left,
        leftmargin=2\parindent,
        label=Caso \arabic*)]
        \item $G \neq \lambda$ y $G = g_1 g_2 \dots g_k$
           
            \begin{gather*}
                E = e_1 \underbrace{e_2 \dots e_{n+1}}_{\widetilde{E}} \; \; 
                G = g_1 \underbrace{g_2 \dots g_k}_{\widetilde{G}} 
            \end{gather*}

            Como $EF = GH \implies \widetilde{E} F = \widetilde{G}H$

            \begin{gather*}
                n = long(\widetilde{E}) = long(E) - 1 
                \geq long(G)-1 = long(\widetilde{G})
            \end{gather*}

            \begin{align*}
                \notamath{Por HI} 
                &\implies \exists \; H' \big/ \widetilde{E} = \widetilde{G} H' \\
                &\implies e_1 \widetilde{E} = g_1 \widetilde{G} H' \\
                &\implies E = GH'
            \end{align*}

        \item $G = \lambda$

            Quiero probar que $\exists \; H' \in A^{*} / E = GH'$

            Tomo, entonces, $H' = E$ y obtengo lo que buscaba.
    \end{enumerate}

\end{proof}

\begin{corolario}{}{alfabeto-igual-long}
    Sean $A$ alfabeto, $E, F, G, H \in A^{*} / EF = GH$

    \medskip

    \begin{gather*}
        long(E) = long(G) \implies E=G \text{ y } F=H
    \end{gather*}
\end{corolario}

\begin{proof} \phantom{.}

    Sean $A$ alfabeto, $E, F, G, H \in A^{*} / EF = GH$

    Si $long(E) = long(G) \implies long(E) \geq long(G)$.

    Por lo tanto, por el Teorema \ref{teo:efgh-longe-geq-longg}, 
    $\exists \; H' \in A^{*}/E=GH'$

    Luego
    \begin{align*}
        long(E) &= long(G) + long(H') \\
        0 &= long(H') \notamath{$long(E) = long(G)$}\\
          &\implies H' = \lambda \\
          &\therefore ~ E= G \underbrace{\implies}_{EF=GH} F = H 
    \end{align*}

\end{proof}

\pagebreak
\section{Sintaxis}

\nota{Santiago: ``\textit{Truco: casi todos los ejercicios de este tema salen
por inducción}''}%
\begin{definicion}{Alfabeto de la lógica proposicional}{}
    
    \begin{gather*}
        A = \mathrm{VAR} \cup \{ (, ) \} \cup C
    \end{gather*}

    Siendo
    \begin{gather*}
        \notamath{$Card(\mathrm{VAR}) = \aleph_0$}
        \mathrm{VAR} = \{ p_n / n \in \mathbb{N} \} = \{ p_0, p_1, p_2, \dots \}
    \end{gather*}

    \medskip

    Y los elementos del conjunto de conectivos 
    $C = \{ \wedge, \vee, \to, \neg \} $: 
    \begin{itemize}
        \item $\wedge$: Conjunción
        \item $\vee$: Disyunción
        \item $\to$: Implicación
        \item $\neg$: Negación
    \end{itemize}
    
\end{definicion}

\bigskip 

\begin{definicion}{Fórmula}{}
    Definimos fórmula como:

    \begin{enumerate}
        \item $\mathrm{VAR} \subseteq \mathrm{FORM}$
        \item $\alpha \in F \implies \neg \alpha \in F$
        \item $\alpha, \beta \in F \implies (\alpha \wedge \beta), 
            (\alpha \vee \beta), (\alpha \to \beta) \in F$
        \item $\alpha \in A^{*}$ es fórmula si se obtiene aplicando 
            finitas veces $1.$, $2.$ y $3.$
    \end{enumerate}

    \bigskip
    \textbf{Notación:}
    Al conjunto de fórmulas del lenguaje proposicional lo llamamos 
    ``$\mathrm{FORM}$'' o ``$F \subset A^{*}$''.
\end{definicion}

\bigskip 

\begin{definicion}{Lenguaje de lógica proposicional}{}
   El lenguaje de lógica proposicional son las fórmulas.
\end{definicion}


\subsubsection{Ejemplos}

\begin{enumerate}
    \item $p_1 \in F$
    \item $\neg p_1 \in F$
    \item $(\neg p_1) \notin F$ \nota{Sobran los $($ $)$}%
    \item $((p_1 \wedge p_2) \to \neg p_3) \in F$
    \item $p_1 \wedge p_2 \notin F$ \nota{Faltan los $($ $)$}%
    \item $((p_1 \wedge p_2) \to (p_2 \vee p_3)) \in F$
    \item $\neg \neg \neg \neg \neg p_6 \in F$
    \item $p_1 \to p_8 \notin F$ \nota{Faltan los $($ $)$}%
\end{enumerate}


\subsection{Eslabones y cadenas de formación}

\begin{definicion}{Cadena de formación}{cf}
    Una sucesión finita $X_1, X_2, \dotsc, X_n$ de expresiones de 
    $A^{*}$ es una cadena de formación (CF) si:

    \begin{center}
        \begin{enumerate}[%
                        labelindent=*,
                        style=multiline,
                        leftmargin=*,
                        align=left,
                        leftmargin=2\parindent,
                        label=Caso \arabic*)]
            \item $X_i \in \mathrm{VAR}$ \nota{$1 \leq i \leq n$}%
            \item $\exists \; j < i / X_i = \neg X_j$ \nota{$1 \leq i, j \leq n$}%
            \item $\exists \; s,t<i / X_i = (X_s * X_t)$
                \nota{$1 \leq s, t, j \leq n$ \\
                    $* \in \underbrace{\{
                    \wedge, \vee, \to \}}_{\substack{
                    \text{Conectivos}\\ \text{binarios}}}$}%
        \end{enumerate}
    \end{center}

    Cada $X_i$ se llama eslabón.
\end{definicion}

\subsubsection{Ejemplos}

\begin{enumerate}
    \item $X_1 = p_1$, $X_2 = \neg X_1$, $X_3 = p_2$, $X_4 = p_3$, 
    $X_5 = (X_3 \to X_4)$, $X_6 = \neg X_5$, $X_7 = \neg X_6$, es una cadena
    de formación pues cada $X_i$ cumple con la definición.
    \nota{$1\leq i \leq 7$}%

    \item $X_1 = p_1$, $X_2 = \neg X_1$, $X_3 = p_2$, $X_4 = p_3$,
        $X_5 = (X_2 \to X_3)$, $X_6 = \neg X_5$
\end{enumerate}

\bigskip
\textit{Observación:}
\nota{$n \geq 2$}%
Si $X_1, \dotsc, X_{n+1}$ es una cadena de formación $\implies$ 
$X_1, \dotsc, X_n$ es una cadena de formación.

\begin{proof} \phantom{.}
    Tarea.
\end{proof}

\subsection{Teorema}

\begin{teorema}{}{}
    \begin{gather*}
        \alpha \in \mathrm{FORM} \iff \exists \; X_1, \dotsc, X_n = \alpha 
        \text{ CF}
    \end{gather*}
\end{teorema}

Este es un teorema importante porque nos dice que $\alpha$ es una fórmula 
sí y sólo sí existe una cadena de formación de la misma.

\begin{proof} \phantom{.}

    \begin{itemize}
        \item $\implies$) Lo vamos a probar por inducción en $long(\alpha)$.
            En principio, utilizando inducción simple.

        \begin{itemize}
            \item CB) $n=1$

                Sea
                $\alpha \in \mathrm{FORM} / long(\alpha) = 1$
                $\implies$ 
                $\alpha = X_1 \in \mathrm{VAR} \subseteq \mathrm{FORM}$
                $\implies$ 
                $X_1$ CF
            \item HI) Sea $\alpha \in \mathrm{FORM}/ long(\alpha) = n \implies$ 
                existe una CF de $\alpha$
                \nota{$\exists\; X_1,\dotsc,X_k=\alpha$}%
            \item T) Sea $\alpha \in \mathrm{FORM}/ long(\alpha) = n + 1 \implies$
                existe una CF de $\alpha$ 
                \nota{$\exists\; X_1,\dotsc,X_{k+1}=\alpha$}%
        \end{itemize}
        
        \medskip

        Sea $\alpha \in F / long(\alpha) = n + 1 > 0$

        \begin{enumerate}[%
                labelindent=*,
                style=multiline,
                leftmargin=*,
                align=left,
                leftmargin=2\parindent,
                label=Caso \arabic*)]
            \item $\alpha = p_j$

                Propongo $X_1 = p_j$

            \item $\alpha = \neg \beta$ \nota{$\beta \in \mathrm{FORM}$}%

                Recordando que $long(\alpha) = n+1$,
                \begin{gather*}
                    long(\alpha) = long(\beta) + 1
                    \implies
                    long(\beta) = n
                \end{gather*}

                \begin{gather*}
                    \therefore ~ \text{ Por HI, } \exists \; X_1, 
                    \dotsc, X_k = \beta \text{ CF de } \beta
                \end{gather*}

                \medskip

                Defino $Y_1 = X_1, Y_2 = X_2, \dotsc, Y_k = X_k, 
                Y_{k+1}=\neg Y_k$.

                \medskip

                Notemos que acá tenemos una cadena de formación para $\alpha$,
                pues hasta $Y_k = X_k$ es una cadena de formación de $\beta$,
                bien definida por la HI, y el eslabón $Y_{k+1} = \neg Y_k$ es
                la negación del eslabón anterior. Por lo tanto, cumple con
                la definición de CF.

            \item $\alpha = (\beta_1 * \beta_2)$, con 
                $* \in \{ \wedge, \vee, \to \}$, 
                $\beta_1, \beta_2 \in \mathrm{FORM}$

                \medskip

                Acá nos damos cuenta que necesitamos inducción completa
                pues $n+1 = \underbrace{long(\beta_1)}_{\geq 0} +
                \underbrace{long(\beta_2)}_{\geq 0} + 3$ 
                $\implies$ $\underbrace{long(\beta_1)}_{\geq 1} \leq n$, 
                $\underbrace{long(\beta_2)}_{\geq 1} \leq n$

                Y la suma de ambas longitudes es $n-2$, entonces $n+1$ puede
                tener una longitud menor estricta que $n$.
                

                Para arreglar esto, cambiamos la hipótesis inductiva del 
                siguiente modo:

                \begin{center}
                    \dashbox{
                    HI) $\alpha \in \mathrm{FORM}, long(\alpha) \; 
                    \boxed{\, \leq \,} \; n \implies$ existe una CF de 
                    $\alpha$ 
                    }
                \end{center}

                Y continuamos utilizando inducción completa.


                Como $\beta_1 \in \mathrm{FORM}$ y $long(\beta_1) \leq n$, por HI:
                \begin{gather*}
                    \exists \; X_1, \dotsc, X_r = \beta_1 \text{ CF} \\
                    \exists \; Y_1, \dotsc, Y_s = \beta_2 \text{ CF} \\
                \end{gather*}

                Defino: 
                \begin{gather*}
                    Z_1 = X_1, \dotsc, Z_r = X_r\\
                    Z_{r+1} = Y_1, \dotsc, Z_{r+s} = Y_s\\
                    Z_{r+s+1} = (Z_r * Z_{r+s}) = \alpha
                \end{gather*}
                
                Esto cumple con la definición de CF pues 
                (1) $Z_1 = X_1, \dotsc, Z_r = X_r$ es CF; 
                (2) $Z_{r+1} = Y_1, \dotsc, Z_{r+s} = Y_s$ es otra CF;
                y (3)
                $Z_{r+s+1} = (Z_r * Z_{r+s})$ 
                \nota{$\alpha = (\beta_1 * \beta_2)$}%
                es un conectivo binario, $*$, entre
                cadenas de formación definidas previamente.
                
        \end{enumerate}

        \item $\impliedby$) Sea $X_1, \dotsc, X_n$ una CF.
            Queremos probar que $X_n \in F$.

            \nota{$1 \leq i \leq n$}%
            Vamos a probar que $X_i \in F$ por inducción en $n$ 
            (es decir, en la longitud de la cadena de formación).

            \begin{itemize}
                \item CB) $n=1$ 

                    $X_1$ es una cadena de formación $\implies$
                    $X_1 \in \mathrm{VAR} \subseteq \mathrm{FORM}$

                \item HI) Sea $X_1, \dotsc, X_n$ CF
                    $\implies$ $X_n \in F$
                    
                \item T) Sea $X_1, \dotsc, X_{n+1}$ CF
                    $\implies$ $X_{n+1}\in F$
            \end{itemize}

            Sea $X_1, \dotsc, X_{n+1}$ una CF. Queremos probar que $X_{n+1}$
            es una fórmula.

            \begin{enumerate}[%
                labelindent=*,
                style=multiline,
                leftmargin=*,
                align=left,
                leftmargin=2\parindent,
                label=Caso \arabic*)]
                \item Si $X_{n+1} \in \mathrm{VAR} \subseteq \mathrm{FORM}$

                \item Si $X_{n+1} = \neg X_j$, con $j \leq n$, tenemos
                    que $X_1, \dotsc, X_j$ es una CF.

                    \medskip
                    \nota{Así permitimos cualquier longitud de cadena de 
                    formación menor o igual que $n$.}%
                    Acá nos damos cuenta que necesitamos inducción completa.

                    Nuevamente, cambiamos la hipótesis inductiva:

                    \begin{center}
                        \dashbox{
                            HI) Sea $X_1, \dotsc, X_k$ CF, con $k \leq n$ 
                            $\implies$ $X_k \in F$
                        }
                    \end{center}

                    \medskip
                    \nota{Por definición de CF.}%
                    Retomando, como tenemos que si $X_{n+1} = \neg X_j$,
                    con $j \leq n$ (es decir, con un eslabón anterior), 
                    $X_1, \dotsc, X_j$ es una 
                    CF.
                    \nota{Recordar la observación dada tras la definición de
                    \nameref{def:cf}}%

                    Entonces, por HI, $X_j \in \mathrm{FORM}$

                    Por lo tanto, por definición de fórmula,
                    $\neg X_j \in \mathrm{FORM}$

                \item Si $X_{n+1} = (X_j * X_t)$, con 
                    $* \in \{ \wedge, \vee, \to \}$ y $j, t \leq n$,
                    tenemos que:

                    \begin{align*}
                        X_1, \dotsc, X_{n+1} &
                        \text{ es una cadena de formación } \\
                        &\implies X_1, \dotsc, X_j
                        \text{ es una cadena de formación }
                        \notamath{$j \leq n < n+1$} \\
                        &\implies X_j \in \mathrm{FORM} \notamath{Por HI} \\
                        X_1, \dotsc, X_{n+1} &
                        \text{ es una cadena de formación } \\
                        &\implies X_1, \dotsc, X_t
                        \text{ es una cadena de formación } 
                        \notamath{$t \leq n < n+1$} \\
                        &\implies X_t \in \mathrm{FORM} \notamath{Por HI} \\
                    \end{align*}

                    Luego, por definición de $\mathrm{FORM}$:
                    \begin{gather*}
                        \underbrace{( X_j * X_k )}_{X_{n+1}} \in \mathrm{FORM}
                    \end{gather*}
            \end{enumerate}
    \end{itemize}
\end{proof}

\subsubsection{Ejemplos}

\begin{itemize}
    \item Sea $\alpha = ( \neg \neg p_1)$, ¿pertenece $E$ a la fórmula ($F$)?

    Supongamos que 
    $\alpha \in F \implies \exists \; X_1, \dotsc, X_n = \alpha$ 
    CF.
    
    Como $X_n$ empieza con `(', por la definición de cadena de formación,
    \begin{gather*}
        \exists \; j, k \leq n-1 / E = (X_j * X_k)
        \notamath{$* \in \{ \wedge, \vee, \to \}$}
    \end{gather*}
    
    ¡Lo cual es un absurdo!
    
    \begin{gather*}
        \therefore ~ E \notin F
    \end{gather*}

    \item $\alpha = (p_1$ no es una fórmula.

    Supongamos que lo es.

    Entonces existe una cadena de formación tal que
    $X_1, \dotsc, X_n = \alpha$.

    Por lo que $X_n$ es eslabón, y tiene tres opciones:
    \begin{enumerate}
        \item $X_n$ es una variable:
            si lo fuese, $X_n = p_k \neq ( p_1$ pues $( p_1$ empieza con
            $($ y $p_k$ no.
            Absurdo.

        \item $X_n = \neg X_i$, con $1 \leq i \leq n$.

            Luego $\neg X_i = ( p_1$ pero $(p_1$ no empieza con $\neg$.
            Absurdo.

        \item $X_n = (X_i * X_j)$, con $1 \leq i,j \leq n$.

            Pero $(X_i * X_j)$ termina con $)$ y $( p_1$ no.
            Absurdo.
    \end{enumerate}

    \begin{center}
        $\therefore ~ (p_1$ no es una fórmula.
    \end{center}

\end{itemize}

\subsection{Subcadena}

\begin{definicion}{Subcadena}{}
    Sea la cadena de formación $X_1, \dots X_n$.

    \medskip

    Decimos que
    $X_{i_1}, X_{i_2}, \dotsc, X_{i_k}$ es una subcadena si:
    \begin{enumerate}
        \item Es cadena de formación.
        \item $X_{i_k} = X_n$ \nota{``El último eslabón tiene que ser el 
            mismo''}%
        \item $1 \leq i_1 < i_2 < \dots < i_k = n$
    \end{enumerate}
\end{definicion}

\subsubsection{Ejemplo}

Sea la cadena de formación $X_1 = p_1$, $X_2 = p_3$, $X_3 = \neg X_2$ y
$X_4 = (X_2 \to X_3)$.

Notemos que toda cadena de formación es subcadena de sí misma.

Si ahora tomo $Y_1 = p_3$, $Y_2 = \neg Y_1$, $Y_3 = (Y_1 \to Y_2)$, esta es
una CF y una subcadena de la CF anterior que se obtiene quitando $X_1 = p_1$.

Como la segunda CF solamente tiene como subcadena a si misma, entonces decimos
que es una cadena de formación minimal.


\subsection{Cadena de formación minimal}

\begin{definicion}{Cadena de formación minimal}{}
    Una cadena de formación es minimal si la única subcadena que tiene es
    ella misma.
\end{definicion}

\medskip
\textit{Observación:}
\begin{enumerate}
    \item Toda fórmula tiene una cadena minimal.
    \item \nota{Ver la siguiente observación.}%
    \underline{Puede} tener más de una subcadena minimal. 
\end{enumerate}

\medskip
\textit{Observación:}

Notemos que ``minimal'' implica la existencia de una relación de orden. En las
cadenas de formación, esta relación de orden está definida como sigue:

\medskip

\begin{definicion}{Relación de orden en cadenas de formación}{}
Sean $A$, $B$ cadenas de formación.

\medskip

Se define $\mathcal{R}$ en el conjunto de cadenas de formación tal que
$A \mathcal{R} B$ si $A$ es subcadena de $B$.

Además, $\mathcal{R}$ es de orden.
\end{definicion}

\begin{proof}[Demostración de que $\mathcal{R}$ es de orden:] Tarea.
\end{proof}

\textit{Observación:}
Notemos que la relación no es de orden total.

Tomemos las cadenas de formación $A$ y $B$ que están definidas como:
\begin{align*}
    A:& \; X_1 = p_1, \; X_2 = p_2, \; X_3 = (X_1 \wedge X_2) \\
    B:& \; Y_1 = p_8, \; Y_2 = \neg Y_1
\end{align*}

Por la manera en que definimos a las CF, $A$ no es subcadena de $B$, y $B$ no 
es subcadena de $A$.

Entonces, si la relación es ser subcadena y $A$ y $B$ no se relacionan, la 
relación no es de orden total.


\subsubsection{Diferencia entre mínimo y minimal}

Sea $X$ un conjunto y $m \in X$.

$m$ es minimal si $\nexists \; a \in X / a < m$

Por otra parte, $m$ es mínimo si $m \leq b$, $\forall \, b \in X$

Análogamente podemos definir máximo y maximal.

\subsubsection{Ejemplo}

Sea $\alpha = \neg p_1$. Entonces $X_1 = p_1$ y $X_2 = \neg p_1$ es minimal 
pues:

Si $Y_1, \dotsc, Y_k$ es subcadena $\implies k \leq 2$ 
\nota{Pues $X_1$, $X_2$ tiene dos eslabones.}%
\begin{itemize}
    \item Si $k = 2$
        \begin{gather*}
            \implies Y_1 = X_1 ~ Y_2 = X_2 
            \implies \text{ es la misma cadena.}
        \end{gather*}
    \item Si $k = 1$
        
        $\implies Y_1 = X_2 = \neg p_1$ 
        \underline{no} es una cadena de formación pues no es variable 
        y no se consigue de eslabones anteriores.
\end{itemize}



\subsection{Complejidad}

\begin{definicion}{Complejidad}{}
    Sea $E \in A^{*}$.

    \medskip

    \begin{enumerate}
        \item La complejidad de $E$ es la cantidad de conectivos que 
            aparecen en $E$.

            \bigskip
            \textbf{Notación:} \phantom{b}$c(E) =$ complejidad de $E$.
        \item La complejidad binaria de $E$ es la cantidad de conectivos
            binarios que aparecen en $E$.

            \bigskip
            \textbf{Notación:} $cb(E) =$ complejidad binaria de $E$.
    \end{enumerate}
\end{definicion}

\bigskip
\textit{Observación:}
$c(p_k) = 0$

\bigskip
\textit{Observación:}
\nota{``$\beta$ es una \nameref{def:subformula} de $\alpha$''.}%
Si $\beta \in S(\alpha) \implies c(\beta) \leq c(\alpha)$

\begin{proof}
    Tarea. Sale por inducción.
\end{proof}

\subsubsection{Ejemplos}

\begin{itemize}
    \item Sea $E = ( \; ) \wedge \to p_1 \wedge p_2$
        \begin{align*}
            c(E) &= 3 \\
            cb(E) &= 3
        \end{align*}

    \item Sea $\alpha = \neg (p_1 \to p_2)$
        \begin{align*}
            c(E) &= 2 \\
            cb(E) &= 1
        \end{align*}

    \item Sea $\alpha = (((p_1 \vee p_2) \to \neg p_3) \vee p_2)$
        \begin{align*}
            c(E) &= 4 \\
            cb(E) &= 3
        \end{align*}
\end{itemize}

\subsection{Peso}

\begin{definicion}{Peso}{}
    Dada una expresión $E \in A^{*}$.

    \medskip

    Definimos el peso de $E$ como la cantidad de paréntesis que abren menos
    la cantidad de paréntesis que cierran.

    \bigskip
    \textbf{Notación:}
    $peso(E) = p(E) =$ peso de $E$
\end{definicion}

\subsubsection{Ejemplo}

\begin{itemize}
    \item Si $E = ( p_1 \wedge p_2))) \to p_8$, entonces $p(E)=1-3=-2$

    \item Si tenemos $\alpha = ((p_1 \to p_2) \vee p_3)$, entonces
        $p(\alpha) = 2-2=0$
\end{itemize}

\begin{lema}{}{}
    Sea $\alpha \in F$.

    \medskip

    \begin{enumerate}
        \item $c(\alpha) = 0 \implies \alpha \in \mathrm{VAR}$
        \item $c(\alpha) > 0 \implies \alpha = \neg \beta$ ó
            $\alpha = (\beta_1 * \beta_2)$
            \nota{Con $\beta \in F$\\
                Con $\beta_1, \beta_2 \in F$ \\
                $*\in \{ \wedge,\vee,\to \}$}%
    \end{enumerate}
\end{lema}

\begin{proof} \phantom{.}
    % Tarea.

    % \nota{Más que darnos una ``pista'', Noni demostró el lema oralmente.}%
    % \textit{Pista:} Si la complejidad de $\alpha$ es cero, quiere decir que no
    % tiene conectivos. La única fórmula que no tiene conectivos es una
    % variable.
    % 
    % Si $\alpha$ es una fórmula, entonces existe una CF de $\alpha$ y, 
    % si además no tiene conectivos, entonces no la pude haber armado a partir
    % de eslabones anteriores. Entonces tiene que ser una variable.


    % Si $\alpha$ tiene un conectivo o la negación, como es una fórmula, 
    % entonces existe una CF de $\alpha$, el último eslabón es una variable, una
    % negación de alguno anterior, o el $\wedge$, $\vee$ o $\to$ de un eslabón
    % previo.

    % Como $c(\alpha)>0$, entonces el caso de la variable no tiene sentido y
    % caemos en los otros dos casos.

    \begin{enumerate}
        \item Como $\alpha \in \mathrm{FORM}$, existe $X_1, \dotsc, X_n$
            cadena de formación tal que 
            $X_n = \alpha$ y $c(\alpha) = c(X_1) = 0$.

            $X_n$ tiene 3 opciones, veamos que sólo puede ser una variable:
            \begin{enumerate}
                \item[2)] $X_n = \neg X_i$ \nota{$1 \leq i \leq n$}%
                    \begin{gather*}
                        0 = c(x_n) = c(\neg X_i) \geq 1
                    \end{gather*}

                    ¡Absurdo!

                \item[3)] $X_n = (X_i * X_j)$ \nota{$1 \leq i, j < n$}%
                    \begin{gather*}
                        0 = c(X_n) = c((X_i * X_j)) \geq 1
                    \end{gather*}

                    ¡Absurdo!
            \end{enumerate}

            La única opción posible es 1) $X_n = p_k \in \mathrm{VAR}$ y
            entonces $\alpha = X_n = p_k$.

        \item Tarea.
    \end{enumerate}
\end{proof}

\subsection{Teorema}

\begin{teorema}{}{peso-formula}
    Sea $\alpha \in F$.

    \medskip

    Entonces:
    \begin{enumerate}
        \item $p(\alpha)= 0$
        \item Si $\alpha = E * F$ entonces $p(E)>0$
            \nota{$* \in \{ \wedge, \vee, \to \}$}%
    \end{enumerate}
\end{teorema}

\subsubsection{Ejemplo}

Si tengo $\alpha = \underbrace{((p_1 \to p_2) \vee (p_3}_{E} \bullet p_4))$, 
entonces $p(E) = 3-1 = 2>0$


\begin{proof}[Demostración del Teorema \ref{teo:peso-formula}:] \phantom{.}

    Lo vamos a demostrar por inducción en $c(\alpha)$.

    Sea $\alpha \in F$.

    \begin{itemize}
        \item CB) 
        \begin{enumerate}
            \item $c(\alpha) = 0 \implies \alpha \in \mathrm{VAR}$, entonces 
            $p(\alpha)=0$
            \item Es verdadero por antecedente falso.
        \end{enumerate}

        Para escribir menos, vamos a llamar a ambos casos del teorema 
        $\mathcal{P}(n)$ y vamos a decir que $c(\alpha)=n$. 
        Además, para evitar abuso de notación, al conectivo binario del caso
        2 lo notaremos como $\bullet$.

        \item HI) $\mathcal{P}(k)$, $k \leq n$
        \item T) $\mathcal{P}(n+1)$
    \end{itemize}       

    Sea $\alpha \in F / c(\alpha) =  n+1 > 0$

    \begin{enumerate}[%
                labelindent=*,
                style=multiline,
                leftmargin=*,
                align=left,
                leftmargin=2\parindent,
                label=Caso \arabic*)]
        \item $\alpha = \neg \beta$, con $\beta \in F$

            Además, $c(\beta) = c(\alpha) -1 = n$

            \begin{enumerate}
                \item Por HI, $p(\beta)=0 \implies p(\alpha) 
                    = p(\neg \beta) = 0$

                    \item Sea $\bullet$ un conectivo binario que aparece
                        en $\alpha$ $\implies$ $\bullet$ está en $\beta$

                Entonces, por HI, la expresión a la izquierda de $\bullet$
                en $\beta$, $E$, tiene peso positivo.

                Si $\widetilde{E}$ es la expresión a la izquierda de $\bullet$ en
                $\alpha$ $\implies \widetilde{E} = \neg E$

                Luego
                \begin{gather*}
                    p(\widetilde{E}) = \underbrace{p(\neg)}_{=0} + 
                    \underbrace{p(E)}_{>0} > 0
                \end{gather*}
            \end{enumerate}
            

        \item $\alpha = ( \beta_1 * \beta_2)$, con 
            $* \in \{ \wedge, \vee, \to \}$, $\beta_1, \beta_2 \in F$

            \begin{gather*}
                \underbrace{c(\beta_1)}_{\geq 0} +
                \underbrace{c(\beta_2)}_{\geq 0} = c(\alpha) - 1 = n \\
                \implies c(\beta_1) \leq n ~ \wedge ~ c(\beta_2) \leq n
            \end{gather*}


            \begin{enumerate}
            \item Por HI, $p(\beta_1) = 0$ y $p(\beta_2) = 0$

            Luego, $p(\alpha) = p(\; (\beta_1 * \beta_2) \;)
            = \underbrace{1}_{(} 
            + \underbrace{p(\beta_1)}_{=0} 
            + \underbrace{p(\beta_2)}_{=0} 
            - \underbrace{1}_{)} = 0$

            \item Sea $\bullet$ un conectivo binario que aparece en $\alpha$.
            \begin{enumerate}
            \item $\bullet$ aparece en $\beta_1$ $\implies$ Por HI la 
                expresión $E$ a la izquierda de $\bullet$ en $\beta$ tiene
                peso mayor a cero.

                $\widetilde{E} = ( \, E$ $\quad$
                es la expresión a la izquierda de $\bullet$ en $\alpha$
                \begin{gather*}
                    \implies p(\widetilde{E}) = 1 + \underbrace{p(E)}_{>0}>0 
                \end{gather*}

            \item $\bullet = *$

                La expresión a la izquierda de $\bullet$ en $\alpha$ es
                $\widetilde{E} = ( \beta_1$
                \begin{gather*}
                    \implies p(\widetilde{E}) = 1 + 
                        \underbrace{p(\beta_1)}_{=0} > 0 
                        \notamath{Pues $\beta_1 \in F$}
                \end{gather*}

            \item $\bullet$ aparece en $\beta_2$

                Por HI, la expresión a la izquierda de $\bullet$ en $\beta_2$
                es $E$ y $p(E)>0$

                $\widetilde{E} = ( \; \beta_1 * E$ es la expresión a la izquierda
                de $\bullet$ en $\alpha$

                \begin{gather*}
                    \implies p(\widetilde{E}) = 1 
                    + \underbrace{p(\beta_1)}_{=0} + \underbrace{p(*)}_{=0}
                    + \underbrace{p(E)}_{>0} > 0
                \end{gather*}
            \end{enumerate}
            \end{enumerate}
    \end{enumerate}
\end{proof}


\begin{corolario}{Unicidad de escritura}{}
       Sea $\alpha \in F$.

       \medskip

       \begin{align*}
           c(\alpha)>0 \implies& \exists ! \; \beta \in F / \alpha 
           = \neg \beta \\
           \text{o }& \exists! \; \beta_1,\beta_2 \in F, 
           *\in\{\wedge,\vee,\to\} /
           \alpha = (\beta_1 * \beta_2) \notamath{Únicos $\beta_1$,
           $\beta_2$ y un único conectivo $*$ }
       \end{align*}
\end{corolario}

Dicho de otra manera:
\begin{align*}
    (\beta_1 * \beta_2 ) = (\gamma_1 \circ \gamma_2) 
    \implies& \beta_1 = \gamma_1, ~
    \beta_2 = \gamma_2 ~ \text{y} ~
    * = \circ \\
    \neg \beta_1 = \neg \beta_2 \implies& \beta_1 = \beta_2
\end{align*}

\begin{proof} \phantom{.}

    \begin{enumerate}
        \item Supongo $\alpha = \neg \beta_1$ y $\alpha = \neg \beta_2$, con
            $\beta_1, \beta_2 \in F$

            \begin{gather*}
                \neg \beta_1 = \neg \beta_2 \implies \beta_1 = \beta_2
            \end{gather*}

        \item Supongo $\alpha = ( \beta_1 * \beta_2)$, 
            $\alpha = (\gamma_1 \circ \gamma_2)$, con 
            $*, \circ \in \{ \wedge,\vee,\to \}$, $\beta_1, \beta_2, 
            \gamma_1, \gamma_2 \in F$

            \begin{gather*}
                (\beta_1*\beta_2) = (\gamma_1 \circ \gamma_2)
                \implies \beta_1 * \beta_2 = \gamma_1 \circ \gamma_2
            \end{gather*}

            \begin{enumerate}
                \item Supongamos que $long(\beta_1) = long(\gamma_1)$.

                    Entonces, por el Corolario \ref{corol:alfabeto-igual-long},
                    \begin{gather*}
                        \beta_1 = \gamma_1
                        ~ \text{ y } ~
                        *\beta_2 = \circ \gamma_2
                        \implies \beta_1 = \gamma_1,
                        ~ * = \circ
                        ~ \text{ y } ~
                        \beta_2 = \gamma_2
                    \end{gather*}
                    
                \item Supongamos $long(\beta_1) > long(\gamma_1)$.

                    Entonces $\exists \; H'\in A^{*} / \beta_1 = \gamma_1 H'$.

                    Como las longitudes son mayores estrictas, entonces $H'$
                    tiene al menos un caracter y, en consecuencia, 
                    $H'=\circ H''$, con $H'' \in A^{*}$.

                    Como $\beta_1 \in F$, la expresión a la izquierda de 
                    $\circ$ en $\beta_1$ tiene peso positivo.
                    \begin{gather*}
                        p(\gamma_1) > 0
                    \end{gather*}

                    ¡Absurdo!
                    Pues $\gamma_1 \in F$

                \item $long(\beta_1) < long(\gamma_1)$

                    Tarea. Análogo al caso anterior.
            \end{enumerate}

        \item $\alpha = \neg \beta$, $\alpha = (\beta_1 * \beta_2)$, con
            $\beta, \beta_1, \beta_2 \in F$, $*\in \{ \wedge,\vee,\to \}$

            \begin{gather*}
                \neg \beta = (\beta_1 * \beta_2)
            \end{gather*}

            Lo cual es absurdo pues las expresiones no coinciden en el primer
            caracter.
    \end{enumerate}
\end{proof}


\subsection{Subfórmula}

\begin{definicion}{Subfórmula}{subformula}
    Sea $\alpha \in F$.

    \medskip

    \begin{itemize}
        \item \nota{$j \in \mathbb{N}$}%
            Si $c(\alpha) = 0 \implies \alpha = p_j $

            En este caso, el conjunto de subfórmulas de $\alpha$ es:
            \begin{gather*}
                S(\alpha)=\{ p_j \}
            \end{gather*}

        \item Si $c(\alpha) > 0$
            \begin{enumerate}
                \item $\alpha = \neg \beta$
                    \nota{$\beta \in F$}%
                    \begin{gather*}
                        S(\alpha) = \{ \alpha \} \cup S(\beta)
                    \end{gather*}
                \item \nota{$\beta_1, \beta_2 \in F$\\
                    $* \in \{ \wedge, \vee, \to \}$}%
                    $\alpha = (\beta_1 * \beta_2)$
                    \begin{gather*}
                        S(\alpha)=\{ \alpha \}\cup S(\beta_1) \cup S(\beta_2)
                    \end{gather*}
            \end{enumerate}
    \end{itemize}

    \bigskip
    \textbf{Notación:}
    $S(\alpha)$ es el conjunto de subfórmulas de $\alpha$.
\end{definicion}

\bigskip
\textit{¡Atención!}
\begin{gather*}
    \# S((\beta_1 * \beta_2)) \leq \# S(\beta_1) + \# S(\beta_2)
\end{gather*}

\subsubsection{Ejemplos}

\begin{itemize}
    \item Sea $\alpha = ((p_1 \wedge p_2) \vee p_3)$

    \begin{align*}
        S(\alpha) &= \{\alpha\} \cup S((p_1 \wedge p_2)) \cup S(p_3) \\
                  &= \{\alpha\} \cup \{p_1 \wedge p_2\} \cup S(p_1) 
                  \cup S(p_2) \cup \{p_3\} \\
                  &= \{\alpha\} \cup \{p_1 \wedge p_2\} \cup \{p_1\} 
                  \cup \{p_2\} \cup \{p_3\} \\
                  &= \{ \alpha, (p_1 \wedge p_2), p_1, p_2, p_3 \}
    \end{align*}

    \item Sea $\alpha = (\overbrace{(p_1 \wedge p_2)}^{\beta_1} 
        \to \overbrace{p_3}^{\beta_2})$

        \begin{align*}
            S(\alpha) &= \{ \alpha \} \cup S(\beta_1) \cup S(\beta_2) \\
            &= \{ \alpha \} \cup \{ \beta_1 \} 
            \cup \underbrace{S(p_1)}_{\{ p_1 \}} 
            \cup \underbrace{S(p_2)}_{\{ p_2 \}} 
            \cup \{ \beta_2 \} \cup \underbrace{S(p_3)}_{\{ p_3 \}}
        \end{align*}

        Entonces
        \begin{gather*}
            S(\alpha) = \{\alpha, (p_1\wedge p_2), p_1, p_2,\neg p_3, p_3 \}
        \end{gather*}

    \item Sea $\beta \in S(\alpha)$, queremos ver que $\beta$ es un eslabón de
    una cadena de $\alpha$.

    \medskip
    Lo vemos por inducción en la complejidad de $\alpha$.

    \begin{itemize}
        \item CB) $c(\alpha) = 0$

            \begin{align*}
                \implies& \alpha = p_k \in \mathrm{VAR}
                \nota{x Propiedad} \\
                \implies& \beta \in S(\alpha) = \{ p_k \} \\
                \implies& \beta = p_k = \alpha
            \end{align*}

        \item HI)
            Si $\alpha \in F$ tal que
            $c(\alpha) \leq k$
            y
            $\beta \in S(\alpha)$
            $\implies \beta$ aparece como eslabón en la cadena de $\alpha$.
        \item TI)
            Dada $\alpha \in F$ con
            $c(\alpha) = k + 1$
            y
            $\beta \in S(\alpha)$.

            \nota{$c(\alpha) > 1$}%
            Queremos ver que $\beta$ aparece como eslabón de la cadena de
            $\alpha$.

            \begin{enumerate}[%
                            labelindent=*,
                            style=multiline,
                            leftmargin=*,
                            align=left,
                            leftmargin=2\parindent,
                            label=Caso \arabic*)]
                \item $\alpha = (\beta_1 * \beta_2)$
                    \nota{$\beta_1, \beta_2 \in \mathrm{FORM}$}%

                    \begin{align*}
                        k + 1 =& c(\alpha) = c(\beta_1) + c(\beta_2) + 1 \\
                        \implies& c(\beta_1), c(\beta_2) \leq k \\
                        \implies& \beta \in S(\alpha) =
                        \{ \alpha \} \cup S(\beta_1) \cup S(\beta_2)
                    \end{align*}

                    \begin{itemize}
                        \item Si $\beta = \alpha$ $\implies$
                            Por el mismo argumento del CB, $\beta$ está en
                            toda cadena de formación.
                        \item Si $\beta \neq \alpha$ $\implies$
                            $\beta \in S(\beta_1)$ ó $\beta \in S(\beta_2)$

                            Supongamos el primer caso, pues el segundo es
                            análogo.

                            Por HI, $\beta$ aparece en toda cadena de
                            formación de $\beta_1$.

                            Sea $X_1, \dotsc, X_n$ CF de $\alpha$.

                            Entonces $\alpha = X_n = (\beta_1 * \beta_2)$.
                            Como es cadena, entonces el único caso posible
                            es $X_n = (X_i \cdot X_j)$ \nota{$i, j < n$}%

                            Luego, por unicidad de escritura, $X_i = \beta_1$,
                            $X_j = \beta_2$ y $\cdot = *$.
                            Entonces $X_1, \dotsc, X_i = \beta_1$ es cadena
                            $\implies$ $\beta = X_l$ \nota{$1 \leq l \leq i$}%
                            $\implies$ $\beta$ aparece en la cadena de
                            $\alpha$.
                    \end{itemize}
            \end{enumerate}
    \end{itemize}

\end{itemize}

\pagebreak
\section{Semántica}

\subsection{Valuaciones}


\begin{definicion}{Valuación}{}
    Una \textit{valuación} (o \textit{asignación}) es una función 
    $v: \mathrm{FORM} \to \{ 0,1 \}$ que verifica:

    \begin{enumerate}[label=\protect\circled{\arabic*}]
        \item \nota{$\forall \alpha \in F$}%
            $v (\neg \alpha) = 1 - v (\alpha)$
        \item \nota{$\forall \alpha, \beta \in F$}%
            $v(\alpha \wedge \beta) = \min \{ v(\alpha), v(\beta) \}$
        \item \nota{$\forall \alpha, \beta \in F$}%
            $v(\alpha \vee \beta) 
            = \max \{ v(\alpha), v(\beta) \}$
        \item \nota{$\forall \alpha, \beta \in F$}%
            $v (\alpha \to \beta) 
            = \max\{ 1-v(\alpha), v(\beta) \}$
    \end{enumerate}
\end{definicion}

\subsubsection{Ejemplo}
$v: \mathrm{FORM} \to \{ 0,1 \} / v(\alpha) = 1$ 

No es una valuación, pues $v(p_1)=1$ y $v(\neg p_1) = 1$
\begin{gather*}
    \implies \text{ no verifica } v(\neg \alpha)=1-v(\alpha)
\end{gather*}

\subsection{Teorema}

\begin{teorema}{}{valuacion-unica}
    Dada $f: \mathrm{VAR}\to \{ 0,1 \}$ función.

    \medskip

    Existe una única valuación
    $v_f: \mathrm{FORM} \to \{ 0,1 \}$ que extiende a $f$.
\end{teorema}

Esto significa que $\mathrm{VAR} \subseteq \mathrm{FORM}$ y $v$ restringido
a las variables es igual a $f$:
\begin{gather*}
    \frest{v}{\mathrm{VAR}} = f
\end{gather*}

Es decir, que $v_f(p_j) = f(p_j)$, con $p_j \in \mathrm{VAR}$.


\begin{proof} \phantom{.}

    Sea $F_m = \{\alpha \in \mathrm{FORM} / c(\alpha) \leq m\}$,
    con $m \in \mathbb{N}$

    Luego $F_0 = \mathrm{VAR}$,
    $F_1 = \mathrm{VAR} \cup \{ \alpha \in F / c(\alpha) = 1 \}$, y
    así sucesivamente.
    Es decir, estamos definiendo $(F_m)_{m \in \mathbb{N}}$ una sucesión
    creciente de conjuntos de fórmulas:
    $F_0 \subseteq F_1 \subseteq F_2 \subseteq \dots$.

    \bigskip

    Veamos por inducción en $m$ que se cumple
    $\mathcal{P}(m)$ para todo $m \in \mathbb{N}$.

    Para ello, definamos $\mathcal{P}(m)$: 
    Existe una única función 
    $v_m: F_m \to \{ 0,1 \}$ tal que $\frest{v_m}{\mathrm{VAR}} = f$ y 
    verifica los cuatro items de la definición de valuación.

    \begin{itemize}
        \item CB) Defino 
            $v_0: \underbrace{F_0}_{\substack{%
            \text{Conjunto} \\ \text{ de } \mathrm{VARS}}} 
            \to \{ 0,1 \} / v_0 (p_j) = f(p_j)$
        \item HI) $\mathcal{P}(k)$, $k \leq m$
        \item T) $\mathcal{P}(m+1)$
     \end{itemize}


    Defino $v_{m+1}: F_{m+1} \to \{ 0,1 \}$.

    Supongamos que tenemos $\alpha \in F_{m+1}/c(\alpha) \leq m: 
    v_{m+1} (\alpha) = v_m (\alpha)$

    Si $\alpha \in F_{m+1}/c(\alpha) = m+1$, tenemos cuatro posibilidades:
    \begin{enumerate}
        \item $\alpha = \neg \beta \implies c(\beta) = m$
            \begin{align*}
                v_{m+1} (\neg \beta) &= 1 - v_{m+1}(\beta) 
                \notamath{Cumple con \circled{1}} \\
                &= 1 - v_m (\beta) \notamath{$c(\beta)=m$}
            \end{align*}

            Notemos que esto extiende a $F$, pues $v_m$ lo hace.

            Cumple las 4 propiedades de valuación porque en la primer igualdad
            lo está haciendo y $v_m$ lo hace por HI.

            También es única la manera de definirla, ya que la primer igualdad
            es única, sino no cumpliría la definición de valuación; en el
            segundo igual, si no cumpliera habría otra forma de definir
            $v_m$ y no estaríamos cumpliendo con la HI.

            \item $\alpha = (\beta_1 \wedge \beta_2)$
                \begin{align*}
                    v_{m+1} (\beta_1 \wedge \beta_2) &= 
                    \min \{ v_{m+1} (\beta_1), v_{m+1}(\beta_2)\}
                    \notamath{Cumple con \circled{2}} \\
                    &= \min \{ v_m(\beta_1), v_m(\beta_2) \}
                    \notamath{Por HI y porque \\ $c(\beta_1 \wedge \beta_2)=$\\
                    $=\overbrace{c(\beta_1)}^{\geq 0} + 
                    \overbrace{c(\beta_2)}^{\geq 0}+1=$ $=m+1$ \\ 
                $c(\beta_i)\leq m$, $i\in \{ 1,2 \}$}
                \end{align*}

                El argumento de validez y unicidad de este caso es análogo el
                anterior.

            \item $\alpha = (\beta_1 \vee \beta_2)$ 

                Tarea.
            \item $\alpha = (\beta_1 \to \beta_2)$

                Tarea.
    \end{enumerate}
        
    Para completar la demostración del teorema, definamos 
    
    \[ v: \mathrm{FORM} \to \{0,1\}: 
        v(\alpha) \underbrace{=}_{c(\alpha)=m} v_m(\alpha) \]
        
    $v(\alpha)$ es una valuación pues extiende a $F$ ya que $v_m$ lo hace 
    $\forall m \in \mathbb{N}$.

    Además cumple las 4 propiedades de valuación pues $v_m$ las cumple 
    $\forall m \in \mathbb{N}$ y, por último, es la única función que cumple 
    las cuatro propiedades y extiende a $F$ pues, en caso contrario, habría 
    otra manera de definir $v_m(\alpha)$ y acabamos de probar que
    es única.

\end{proof}


\subsubsection{Ejemplo}

Sea $f: \mathrm{VAR} \to \{ 0, 1 \}$ tal que
$f(p_k) = \begin{cases}
    1 & \text{si } k \text{ es par} \\
    0 & \text{si } k \text{ es impar}
\end{cases}$

\medskip
Entonces, por el teorema, existe una única $v_f: \mathrm{FORM} \to \{ 0, 1 \}$
tal que $v_f$ extiende a $f$.

\begin{gather*}
    v_f (p_1 \to (\neg p2 \vee p_3)) =
    \max{
        \{ \underbrace{1 - v_f (p_1)}_{= f(p_1) = 0},
        v_f (\neg p_2 \vee p_3) \}
    }
    = 1
\end{gather*}

\subsection{Clasificación semántica de las fórmulas}

\begin{definicion}{Clasificación semántica de las fórmulas}{}
    Sea $\alpha \in F.$

    \begin{enumerate}
        \item $\alpha$ es \textit{tautología} si $v(\alpha)=1$
            $\forall v$ valuación.
        \item $\alpha$ es \textit{contradicción} si $v(\alpha)=0$
            $\forall v$ valuación.
        \item $\alpha$ es \textit{contingencia} si:
            \begin{itemize}
                \item $\exists \; v \text{ valuación}/ v(\alpha) = 1$
                \item $\exists \; w \text{ valuacíon}/ w(\alpha)=0$
            \end{itemize}
    \end{enumerate}
\end{definicion}

\subsubsection{Ejemplos}

Clasificar las siguientes fórmulas:

\begin{enumerate}
    \item $\alpha = (p_1 \wedge \neg p_1)$ es una contradicción.

        Tenemos que probar que es una contradicción 
        \textbf{para toda valuación}. No podemos definir una única valuación.

        \begin{proof} \phantom{.}
        
            Sea $v$ una valuación.

            \begin{align*}
                v(\alpha) &= v(p_1 \wedge \neg p_1) \\
                &= \min \{ v(p_1), v(\neg p_1) \} \\
                &= \min \{ v(p_1), 1-v(p_1) \}
            \end{align*}

            \begin{enumerate}[%
                labelindent=*,
                style=multiline,
                leftmargin=*,
                align=left,
                leftmargin=2\parindent,
                label=Caso \arabic*)]
                \item $v(p_1) = 1 \implies v(\alpha) = \min \{ 1,0 \} = 0$
                \item $v(p_1) = 0 \implies v(\alpha)=\min \{ 0,1 \} = 0$
            \end{enumerate}
            \begin{gather*}
                \therefore ~ v(\alpha) = 0 ~ \forall v \text{ valuación}
            \end{gather*}
            
        \end{proof}
        
    \item $\alpha = (p_1 \wedge p_2)$ es una contingencia.

        Como en este caso tenemos que probar que existen dos valuaciones,
        una tal que $v(\alpha)=1$ y otra $w(\alpha)=0$, tenemos que 
        definirlas. Para ello nos ayuda el Teorema \ref{teo:valuacion-unica}


        \begin{proof} \phantom{.}
        
            \begin{enumerate}
                \item Defino $f: \mathrm{VAR} \to \{ 0,1 \}/ f(p_j) = 1$, $\forall j$.
            Sea $v_f$ la única valuación que extiende a $f$.

            \begin{align*}
                v_f(\alpha) &= \min \{ v_f(p_1), v_f(p_2) \}
                \notamath{Esto nos asegura que $\alpha$ \textit{no} es una 
                contradicción.} \\
                &= \min \{ \underbrace{f(p_1)}_{=1}, 
                \underbrace{f(p_2)}_{=1} \} \\
                &= 1 \\
            \end{align*}
            \begin{gather*}
                \therefore ~ v_f(\alpha)=1
            \end{gather*}

            \item Defino $g: \mathrm{VAR} \to \{ 0,1 \}/ g(p_j) = 0$, $\forall j$.

            Sea $v_g$ la única valuación que extiende a $g$.

            \begin{align*}
                v_g(\alpha) &= \min \{ v_g(p_1), v_g(p_2) \} 
                \notamath{Análogamente, esto nos asegura que $\alpha$ 
                    \textit{no} es una tautología.}\\
                &= \min \{ \underbrace{g(p_1)}_{=0},
                \underbrace{g(p_2)}_{=0} \} \\
                &= 0 \\
            \end{align*}
            \begin{gather*}
                \therefore ~ v_g(\alpha)=0
            \end{gather*}
            \end{enumerate}

            Por lo tanto, por 1 y 2, $\alpha$ es una contingencia.
        \end{proof}
    \item $\alpha = (p_1 \vee \neg p_1)$ es una tautología.

        \begin{proof} \phantom{.}
        
            Sea $v$ valuación.

            \begin{align*}
                v(\alpha) &= \max{\{ v(p_1), v(\neg p_1) \}} \\
                          &= \max{\{ v(p_i), 1-v(p_i) \}}
            \end{align*}

            \begin{enumerate}[%
                            labelindent=*,
                            style=multiline,
                            leftmargin=*,
                            align=left,
                            leftmargin=2\parindent,
                            label=Caso \arabic*)]
                \item Si $v(p_1) = 1 \implies v(\alpha) = \max{\{ 1,0 \}}=1$
                \item Si $v(p_1) = 0 \implies v(\alpha) = \max{\{ 1,0 \}}=1$
            \end{enumerate}
            \begin{gather*}
                \therefore ~  v(\alpha) = 1 ~ \forall v \text{ valuación}
            \end{gather*}
        \end{proof}

    \item $\alpha = (p_1 \to \neg p_1)$
        \begin{itemize}
            \item Si $v(p_1) = 0$ $\implies$
                $v(\alpha) = \max{\{ 1 - v(p_1), 1 - v(p_1) \}} = 1$
            \item Si $v(p_1) = 1$ $\implies$
                $v(\alpha) = \max{\{ 1 - v(p_1), 1 - v(p_1) \}} = 0$
        \end{itemize}

        \textit{Santiago:} \textbf{¡Cuidado!}
        Tenemos un condicional, \textit{``si''}.
        Para afirmar que es contingencia la valuación tiene que existir,
        y no \textit{``existir condicionada''}.

        Definamos $f: \mathrm{VAR} \to \{ 0, 1 \} / f(p_i) = \begin{cases}
            0 & i = 1 \\
            0 & \text{sino}
        \end{cases}$.
        Sea $v_f$ la valuación que extiende a $f$.

        Entonces $v_f(\alpha) \underbrace{=}_{v_f(p_1) = f(p_1) = 0} 1$

        Luego, definamos
        $g: \mathrm{VAR} \to \{ 0, 1 \} / g(p_i) = \begin{cases}
            1 & i = 1 \\
            0 & \text{sino}
        \end{cases}$.
        Sea $v_g$ la valuación que extiende a $g$.

        Entonces $v_g(\alpha) \underbrace{=}_{v_g(p_1) = g(p_1) = 0} 0$

        \begin{center}
            $\therefore ~ \alpha$ es contingencia.
        \end{center}
\end{enumerate} 


\subsection{Teorema}

\begin{teorema}{Otra forma del Teorema \ref{teo:valuacion-unica}}{igualdad-valuaciones-frest}
    \nota{$var(\alpha)$ es el conjunto de variables de $\alpha$}%
    Sea $\alpha \in F$. 

    Sea $var(\alpha) = 
    \{ p_j \in \mathrm{VAR} / p_j \text{ aparece en } \alpha \}$

    \medskip

    Si $v, w$ son valuaciones tales que 
    $\frest{v}{var(\alpha)} = \frest{w}{var(\alpha)}$
    \begin{gather*}
        \implies v(\alpha) = w(\alpha)
    \end{gather*}

\end{teorema}

\begin{proof} \phantom{.}

    Por inducción en $c(\alpha)$.

    \begin{itemize}
        \item CB) Sea $\alpha \in F/c(\alpha) = 0$.
            \begin{gather*}
                \implies \alpha = p_j
            \end{gather*}

            Si tenemos $v, w$ valuaciones tales que 
            $\frest{v}{var(\alpha)} = \frest{w}{var(\alpha)}$

            \begin{align*}
                \implies v(p_j) &= w(p_j) \\
                \implies v(\alpha) &= w(\alpha)
            \end{align*}

        \item HI) Supongamos que $\alpha \in F/ c(\alpha) = k$.

            Sean $v, w$ valuaciones tales que 
            $\frest{v}{var(\alpha)} = \frest{w}{var(\alpha)}$ 
            $\implies v(\alpha) = w(\alpha)$

            \medskip

            Por simplicidad, vamos a llamar a lo que acabamos de escribir
            $\mathcal{P}(k)$.

            Así, la hipótesis inductiva es $\mathcal{P}(k)$, con $k \leq n$.

        \item T) $\mathcal{P}(n+1)$
    \end{itemize}

    Sea $\alpha \in F/ c(\alpha) = n+1 > 0$

    \begin{enumerate}[%
                labelindent=*,
                style=multiline,
                leftmargin=*,
                align=left,
                leftmargin=2\parindent,
                label=Caso \arabic*)]
        \item $\alpha = \neg \beta$

            \begin{gather*}
                n+1 = c(\alpha) = 1+c(\beta) \implies \dashbox{$c(\beta)=n$}
            \end{gather*}

            Sean $v$ y $w$ valuaciones tales que 
            $\frest{v}{var(\alpha)}=\frest{w}{var(\alpha)}$

            Como $var(\alpha) = var(\beta)$, entonces 
            \dashbox{$\frest{v}{var(\beta)}=\frest{w}{var(\beta)}$}

            Por esto, y por la HI: 
            \begin{align*}
                v(\beta) = w(\beta)& \\
                \implies& 1 - v(\beta) 
                    = 1 - w(\beta) \\
                \implies& \boxed{ v(\alpha) = w(\alpha) }
            \end{align*}

    \item $\alpha = (\beta_1 \vee \beta_2)$

    \begin{align*}
        c(\alpha) =& \, n + 1 \\
        =& \, 1 + \underbrace{c(\beta_1)}_{\geq 0} 
        + \underbrace{c(\beta_2)}_{\geq 0} \\
        \implies& c(\beta_1) + c(\beta) = n \\
        \implies& \dashbox{$c(\beta_1) \leq n \text{ y } c(\beta_2) \leq n$}
    \end{align*}

    Notemos que $var(\alpha) = var(\beta_1) \cup var(\beta_2)$

    \medskip

    Luego, sean $v$ y $w$ valuaciones tales que 
    $\frest{v}{var(\alpha)}=\frest{w}{var(\alpha)}$

    Como $var(\beta_1) \subseteq var(\alpha)$ $\implies$ 
        \dashbox{$\frest{v}{var(\beta_1)}=\frest{w}{var(\beta_1)}$}

    Como $var(\beta_2) \subseteq var(\alpha)$ $\implies$ 
        \dashbox{$\frest{v}{var(\beta_2)}=\frest{w}{var(\beta_2)}$}


    Entonces, por HI, $v(\beta_1) = w(\beta_1)$ y $v(\beta_2) = w(\beta_2)$

    Resultando:
    \begin{align*}
        v(\alpha) 
            &= \min \{ v(\beta_1), v(\beta_2) \} \\
            &= \min \{ w(\beta_2), w(\beta_2)\} \\
        \implies &\boxed{v(\alpha) = w(\alpha)}
    \end{align*}

    \item $\alpha = (\beta_1 \wedge \beta_2)$

        Tarea. Análogo a los casos anteriores.

    \item $\alpha=(\beta_1 \to \beta_2)$

        Tarea. Análogo a los casos anteriores.
    \end{enumerate}

\end{proof}

\subsubsection{Ejemplos}

\begin{itemize}
    \item Sea $f: \mathrm{VAR} \to \{ 0,1 \} / f(p_i) =
        \begin{cases}
            1 & i \text{ es par} \\
            0 & i \text{ es impar}
        \end{cases}$

    y sea $v_f$ la valuación que extiende a $f$.


    Por ejemplo,
    \begin{align*}
        v_f(p_1 \wedge p_3) &= \min{\{ v_f(p_1), v_f(p_3)\}} \\
                            &= \min{\{ \underbrace{f(p_1)}_{=0},
                            \underbrace{f(p_\beta)}_{=0} \}} = 0
    \end{align*}


    \item Sea $f:\mathrm{VAR} \to \{ 0,1 \} / f(p_i) = 1$, $\forall i \in \mathbb{N}$. 
 
    Sea $g: \mathrm{VAR} \to \{ 0,1 \} / g(p_i) =
        \begin{cases}
            1 & i \text{ es par} \\
            0 & i \text{ es impar}
        \end{cases}$
 
    Y sean $v_f$ y $v_g$ son las valuaciones que extienden a $f$ 
    y $g$ respectivamente.
 
    Sea $\alpha = (p_0 \to (\neg p_2 \wedge p_4))$, 
    $\mathrm{VAR}(\alpha) = \{ p_0,p_2,p_4 \}$

    \begin{gather*}
        \frest{v_f}{\mathrm{VAR}(\alpha)} = \frest{v_g}{\mathrm{VAR}(\alpha)} 
        \implies v_f(\alpha) = v_g(\alpha)
    \end{gather*}

     \item $\alpha = (\neg p_1 \to (p_2 \vee p_3))$

         Defino $f: \mathrm{VAR} \to \{ 0, 1 \}$ tal que
         $f(p_i) = \begin{cases}
             0 & i = 1, 2 \\
             1 & \text{en otro caso}
         \end{cases}$

         Entonces existe una única $v_f \in \mathrm{VAL}$ tal que $v_f$ 
         extiende a $f$.


         \begin{align*}
             v_f(\alpha) &= \max{\{ 1 - v_f(\neg p_1), v_f(p_2 \vee p_3) \}}\\
                         &= \max{\{ v_f(p_1),
                            \max{\{ v_f(p_2), v_f(p_3) \}} \}} \\
                         &= 1
         \end{align*}

         Defino $g: \mathrm{VAR} \to \{ 0, 1 \}$ tal que
         $g(p_i) = \begin{cases}
             1 & i = 3 \\
             0 & \text{en otro caso}
         \end{cases}$

        Entonces $v_g \in \mathrm{VAL}$ extiende a
        $v_g(\alpha) = v_f(\alpha) = 1$
        porque:
        \begin{itemize}
            \item $\frest{g}{\mathrm{VAR}(\alpha)} =
                \frest{f}{\mathrm{VAR}(\alpha)}$
        \item $\frest{v_g}{\mathrm{VAR}(\alpha)} =
            \frest{g}{\mathrm{VAR}(\alpha)}$
        \item $\frest{v_f}{\mathrm{VAR}(\alpha)} =
            \frest{f}{\mathrm{VAR}(\alpha)}$
        \end{itemize}
\end{itemize}


\subsection{Proposición sobre las tautologías}


\nota{\textit{Noni:} ``La demostración de esta proposición es súper importante.
Hay varios ejercicios parecidos en la práctica y es un tema que evaluamos
mucho en exámenes.''}%
\begin{proposicion}{}{p-no-varsalpha-alpha-tautologia}
    Sea $\alpha \in F$, $(p_1 \to \alpha)$ tautología.

    \medskip

   \begin{gather*}
       \text{Si } p_1 \notin var(\alpha) 
        \implies \alpha \text{ es tautología}
   \end{gather*} 
\end{proposicion}

\begin{proof}[\textit{Noni}: ``Esta es una MALA demostración que vi en 
    exámenes.'']

    \bigskip

    \begin{align*}
        & v(p_1 \to \alpha) = 1 \notamath{$\forall v$ valuación} \\
        \iff & \max \{ 1-v(p_1), v(\alpha) \} = 1 
        \notamath{$\forall v$ valuación} \\
        \iff & 1-v(p_1) = 1 \text{ ó } v(\alpha)=1 
        \notamath{$\forall v$ valuación} \\
        \iff & v(p_1) = 0 \text{ ó } v(\alpha) = 1
        \notamath{Ambos, $\forall v$ valuación}
    \end{align*}

    Como $v(p_1) = 0$ $\forall v$ valuación es falso, pues para algunas
    valuaciones $v(p_1)=1$, entonces el caso verdadero es el segundo:
    $v(\alpha)=1$, $\forall v$ valuación.

    \begin{gather*}
        \therefore ~ \alpha \text{ tautología.}
    \end{gather*}
\end{proof}

En la demostración anterior, lo primero que notamos es que falta el dato de
la proposición: $p_1 \notin var(\alpha)$.
Particularmente, en esta proposición el dato mencionado es imprescindible.

\medskip

Tomemos por ejemplo $\alpha=(p_1 \to p_1)$. 

Esta es una tautología pues 
$v(\alpha)=\max \{ 1 - v(p_1), v(p_1) \}$
y esto siempre vale $1$.

Sin embargo, la fórmula recuadrada es una contingencia: 
$(p_1 \to \boxed{p_1})$

Con esto vemos que si el dato de la Proposición 
\ref{propo:p-no-varsalpha-alpha-tautologia}, $p_1 \notin var(\alpha)$, no
se cumple entonces no es necesariamente verdad que $\alpha$ es tautología.

En conclusión, si no lo escribimos en nuestra demostración, esta es una pista
de que \textit{algo} está mal en la misma.

\medskip

El error fundamental de la demostración está en el último ``$\iff$'' pues
el ``\verb+para todo+'' \textbf{no} se distribuye con el ``\verb+ó+''.

Es decir, es falso que
\begin{gather*}
    1-v(p_1) = 1 \text{ ó } v(\alpha)=1 ~ \dashbox{$\forall v$ valuación} \\
    ~\boxed{\iff}~ \\
    v(p_1) = 0 ~ \dashbox{$\forall v$ valuación} 
    \text{ ó } v(\alpha) = 1 ~ \dashbox{$\forall v$ valuación} 
\end{gather*}


Realicemos la prueba correcta.

\begin{proof} \phantom{.}

    \nota{El ``cualquiera'' es redundante, pero Noni lo enfatizó para que 
    quede claro.\\
    Notemos que $f$ está bien definida pues $p_1 \notin var(\alpha)$.}%
    Sea $v$ una valuación cualquiera.

    Defino 
    \begin{gather*}
        f:\mathrm{VAR}\to \{ 0,1 \} / f(p_j) =
            \begin{cases}
                v(p_j) & \text{si } p_j \in var(\alpha) \\
                1 & \text{sino}
            \end{cases}
    \end{gather*}

    Y sea $v_f$ la valuación que extiende a $f$.

    Entonces

    \begin{align*}
        1 =& \, v_f(p_1 \to \alpha) \notamath{Por ser tautología.} \\
        =& \, \max \{ 1-v_f (p_1), v_f(\alpha) \} \\
        \implies& 1 - \underbrace{v_f(p_1)}_{f(p_1)} = 1 
        \text{ ó } v_f(\alpha) = 1 \\
        \implies& \underbrace{0 = 1}_{\text{Falso}} \text{ ó } v_f(\alpha)=1
    \end{align*}

    \begin{gather*}
        \dashbox{$\therefore ~ v_f(\alpha)=1$}
        \notamath{$\forall v$ valuación}
    \end{gather*}

    \medskip

    Por otra parte tenemos que:
    \begin{gather*}
        \frest{v}{var(\alpha)} = \frest{v_f}{var(\alpha)}
    \end{gather*}

    Entonces, por el Teorema \ref{teo:igualdad-valuaciones-frest}:

    \nota{\textit{Noni:} ``Si entienden realmente esta demostración, están 
    comenzando a entender el tema.''}%
    \begin{gather*}
        \boxed{ v(\alpha) = v_f(\alpha)=1}
    \end{gather*}

\end{proof}



\subsection{Equivalencia}

\begin{definicion}{}{}
    Sean $\alpha, \beta \in F$.

    \medskip

    Decimos que $\alpha$ es equivalente a $\beta$ si $v(\alpha)=v(\beta)$, 
    $\forall v$ valuación.

    \bigskip
    \textbf{Notación:}
    \( \alpha \equiv \beta \)
\end{definicion}


\bigskip
\textit{Observación:}

Definimos $\mathcal{R}$ en $F$ tal que $\alpha \mathcal{R} \beta$
si $\alpha \equiv \beta$.
Entonces $\mathcal{R}$ es de equivalencia.

\begin{proof} \phantom{.}
    Tarea.

    \textit{Demostrada ``en el aire'' por Noni:}
    Para ser una relación de equivalencia tiene que ser reflexiva, entonces
    tiene que ocurrir que $\alpha \equiv \alpha$. Esto es fácil de probar
    porque $v(\alpha) = v(\alpha)$ para toda $v$ valuación.

    En cuanto a la simetría, supongamos que $\alpha \mathcal{R} \beta$, 
    entonces $\alpha \equiv \beta$, entonces $v(\alpha)=v(\beta)$ para toda
    $v$ valuación. Pero, entonces, $v(\beta)=v(\alpha)$ para toda $v$
    valuación, con lo cual $\beta \mathcal{R} \alpha$.

    Por último, para la transitividad, $\alpha \mathcal{R} \beta$ y
    $\beta \mathcal{R} \gamma$, entonces $v(\alpha) = v(\beta)$ para toda
    $v$ valuación y $v(\beta)=v(\gamma)$ para toda $v$ valuación. Pero por
    transitividad de la igualdad, $v(\alpha) = v(\gamma)$ para toda $v$
    valuación. Por lo tanto $\alpha \mathcal{R} \gamma$.

    Como es reflexiva, simétrica y transtiva, entonces es una relación de
    equivalencia.
\end{proof}


Como es una relación de equivalencia, entonces tiene clases de equivalencia.
¿Cuáles son?

$\left[ ( p_1 \wedge \neg p_1 )  \right] \{ \alpha\in F / 
\alpha \text{ es contradicción} \}$ 


$\left[ ( p_1 \vee \neg p_1 )  \right] \{ \alpha\in F / 
\alpha \text{ es tautología} \}$ 

Por otra parte, vamos a tener infinitas clases de contingencias.

Tarea: ¿Cómo caracterizaríamos todas las clases de equivalencia que hay?


\subsubsection{Ejemplo}

\nota{Notemos que \textbf{no} es la misma fórmula. Es decir, 
$\alpha \neq \neg \neg \alpha$ porque \\
$c(\neg \neg \alpha) = 2 + c(\alpha)$}%
\begin{gather*}
    \alpha \equiv \neg\neg \alpha
\end{gather*}

\begin{proof} \phantom{.}

    \begin{align*}
        v(\neg \neg \alpha) =& 1 \\
        =& 1-(1-v(\alpha)) \\
        =& 1 - 1 + v(\alpha) \\
        =& v(\alpha)
    \end{align*}
\end{proof}

\subsection{Función booleana}

\begin{definicion}{Función booleana}{}    
    Sea la función $f: {\{ 0,1 \}}^{n} \to \{ 0,1 \}$ con 
    $n \in \mathbb{N}_{\geq 1}$.

    Entonces $f$ se denomina función booleana.
\end{definicion}

\subsubsection{Ejemplo}

\begin{gather*}
    f: {\{ 0,1 \}}^{3} \to \{ 0,1 \} / f(x,y,z) = \begin{cases}
        1 & \text{si } x = y\\
        0 & \text{sino} 
    \end{cases}
\end{gather*}

\begin{center}
    \begin{tabular}{c c c c} 
        $f(1,1,0)=1$ & $f(1,0,1)=0$ & $f(1,1,1)=1$ & etc...
    \end{tabular}
\end{center}

\begin{teorema}{}{}
    \begin{gather*}
        \notamath{$\frest{\mathrm{FORM}}{\equiv}$ es el conjunto de clases 
        de equivalencia inducidas por la relación de equivalencia sobre 
        las fórmulas.}
        \{ f / f \text{ es función booleana} \} 
    \xrightarrow[\text{biyección}]{} \frest{\mathrm{FORM}}{\equiv}
    \end{gather*}%
\end{teorema}

El teorema nos está diciendo que existe una función biyectiva entre ambos
conjuntos.

\begin{proof} \phantom{.}
    No la vemos.
\end{proof}

\subsubsection{Ejemplos}

\begin{enumerate}
    \item $\alpha = (p_1 \to p_2)$ ¿Con qué función booleana la vinculo?
        \nota{Notar que $n$ es la potencia del conjunto en el dominio de $f$}%

        Como $n = \# var(\alpha) = 2$, defino $f: {\{ 0,1 \}}^{2} \to 
        \{ 0,1 \}$ tal que:
            $f(0,0) = 1$,
            $f(0,1) = 1$,
            $f(1,0) = 0$,
            $f(1,1) = 1$

    \item $\alpha = ((p_1 \wedge \neg p_2) \to p_3)$ 
        ¿Con qué función booleana la vinculo?

        Defino $f: {\{ 0,1 \}}^3 \to \{ 0,1 \}$ tal que:
       \begin{gather*}
            f(x,y,z) = \begin{cases}
                0 & x = 1, \; y=0, \; z=1 \\
                1 & \text{ otro caso}
            \end{cases} 
        \end{gather*} 

    \item $f: {\{ 0,1 \}}^2 \to \{ 0,1 \} / f(x,y) = \begin{cases} 
            1 & \text{si } x=y \\
            0 & \text{sino}
            \end{cases}$
        ¿Con qué $\alpha \in F$ la vinculo?

        Obervemos que
        \begin{gather*}
            \begin{array}{c | c | c}
                x & y & f(x,y) \\
                \hline 
                0 & 0 & 1 \\
                0 & 1 & 0 \\
                1 & 0 & 0 \\
                1 & 1 & 1
            \end{array}
        \end{gather*}

        \nota{La fórmula es la disyunción de tantas fórmulas como renglones
        vayan a parar al $1$. En este caso hay dos.}%
        Entonces defino:
        $\alpha = (\neg p_1 \wedge \neg p_2) \vee (p_1 \wedge p_2)$.

    \item $f: {\{ 0,1 \}}^3 \to \{ 0,1 \} / f(x,y,z) = \begin{cases}
            1 & x = 1 \\ 0 & \text{sino} \end{cases}$
        ¿Con qué $\alpha \in F$ la vinculo?

        En este caso:
        \begin{gather*}
            \begin{array}{c | c | c | c}
                x & y & z & f \\
                \hline 
                0 & 0 & 0 & 0 \\
                0 & 0 & 1 & 0 \\
                0 & 1 & 0 & 0 \\
                0 & 1 & 1 & 0 \\
                1 & 0 & 0 & 1 \\
                1 & 0 & 1 & 1  \\
                1 & 1 & 0 & 1  \\
                1 & 1 & 1 & 1 
            \end{array}
        \end{gather*}

        Nos concentramos en los renglones que van a parar al uno y
        obtenemos la siguiente fórmula:

        \begin{gather*}
            \alpha = (p_1 \wedge \neg p_2 \wedge \neg p_3) \vee
            (p_1 \wedge \neg p_2 \wedge p_3) \vee 
            (p_1 \wedge p_2 \wedge \neg p_3) \vee (p_1 \wedge p_2 \wedge p_3)
        \end{gather*}

        \medskip
        Uno podría probar (\textit{tarea}) que $\alpha \equiv p_1$.

        Además es mejor trabajar con $p_1$, pues $c(\alpha) = 15$ mientras 
        que $c(p_1) = 0$ y es siempre más fácil trabajar con fórmulas de 
        complejidad mínima.
\end{enumerate}

En general, dada $\alpha$ tal que
$var(\alpha) \subseteq \{ p_0, \dotsc, p_{n-1} \}$, podemos
definir para cada $a \in {\{ 0, 1 \}}^n$ una función
$f_a : \mathrm{VAR} \to \{ 0, 1 \} /
f_a(p_k) = \begin{cases}
    a_k & \text{si } 0 \leq k \leq n-1 \\
    1 & \text{sino}
\end{cases}$

Siendo $v_{f_a}$ la valuación que extiende a $f_a$.

Definiendo luego $f_\alpha : {\{ 0, 1 \}}^n \to \{ 0, 1 \}$, entonces
$f_\alpha(a) = v_{f_a} (\alpha)$

\subsection{Propiedad}

\begin{itemize}
    \item Sean $var(\alpha), var(\beta) \subseteq \{ p_0, \dotsc, p_{n-1} \}$.

        Entonces
        $\alpha \equiv \beta \iff f_\alpha = f_\beta$
\end{itemize}

\begin{proof} \phantom{.}

    \begin{itemize}
        \item $\implies$)
            Sean
                $f_\alpha : {\{ 0, 1 \}}^n \to \{ 0, 1 \}$
                y
                $f_\beta : {\{ 0, 1 \}}^n \to \{ 0, 1 \}$

            Luego
            \begin{gather*}
                f_\alpha (a) = v_{f_a}(\alpha)
                \underbrace{=}_{\alpha \equiv \beta}
                v_{f_a}(\beta) = f_\beta (a)
                \notamath{$\forall a \in {\{ 0, 1 \}^n}$} \\
                \implies f_\alpha = f_\beta
            \end{gather*}
        \item $\impliedby$)
            Si $\alpha \not \equiv \beta$ $\implies$ existe
            $v \in \mathrm{VAL}$ tal que $v(\alpha) \neq v(\beta)$.

            Sin pérdida de generalidad, podemos afirmar que
            $v(\alpha) = 1$,
            $v(\beta) = 0$
            y
            $a = (v(p_0), v(p_1), \dotsc, v(p_k))$

            \nota{Por hipótesis}%
            Luego $v_{f_a}(\alpha) = f_\alpha (a) = f_\beta (a)$

            Entonces $f_a (p_k) = \begin{cases}
                a_k = v(p_k) & \text{si } 0 \leq k \leq n - 1 \\
                1 & \text{sino}
            \end{cases}$

            $\implies$ $v_{f_a}(p_k) = v(p_k)$ si $0 \leq k \leq n - 1$

            Por otra parte, como
            $var(\alpha), var(\beta) \subseteq \{ p_0, \dotsc, p_{n-1} \}$,
            entonces
            \begin{gather*}
            \frest{v_{f_a}}{var(\alpha)} = \frest{v}{var(\alpha)}
            ~ \text{y} ~
            \frest{v_{f_a}}{var(\beta)} = \frest{v}{var(\beta)}
            \end{gather*}

            Por lo tanto, por teorema,
            $v_{f_a}(\alpha) = v(\alpha) = 1$.
            Y esto a su vez es igual a $f_\alpha(a)$, que es igual a
            $v_{f_a}(\beta) = v(\beta) = 0$

            Lo cual es un absurdo.

            Por lo tanto, $\alpha \equiv \beta$.
    \end{itemize}
\end{proof}

\begin{teorema}{}{}
    Sea $f: {\{ 0, 1 \}}^n \to \{ 0, 1 \}$ función booleana.

    \medskip

    \nota{Recuerdo:\\
        $f_{\alpha} (a_0, \dotsc, a_{n-1}) 
        = v_{f_{(a_0, \dotsc, a_{n-1})}} (\alpha)$

        $f_{(a_0, \dotsc, a_{n-1})} (p_k) = \begin{cases}
            a_i & \text{si } k=i \\
            0 & \text{sino}
        \end{cases}$
    }%
    Existe $\alpha \in F$ tal que 
    $var(\alpha) \subseteq \{ p_0, \dotsc, p_{n-1} \}$
    y
    $f_{\alpha} : {\{ 0, 1 \}}^n \to \{ 0, 1 \}$
    y
    $f = f_{\alpha}$
\end{teorema}

\begin{proof} \phantom{.}

    Si $f(a) = 0$, $\forall ~ a \in {\{ 0, 1 \}}^n$

    Sea $\alpha = p_0 \wedge \neg p_0$.
    Luego $f_{\alpha}(a) = 0$ pues $\alpha$ es contradicción.

    Por lo tanto, $f_{\alpha} = f$.

    Por otra parte, si 
    $T = \{ a \in {\{ 0, 1 \}}^n / f(a) = 0 \} \neq \varnothing$

    Entonces
    $(a \in T \iff f(a) = 1)$

    Sea $a \in T$
    $\implies a = (a_0, \dotsc, a_{n-1})$

    Sea $x_i = \begin{cases}
        p_i & a_i = 1 \\
        \neq p_i & a_i = 0
    \end{cases}$

    Luego, $\beta_a = ( \dots ((x_0 \wedge x_1) \wedge \dots x_{n-1}) \dots)$

    \bigskip
    \textit{Observación:}
    Si $v$ es una valuación cualquiera, entonces
    \begin{align*}
        v(\beta_a) = 1 \iff& v(x_i) = 1 
        \notamath{$\forall \; 0 \leq i \leq n-1$} \\
        \iff& v(p_i) = a_i
        \notamath{$\forall \; 0 \leq i \leq n-1$} 
    \end{align*}

    Sea 
    $\alpha = \vee_{a \in T} \beta_a 
    =
    \beta_{0, 0, \dotsc, 1, 0)} \vee \beta_{0, 0, \dotsc, 1, 1)} \vee \dots$

    Sea $v \in \mathrm{Val}$ tal que $v(\alpha) = 1$.
    \begin{align*}
        v(\alpha) = 1 \iff& \text{existe } a \in T: v(\beta_a) = 1 \\
        \iff& \text{existe } a \in T: v(p_i) = a_i
        \notamath{$\forall \; 0 \leq i \leq n-1$}
    \end{align*}

    Como $var(\alpha) \subseteq \{ p_0, \dotsc, p_{n-1} \}$,
    $v(\alpha)$ ``sólo depende de $v(p_0), \dotsc, v(p_{n-1})$''

    \begin{align*}
        f_{\alpha} (a_0, \dotsc, a_{n-1})
        =& v_{f_{(a_0, \dotsc, a_{n-1})}}(\alpha) \\
        =& 1 \\
        v_{f_{(a_0, \dotsc, a_{n-1})}}(p_k) =& \begin{cases}
            a_i & \text{si } k = i, 0 \leq i \leq n-1 \\
            0
        \end{cases}\\
    \end{align*}

    Es decir:
    \begin{align*}
        f_{\alpha}(a) = 1 \iff& a \in T \iff f(a) = 1 \\
        \implies& f_{\alpha} = f
        \notamath{Pues si\\$f_{\alpha} (a) = 0 \Leftrightarrow f(a) = 0$}
    \end{align*}
\end{proof}

\bigskip
\textit{Observación:}
\begin{definicion}{Forma disyuntiva normal}{}
    A la fórmula de $\alpha$ del teorema anterior se le dice ser la 
    ``forma disyuntiva normal'' de la función $f$.

    \bigskip
    \textbf{Notación:}
    FDN
\end{definicion}

\begin{corolario}{}{}
    Sea $\alpha \in \mathrm{FORM}$

    \medskip

    Entonces existe una fórmula $\gamma$ en FDN tal que 
    $\alpha \equiv \gamma$.
\end{corolario}

\begin{proof} \phantom{.}

    Dado $var(\alpha) \subseteq \{ p_0, \dotsc, p_{n-1} \}$, sea
    $f_{\alpha} : {\{ 0, 1 \}}n \to \{ 0, 1 \}$ su función booleana asociada.

    $\implies$ Por el teorema anterior y la observación, existe
    $\gamma \in \mathrm{FORM}$ en FDN tal que $f_{\alpha} = f_{\gamma}$,
    $var(\gamma) \subseteq \{ p_0, \dotsc, p_{n-1} \}$

    \nota{Por propiedad}%
    $\implies$ $\alpha \equiv \gamma$
\end{proof}

\subsection{Conectivos adecuados}

\begin{definicion}{Conectivos adecuados}{}
    Sea $C$ un conjunto de conectivos. 

    $F_C = \{$fórmulas que tienen conectivos de $C \} \cup \mathrm{VAR}$

    \medskip

    $C$ es adecuado si 
    \begin{gather*}
        \forall \alpha \in \mathrm{FORM}, \exists \; \beta \in F_C / 
        \beta \equiv \alpha
    \end{gather*}

\end{definicion}

\subsubsection{Ejemplos}

\begin{enumerate}
    \item Sea $C = \{ \neg, \wedge \}$. ¿Puedo escribir cualquier fórmula
        de la lógica proposicional usando solamente \textit{no} e \textit{y}?
        Es decir, ¿siempre voy a poder encontrar un fórmula equivalente
        que tenga solamente estos dos conectivos?

        La respuesta es sí, el conjunto $\{ \neg, \wedge \}$ es adecuado.
        \begin{proof} \phantom{.}
        \begin{align*}
            (\alpha \vee \beta) \equiv& \\
            \equiv& \neg \neg (\alpha \vee \beta) \notamath{Teórica} \\
            \equiv& \neg(\neg \alpha \wedge \neg \beta) \notamath{De Morgan}\\
            (\alpha \to \beta) \equiv& \\
            \equiv& \neg \alpha \vee \beta \notamath{Tarea: demo. del 
            $\equiv$} \\
            \equiv& \neg(\neg \neg \alpha \wedge \neg \beta) \\
            \equiv& \neg(\alpha \wedge \neg \beta)
        \end{align*}
        \end{proof}

    \item $C = \{ \wedge, \vee, \neg, \to, \leftrightarrow \}$ es adecuado.
        \begin{proof} \phantom{.}
        
        Es trivial porque tiene más conectivos. Puedo escribir las fórmulas
        con $\{ \wedge, \vee, \neg, \to \}$ y, además, le estoy agregando
        un conectivo.
        \begin{gather*}
            \alpha \leftrightarrow \beta 
            \equiv (\alpha \to \beta) \wedge (\beta \to \alpha)
        \end{gather*}
        \end{proof}

    \item $C = \{ \wedge, \to \}$ no es adecuado. 
        \nota{\textit{Noni:} ``Esta técnica sirve para 
        muchos ejemplos de no adecuados (pero no para todos).''}%
        \begin{proof} \phantom{.}
        
            Defino $f: \mathrm{VAR} \to \{ 0,1 \} / f(p_j) = 1$. Sea $v_f$ la 
            valuación que extiende a $f$.

            Veamos que $v_f(\alpha) = 1$ $\forall \alpha \in F_C$.
            \nota{\textit{Noni:} ``Si digo ``\underline{La} novia de Juan'' 
                está clarísimo que tiene una única novia; ``\underline{Una} 
                novia de Juan'' implica que puede haber más.
                Lo mismo ocurre acá: al decir ``\underline{la}'' 
                es única.''}%

            Sea $C(\alpha) = $ cantidad de veces que aparece $\wedge, \to$ en
            $\alpha$. Por inducción en $C$.

            \begin{itemize}
                \item CB) $c(\alpha) = 0 \implies \alpha = p_j \in \mathrm{VAR}$
                   \begin{gather*}
                       \implies v_f(\alpha) = f(p_j) = 1
                   \end{gather*} 

                \item HI) $\alpha \in F_C / c(\alpha) \leq n 
                    \implies v_f(\alpha)=1$

                \item T) $\alpha \in F_C / c(\alpha) = n+1 
                    \implies v_f(\alpha) = 1$
            \end{itemize}

            Sea $\alpha \in F_C/ c(\alpha) = n + 1$

            \begin{enumerate}[%
                            labelindent=*,
                            style=multiline,
                            leftmargin=*,
                            align=left,
                            leftmargin=2\parindent,
                            label=Caso \arabic*)]
                \item $\alpha = (\beta_1 \wedge \beta_2)$
                    \nota{$\beta_1,\beta_2 \in F_C$}%
                    \begin{gather*}
                        c(\alpha) = \underbrace{c(\beta_1)}_{\geq 0} 
                        + \underbrace{c(\beta_2)}_{\geq 0} + 1 = n+1 \\
                        \implies c(\beta_i) \leq n \notamath{$i\in\{1,2\}$}\\
                        % \therefore ~ v_f(\beta_1) \notamath{Por HI} 
                        % - v_f(\beta_2) = 1 \\
                        \implies \dashbox{$v_f(\alpha) = 
                        \min \{ \underbrace{v_f(\beta_1)}_{=1},
                        \underbrace{v_f(\beta_2)}_{=1} \} = 1$}
                    \end{gather*}

                \item $\alpha = (\beta_1 \to \beta_2)$
                    \begin{gather*}
                        c(\alpha) = \underbrace{c(\beta_1)}_{\geq 0} 
                        + \underbrace{c(\beta_2)}_{\geq 0} + 1 = n+1 \\
                        c(\beta_i) \leq n \notamath{$i \in \{ 1,2 \}$}\\
                        \implies  v_f(\beta_i) = 1 \notamath{Por HI}\\
                        \implies \dashbox{$v_f(\alpha) 
                            = \max \{ 1-v_f(\beta_1), v_f(\beta_2) \}
                        = 1$}
                    \end{gather*}
            \end{enumerate}

            Sea $\alpha = (p_1 \wedge \neg p_1) \in F$. Supongo $C$ adecuado.
            \begin{gather*}
                \implies \exists \; \beta \in F_C/ \beta \equiv \alpha \\
                \implies \underbrace{v_f(\alpha)}_{=1} 
                = \underbrace{v_f(\beta)}_{=0} = 1
            \end{gather*}

            ¡Absurdo!
            Pues $\alpha$ es una contradicción.

            El absurdo vino de suponer que el conjunto de conectivos es
            adecuado.
            \begin{center}
                \fbox{$\therefore ~ C$ no es adecuado.}
            \end{center}

        \end{proof}

    \item Dado $C = \{ \downarrow \}$, que se interpreta como 
        $\alpha \downarrow \beta \equiv \neg \alpha \wedge \neg \beta$

        Decidir si es adecuado.

        Para conseguir $\neg$, propongo
        \begin{gather*}
            \alpha \downarrow \alpha \equiv
             \neg \alpha \wedge \neg \alpha \equiv
             \neg \alpha
        \end{gather*}

        Luego, para conseguir el $\wedge$:
        \begin{gather*}
            \alpha \wedge \beta \equiv
            \neg \alpha \downarrow \neg \beta \equiv
            (\alpha \downarrow \alpha) \downarrow (\beta \downarrow \beta)
        \end{gather*}

        Por lo tanto, como puedo conseguir $\{ \neg, \wedge \}$ a partir de
        $C$, y como $\{ \neg, \wedge \}$ es adecuado
        $\implies C$ es adecuado

    \item $C = \{ \neg \}$ no es adecuado.

        Sea $\alpha = (p_0 \vee \neg p_0)$

        Afirmamos que no hay fórmula $\gamma \in \mathrm{FORM}_C$ tal que
        $\gamma \equiv \alpha$.

        Supongamos que sí existe $\gamma \equiv \alpha$.

        Si $c(\gamma) = 0 \implies \gamma = p_k$.
        Sea $f: var \to \{ 0, 1 \} / f(p_k) = 0 \; \forall \; k$ y
        sea $v_f$ la valuación que la extiende.

        Tomando $v_f (\alpha) = 1$ y $v_f (\gamma) = f(p_k) = 0$, notamos que
        no son equivalentes.
    
        Si $c(\gamma) \geq 1$. 
        Tomando $\gamma = \underbrace{\neg \dots \neg}_{n} p_k$ tenemos
        dos posibilidades:
        \begin{itemize}
            \item $n$ es par.
                \begin{gather*}
                    v(\gamma) = v(p_k) 
                    \implies 
                    v_f(\gamma) = v_f(p_k) = 0 
                    ~ \wedge ~
                    v_f(\alpha) = 1 \\
                    \implies \gamma \not\equiv \alpha
                \end{gather*}
            \item $n$ es impar.

                \begin{gather*}
                    v(\gamma) = 1 - v(p_k) 
                \end{gather*}

                Sea $g: var \to \{ 0, 1 \} / g(p_k) = 1 \; \forall \; k$, y
                sea $v_g$ la valuación que extiende a $g$.

                \begin{gather*}
                    v_g(\gamma) = 1 - v_g(p_k) = 1 - 1 = 0 
                    \text{ y }
                    v_g(\gamma) = 1 \\
                    \implies \alpha \not\equiv \gamma
                \end{gather*}
        \end{itemize}
\end{enumerate}

\subsection{Propiedad}

Sea $C$ un conjunto de conectivos (no vacío) tal que podemos ``reconstruir''
un conjunto de conectivos que ya es adecuado.

Entonces $C$ es adecuado.

\subsection{Definiciones}

\begin{definicion}{}{}
    Sean $\alpha \in F$ y $v$ valuación.

    \medskip 

    Decimos que $v$ satisface $\alpha$ si $v(\alpha)=1$
\end{definicion}

\medskip

\begin{definicion}{}{}
    Sea $\alpha \in F$.
    
    \medskip

    Decimos que $\alpha$ es satisfacible si $\exists \; v$ valuación tal que
    $v(\alpha) = 1$.
\end{definicion}

Otra manera (equivalente) de escribir esta definición es:
\begin{gather*}
    \exists \; v \text{ valuación } (\alpha \in \Gamma \implies v(\alpha) = 1)
\end{gather*}

\medskip

\begin{definicion}{}{}
    Sea $\Gamma \subseteq F$.

    \medskip

    Decimos que $\Gamma$ es \textit{satisfacible} si $\exists \; v$ valuación tal que
    $v(\alpha) = 1$ $\forall \alpha \in \Gamma$.

    \bigskip
    \textbf{Notación:}
    $v(\Gamma) = 1$ 

    \medskip

    \nota{$v(\Gamma) = 0$ \underline{NO} existe}%
    Decimos que $\Gamma$ es \textit{insatisfacible} si $\nexists \; v$ 
    valuación tal que $v(\Gamma)=1$.

    Es decir, $\forall v$ val. $\exists \; \alpha \in \Gamma$ $/$ $v(\alpha)=0$
\end{definicion}

\subsubsection{Ejemplos}

Decidir si $\Gamma$ es satisfacible.

\begin{enumerate}
    \item $\Gamma = \{ p_1, (p_1 \wedge p_2) \}$

        Sí, es satisfacible.

        \begin{proof} \phantom{.}

        Defino $f:\mathrm{VAR}\to \{ 0,1 \} / f(p_j)=1$. 

        Sea $v_f$ la valuación que extiende a $f$.
        \begin{gather*}
            v_f(p_1) = f(p_1) = 1 \\
            v_f(p_1 \wedge p_2) = \min \{ v_f (p_1), v_f(p_2) \} =
            \min \{\underbrace{f(p_1)}_{=1}, \underbrace{f(p_2)}_{=1}\} = 1 \\
        \end{gather*}
        \begin{gather*}
            \boxed{\therefore ~ v_f \text{ satisface } \Gamma}
        \end{gather*}

        \end{proof}

    \item $\Gamma = \{ p_1, \neg p_2, (\neg p_1 \wedge \neg p_2) \}$

        No es satisfacible.

        \begin{proof} \phantom{.}

        Supongo $\Gamma$ satisfacible $\implies \exists \; v$ valuación tal
        que $v(\Gamma)=1$

        \begin{gather*}
            v(p_1) = 1 \wedge v(\neg p_2) = 1
            \implies v(p_2)=0
        \end{gather*}

        Luego,
        \begin{gather*}
            v(\neg p_1 \wedge \neg p_2) = 1 = \min \{ 1-v(p_1),1-v(p_2) \}
            \implies v(p_1)=0 \text{ y } v(p_2)=0
        \end{gather*}

        Lo cual es absurdo pues $1 \neq 0$. El mismo vino de suponer que 
        $\Gamma$ es satisfacible.

        \begin{gather*}
            \boxed{\therefore ~ \Gamma \text{ es insatisfacible.}}
        \end{gather*}

        \end{proof}

    \item $\Gamma = \varnothing$ es satisfacible


        \begin{proof} \phantom{.}
            
            Supongo $\Gamma$ insatisfacible.

            Dada $v$ valuación, 
            \begin{gather*}
                \underbrace{\exists \; \alpha \in \Gamma}_{\text{Falso}} / 
                v(\alpha)=0
            \end{gather*}

            Otra manera:
            \begin{gather*}
                \underbrace{(\underbrace{\alpha \in \Gamma}_{\text{Falso}} 
                \Rightarrow v(\alpha)=1)}_{\text{Verdadero}}
            \end{gather*}

            \begin{gather*}
                \boxed{\therefore ~ \Gamma = \varnothing \text{ es 
                satisfacible y, además, la satisfacen todas las 
                valuaciones.}}
            \end{gather*}
                
        \end{proof}

    \item $\Gamma = \mathrm{FORM}$ es insatisfacible.

        \begin{proof} \phantom{.}
            
            Supongo $\Gamma$ satisfacible $\implies \exists \; v$ valuación
            tal que $v(\Gamma)=1$

            Entonces, como $\{ p_1, \neg p_1 \} \subseteq \Gamma$:
            \begin{gather*}
                v(p_1)=1 \text{ y } v(\neg p_1)= 1
            \end{gather*}

            ¡Absurdo! Vino de suponer que $\Gamma$ es satisfacible.

            Por lo tanto, $\Gamma$ es insatisfacible.

        \end{proof}
\end{enumerate}

\subsection{Consecuencia}

\begin{definicion}{Consecuencias}{}
    \nota{\textit{Noni}: ``\textbf{LA} definición.''}%
    Sean $\Gamma \subseteq \mathrm{FORM}$, $\alpha \in \mathrm{FORM}$

    \medskip

    Decimos que $\alpha$ es consecuencia de $\Gamma$ si
    \begin{gather*}
        \left( ~ v(\Gamma)=1 \implies v(\alpha) = 1 ~ \right) 
        \notamath{$\forall v$ valuación.}
    \end{gather*}

    \bigskip
    \textbf{Notación:}
    $\underbrace{C(\Gamma)}_{\substack{
        \text{Consecuencias}\\\text{de } \Gamma}}
    = ~ \{\alpha \in F / \alpha \text{ es consecuencia de } \Gamma\}$
\end{definicion}

Decimos que $\alpha \notin C(\Gamma)$ si
$\exists \; v \text{ valuación}/ v(\Gamma) = 1 \text{ y } v(\alpha) = 0$

\subsubsection{Ejemplos}

\begin{enumerate}
    \item Decidir si $\alpha \in C(\Gamma)$.

\begin{enumerate}
    \item $\Gamma = \{ p_1, (p_1 \to p_2) \}$, $\alpha = p_2$

        \nota{\textit{Noni}: ``Esto es siempre igual. No innoven por favor.''}%
        Vamos que $\alpha \in C(\Gamma)$

        \begin{proof} \phantom{.}
        
            Sea $v \text{ valuación} / v(\Gamma)=1$

            Entonces se cumple que
            \begin{gather*}
                v(p_1) = 1 \text{ y } v(p_1 \to p_2) = 1
            \end{gather*}

            \begin{gather*}
                1 = v(p_1 \to p_2)
                = \max \{1-\underbrace{v(p_1)}_{=1}, v(p_2)\} = v(p_2) \\
                \implies v(p_2)=1
            \end{gather*}

            \begin{gather*}
                \therefore ~ v(\Gamma) = 1 \implies v(p_2) = 1 \\
                \boxed{\text{Conclusión: } \alpha \in C(\Gamma)}
            \end{gather*}
        \end{proof}

    \item $\Gamma = \{ p_1, (p_1 \to p_2) \}$, $\alpha = p_3$

        Veamos que $\alpha \notin C(\Gamma)$

        \begin{proof} \phantom{.}
        
            Defino $f:\mathrm{VAR} \to \{ 0,1 \}/f(p_j) = \begin{cases}
                0 & j =3 \\
                1 & \text{sino}
            \end{cases}$

            Sea $v_f$ la valuación que extiende a $f$

            \begin{gather*}
                \left.\begin{aligned}
                    v_f(p_1) = f(p_1)=1 \\
                    \\
                    v_f (p_1 \to p_2) = \max \{ 1-\underbrace{v_f(p_1)}_{=1},
                    \underbrace{v_f(p_2)}_{=1} \}=1
                \end{aligned} \right\}
                \quad \dashbox{$v_f(\Gamma) = 1$}\\
            \end{gather*}

            Por otra parte,
            \begin{gather*}
                \dashbox{$v_f(p_3) = f(p_3) = 0$}
            \end{gather*}

            Entonces 
            \begin{gather*}
                \boxed{\alpha \notin C(\Gamma)}
            \end{gather*}

        \end{proof}


    \item $\Gamma = \{ p_1, p_2 \}$, $\alpha = \neg p_1$

        Veamos que $\alpha \notin C(\Gamma)$

        \begin{proof} \phantom{.}
        
            Defino $f: \mathrm{VAR}\to \{ 0,1 \}/ f(p_j) = 1$.

            Sea $v_f$ la valuación que extiende a $f$.

            \begin{gather*}
                v_f(p_1)=1 \wedge v_f(p_2) = 1 \implies v_f(\Gamma)=1
            \end{gather*}

            Pero $v_f(\alpha) = 1 - v_f(p_1) = 0$

            \begin{gather*}
                \boxed{\therefore ~ \alpha \notin C(\Gamma)}
            \end{gather*}
        \end{proof}
        
\end{enumerate}

\item Hallar $C(\Gamma)$.

\begin{enumerate}
    \item $C(\varnothing) = ?$

        Veamos que $C(\varnothing) = $ tautologías 
        $= \{ \alpha\in F /\alpha \text{ es tautología} \}$

        \begin{proof} \phantom{.}
        
            \begin{itemize}
                \item[$\subseteq$)] $\alpha \in C(\varnothing)$.

                    Quiero ver que $\alpha$ es tautología.

                    \medskip

                    Supongo que no 
                    $\implies \exists \; v \text{ valuación}/v(\alpha)=0$

                    \nota{$\varnothing$ es satisfacible.}%
                    Pero $v(\varnothing)=1 \implies \alpha \notin
                    C(\varnothing)$

                    ¡Absurdo!
                    Vino de suponer que $\alpha$ no era una tautología.

                \item[$\supseteq$)] $\alpha$ tautología.

                    Quiero ver que $\alpha \in C(\varnothing)$

                    \medskip

                    Sea $v \text{ valuación} / v(\varnothing)=1$
                    $\implies \underbrace{v(\alpha)=1}_{\alpha \text{ taut.}}$
            \end{itemize}
        \end{proof}

        \bigskip

        Otra forma:

        \begin{proof} \phantom{.}
        
            Como toda valuación satisface el $\varnothing$, si 
            \begin{gather*}
                \alpha \in C(\varnothing) \implies v(\alpha) = 1
                \notamath{$\forall v$ valuación} \\
            \end{gather*}
            \begin{gather*}
                \therefore ~ \alpha \in C(\varnothing) \iff \alpha 
                \text{ es tautología}
            \end{gather*}
        \end{proof}

    \item Supongamos que tengo un conjunto $\Gamma$ insatisfacible.
        $C(\Gamma) = ?$

        Veamos que $C(\Gamma) = \mathrm{FORM}$.

        \begin{proof} \phantom{.}
        
        \begin{itemize}
            \item $\subseteq$) Es trivial por definición.
            \item $\supseteq$) Supongo
                $\exists \; \alpha \in \mathrm{FORM} / \alpha \notin C(\Gamma)$
                \begin{gather*}
                    \exists \; v \text{ valuación}/v(\alpha)=0 
                    \text{ y } \overbrace{v(\Gamma) = 1}^{\text{¡Absurdo!}}
                \end{gather*}

                \medskip

                $V(\Gamma)=1$ es absurdo pues $\Gamma$ es insatisfacible.

                El mismo vino de suponer que había una fórmula que no está
                en las consecuencias de $\Gamma$.
        \end{itemize}

        De esta manera probamos que $C(\Gamma)$ son todas las fórmulas.

        \end{proof}

        \bigskip

        Otra manera:
        \begin{proof} \phantom{.}
        
            Sea $v$ valuación.

            \begin{gather*}
                \underbrace{(\underbrace{v(\Gamma) = 1}_{\text{Falso}}
                \implies v(\alpha) = 1)}_{\text{Verdadero}}
            \end{gather*}
        \end{proof}

    \item Tautologías $\subseteq C(\Gamma)$, $\Gamma \subseteq \mathrm{FORM}$

    \begin{proof} \phantom{.}
    
        Sea $\alpha \in$ Tautologías, quiero ver que $\alpha \in C(\Gamma)$.

        Sea $v$ val.$/v(\Gamma) = 1$

        \begin{center}
            $v(\alpha)=1$ pues $\alpha$ es tautología.
        \end{center}
    \end{proof}
\end{enumerate}

\end{enumerate}

\subsection{Teorema}

\nota{\textit{Noni}: ``Entra en la categoría de \textit{``fácil''} ''}%
\begin{teorema}{}{}
    Sea $\Gamma \subseteq \mathrm{FORM}$, $\alpha \in \mathrm{FORM}$

    \medskip

    \begin{gather*}
        \alpha \in C(\Gamma) \iff 
        \Gamma \cup \{ \neg \alpha \} \text{ es insatisfacible.}
    \end{gather*}
\end{teorema}

\begin{proof} \phantom{.}

    \begin{itemize}
        \item $\implies$) Por dato, $\alpha \in C(\Gamma)$. Quiero ver que
            $\Gamma \cup \{ \neg \alpha \}$ es insatisfacible.

            Supongo que $\Gamma \cup \{ \neg \alpha \}$ es satisfacible.
            \begin{align*}
                \implies& \exists  \; v \text{ valuación}/
                v\left(\Gamma \cup \{ \neg\alpha \}\right) = 1 \\
                \implies& v(\Gamma) = 1 ~ \wedge ~ v(\neg \alpha) = 1 \\
                \implies& v(\Gamma) = 1 ~ \wedge ~ v(\alpha) = 0
            \end{align*}

            Entonces $\alpha \notin C(\Gamma)$, lo cual es absurdo.

        \item $\impliedby$) Por dato, $\Gamma \cup \{ \neg\alpha \}$ es
            insatisfacible. Quiero ver que $\alpha \in C(\Gamma)$

            Supongo que $\alpha \notin C(\Gamma)$
            \begin{align*}
                \implies& \exists \; v \text{ valuación}/ v(\Gamma)=1 
                ~ \wedge ~ v(\alpha)= 0 \\
                \implies& v(\Gamma) = 1 ~ \wedge ~ v(\neg \alpha)=1 \\
                \implies& v(\Gamma \cup \{ \neg \alpha \}) = 1 \\
                \implies& v \text{ satisface } \Gamma \cup \{ \neg\alpha \}
            \end{align*}

            Lo cual es un absurdo porque ese conjunto es insatisfacible.
    \end{itemize}
\end{proof}

\subsection{Teorema}

\nota{\textit{Noni}: ``Entra en la categoría de \textit{``fácil''} ''}%
\begin{teorema}{}{}
    Sean $\Gamma = \{ \gamma_1, \dotsc, \gamma_n \} \subseteq \mathrm{FORM}$, 
    $\alpha \in \mathrm{FORM}$

    \medskip

    \begin{gather*}
        \alpha \in C(\Gamma) \iff 
        \left( (\gamma_1 \wedge \dots \wedge \gamma_n) \to \alpha  \right) 
        \text{ es tautología}
    \end{gather*}
\end{teorema}

\begin{proof} \phantom{.}

    \begin{itemize}
        \item $\implies$) Por dato: $\alpha \in C(\Gamma)$, con 
            $\Gamma = \{ \gamma_1, \dotsc, \gamma_n \}$

            Quiero ver que 
            $\beta = (\gamma_1 \wedge \dots \wedge \gamma_n) \to \alpha$
            es tautología.

            \medskip

            Supongo que $\beta$ no es tautología.
            \begin{align*}
                \implies& \exists \; v \text{ valuación}/ v(\beta) = 0 \\
                \implies& \underbrace{
                    v(\gamma_1 \wedge \dots \wedge\gamma_n) = 1}_{
                    v(\gamma_i)=1 \implies v(\Gamma)=1
                \notamath{$1 \leq i \leq n$}}
                ~\wedge~ v(\alpha)=0
            \end{align*}

            Entonces $\alpha \notin C(\Gamma)$ pero $v(\Gamma)= 1$

            ¡Absurdo! Vino de suponer que $\beta$ no es tautología.

            Por lo tanto, si $\alpha \in C(\Gamma)$, entonces 
            $(\gamma_1 \wedge \dots \wedge \gamma_n) \to \alpha$
            es tautología.

        \item $\impliedby$) El dato que tenemos es que 
            $(\gamma_1 \wedge \dots \wedge \gamma_n) \to \alpha$ es
            tautología.

             Quiero ver que $\alpha \in C(\Gamma)$.

             \medskip

             Sea $v \text{ valuación}/v(\Gamma)=1$, quiero ver que 
             $v(\alpha)=1$
             \begin{align*}
                 \notamath{Como $v(\Gamma)=1$, entonces $v$ de cada fórmula 
                 de $\Gamma$ vale uno, es decir, $v(\gamma_1) = 1, 
                 v(\gamma_2)=1$, etc.}
                 1 &= v((\gamma_1 \wedge \dots \wedge \gamma_n) \to \alpha)\\
                   &= \max \{ 1-
                    \underbrace{v(\gamma_1 \wedge \dots \wedge \gamma_n)}_{=1},
                   v(\alpha) \} \\
                   &= v(\alpha)
             \end{align*}

             \begin{gather*}
                 \therefore ~ v(\alpha) = 1
             \end{gather*}

             Por lo tanto, $\alpha \in C(\Gamma)$.
    \end{itemize}
\end{proof}


\subsection{Teorema de la deducción}

\begin{teorema}{Teorema de la deducción %
(versión semántica)}{deduccion-semantica}
    Sean $\Gamma \subseteq \mathrm{FORM}$, $\alpha, \beta \in \mathrm{FORM}$.

    \medskip

    \begin{gather*}
        (\alpha \to \beta) \in C(\Gamma) \iff \beta \in 
        C(\Gamma \cup \{ \alpha \})
    \end{gather*}
\end{teorema}

\nota{\textit{Noni}: ``Es similar a las anteriores, inténten demostrarlo
 ustedes. Estos teoremas los tomamos mucho en los finales.''}%

\begin{proof} \phantom{.}

    \begin{itemize}
        \item $\implies$) Sea $v \text{ valuación}/ 
            v(\Gamma \cup \{ \alpha \}) = 1$.

            Quiero ver que $v(\beta)=1$.

            \medskip
            \begin{align*}
                v(\Gamma) = 1 ~ &\wedge ~ v(\alpha) = 1 \\
                1 &= v(\alpha\to\beta) = \max \{ 1 - 
                    \overbrace{v(\alpha)}^{=1}, v(\beta) \} 
                \notamath{Dato: $(\alpha\to\beta)\in C(\Gamma)$} \\
                  &= v(\beta)
            \end{align*}

            \begin{gather*}
                \therefore ~ v(\beta) = 1
            \end{gather*}

        \item $\impliedby$) Sea $v \text{ valuación}/ v(\Gamma)=1$

            \begin{enumerate}[%
                            labelindent=*,
                            style=multiline,
                            leftmargin=*,
                            align=left,
                            leftmargin=2\parindent,
                            label=Caso \arabic*)]
                \item $v(\alpha)=1 \implies v(\Gamma \cup \{ \alpha \})=1$

                    Como $\beta \in C(\Gamma \cup \{ \alpha \})$, $v(\beta)=1$

                    \begin{gather*}
                        v(\alpha\to\beta)
                        = \max \{ 1-v(\alpha), \underbrace{v(\beta)}_{=1}\}
                        = 1
                    \end{gather*}

                \item $v(\alpha)=0$

                    \begin{gather*}
                        v(\alpha\to\beta) 
                        =\max \{ 1-\underbrace{v(\alpha)}_{=0}, v(\beta)\}
                        = 1
                    \end{gather*}
            \end{enumerate}
    \end{itemize}
\end{proof}


\subsubsection{Ejemplo}

Probar que $\beta \in C( \{ \alpha, (\alpha\to\beta) \})$

Por el Teorema \ref{teo:deduccion-semantica}:

\begin{gather*}
    \beta \in C( \{ \alpha, (\alpha\to\beta) \}) \iff
    (\alpha\to\beta) \in C(\{ \alpha\to\beta \})
\end{gather*}

Lo cual es cierto porque es cierto que $\Gamma \subseteq C(\Gamma)$


\subsection{Independencia de fórmulas}

\begin{definicion}{Conjunto de fórmulas independientes}{}
    Sea $\Gamma$ un conjunto de fórmulas.

    \medskip

    Decimos que $\Gamma$ es un conjunto de fórmulas independientes si para
    todo $\alpha \in \Gamma$ se tiene que $\alpha \notin C(\Gamma-\{\alpha\})$.

    En caso contrario, $\Gamma$ es dependiente.
\end{definicion}

\subsubsection{Ejemplo}

Decidir si los siguientes conjuntos son independientes.

\begin{enumerate}
    \item $\Gamma = \{ p_1, (p_1 \to p_2)\}$

        Veamos que son independientes.

        \begin{proof} \phantom{.}
        \begin{enumerate}[%
                        labelindent=*,
                        style=multiline,
                        leftmargin=*,
                        align=left,
                        leftmargin=2\parindent,
                        label=Caso \arabic*)]
            \item $\alpha = p_1$ %$p_1 \notin C(\Gamma - \{ p_1 \})
                %= C\left(\{(p_1\to p_2)\}\right)$
                
                    \nota{$i=\{ 1,2 \}$}%
                    Sea $f:\mathrm{VAR}\to \{ 0,1 \} / f(p_i)=0$ 
                    para todo $p_i$.

                    Sea $v_f$ la única valuación que extiende a $f$.

                    \begin{align*}
                        \implies& v_f(p_1) = f(p_1) = 0 \\
                        & v_f(p_1\to p_2) = \max \{ 
                        \underbrace{1-\underbrace{v(p_1)}_{=0}}_{=1},v(p_2)\}
                        = 1
                    \end{align*}

                    Entonces,
                    \begin{gather*}
                        \alpha \notin C(\Gamma \backslash \{ \alpha \})
                        \notamath{$\Gamma \backslash \{ \alpha \}$ equivale
                        a $\Gamma \mathrm{-} \{ \alpha \}$}
                    \end{gather*}

            \item $\alpha = (p_1 \to p_2)$ 
                %$(p_1 \to p_2) \notin C(\Gamma - \{ (p_1\to p_2) \}
                %= C(\{ p_1 \})$

                    Sea $g: \mathrm{VAR}\to \{ 0,1 \} / f(p_i) = \begin{cases}
                        1 & \text{si } i=2 \\
                        0 & \text{si } i\neq 2
                    \end{cases}$

                    Sea $v_g$ la única valuación que extiende a $g$.

                    \begin{align*}
                        \implies& v_g(p_1)=g(p_1)=1 \\
                        & v_g(p_1 \to p_2)= \max 
                        \{\underbrace{1-\underbrace{v_g(p_1)}_{=1}}_{=0}, 
                        \underbrace{v_g(p_2)}_{=0}\} = 0
                    \end{align*}

                    Es decir,
                    $v_g(\Gamma \backslash \{ \alpha \}) = 1$ y
                    $v_g(\alpha) = 0$

                    \begin{gather*}
                        \implies \alpha \notin 
                        C(\Gamma \backslash \{ \alpha \})
                    \end{gather*}
        \end{enumerate}

        \begin{gather*}
            \therefore ~ \Gamma \text{ es independiente.}
        \end{gather*}

        \end{proof}

    \item $\Gamma = \{ p_1, p_2, (p_1 \to p_2)\}$

        Veamos que $\Gamma$ no es independiente.

        \begin{proof} \phantom{.}
                
            Notemos que 
            $(p_1 \to p_2) \in C(\Gamma \backslash \{ (p_1\to p_2 \}) 
            = C(\{ p_1, p_2 \})$.

            \medskip

            Sea $v \text{ valuación}/v(\{ p_1,p_2 \})=1$

            En particular,
            \begin{align*}
                v(p_1) = v(p_2) = 1 \\
                & v(p_1\to p_2) = \max \{1-v(p_1), 
                \underbrace{v(p_2)}_{=1}\} = 1
            \end{align*}

            Lo cual está contradiciendo a la definición de 
            independencia.

            \begin{gather*}
                \therefore ~ \Gamma \text{ es dependiente}
            \end{gather*}
        \end{proof}

    \item $\Gamma=\{ (p_n \wedge p_{n+1} / n \in \mathbb{N} \}$

        Notemos que $\Gamma = \{(p_0 \wedge p_1), (p_1\wedge p_2),
        (p_2 \wedge p_3), \dotsc\}$

        \nota{$(p_1 \wedge p_2)$ $\in$ \\
        $C\left(\Gamma \backslash \{(p_1 \wedge p_2)\}\right)$}%
        Como en el primer elemento del conjunto vemos que vale $p_1$ y en el
        tercero que vale $p_2$, podemos deducir que vale $(p_1 \wedge p_2)$.
        Es decir, la fórmula se puede deducir a partir de las demás.

        Por lo tanto, el conjunto es dependiente.
        
        \begin{proof} \phantom{.}
        
            Sea $v \text{ valuación}/
            v(\Gamma \backslash \{(p_1\wedge p_2)\})=1$

            \begin{align*}
                \implies& v(p_0 \wedge p_1) = 1 \\
                & v(p_2 \wedge p_3)=1 \\
                \implies& v(p_0\wedge p_1) = \min \{ \underbrace{v(p_0)}_{=1},
                \underbrace{v(p_1)}_{=1}\} = 1 \notamath{Como estamos 
            buscando el mínimo, ambos tienen que valer 1.}
            \end{align*}

            Entonces, como $v(p_0 \wedge p_1) = 1 \implies v(p_1)=1$ y como
            $v(p_2 \wedge p_3)=1 \implies v(p_2)=1$
            

            Por lo tanto, $v(p_1 \wedge p_2) = \min\{\underbrace{v(p_1)}_{=1},
            \underbrace{v(p_2)}_{=1}\} = 1$
            
            Con lo cual llegamos a probar que dada una valuación que 
            satisface el conjunto de fórmulas 
            $\{ \Gamma - \{(p_1\wedge p_2)\} \}$, entonces satisface 
            $(p_1\wedge p_2)$. Y eso contradice la definición de 
            independencia.

            \begin{gather*}
                \therefore ~ \Gamma \text{ es dependiente.}
            \end{gather*}

        \end{proof}
\end{enumerate}


\subsection{Base}

\begin{definicion}{Base}{}
    Sea $\Gamma$ un conjunto de fórmulas.

    \medskip

    Decimos que $\Gamma$ es una base si:
    \begin{enumerate}
        \item Es independiente.
        \item Es maximal con respecto a la independencia.
    \end{enumerate}
\end{definicion}


Recordemos que decimos que un conjunto va a ser ``maximal'' respecto al orden
dado si no hay ningún otro objeto estrictamente más grande. Esto es distinto
a ``máximo''.

El orden que nos interesa en el caso de una base, donde $\Gamma$ es un
conjunto de fórmulas, es la contención de conjuntos. Entonces, cuando decimos
un conjunto maximal estamos diciendo que no hay ningún otro conjunto que lo
contenga que sea independiente.


Formalmente, si $\Sigma$ es un conjunto independiente y 
$\Gamma \subseteq \Sigma$ $\implies \Gamma = \Sigma$

\nota{$\Gamma \varsubsetneq \Sigma$ se lee \textit{``$\Gamma$ es un 
subconjunto propio de $\Sigma$''}. Esto significa que $\Gamma$ tiene menos 
elementos que $\Sigma$.}%
Además, si tomamos un conjunto $\Sigma$ que agrega un elemento a una base 
$\Gamma$, es decir, $\Gamma \varsubsetneq \Sigma = \Gamma \cup \{\alpha\}$, 
con $\alpha \notin \Gamma$, entonces $\Sigma$ resulta ser dependiente.

\subsubsection{Ejemplos}

Decidir si los siguiente conjuntos son bases.

\begin{itemize}
    \item $\Gamma = \{ p_1, \neg p_1 \}$

        \begin{enumerate}
            \item Comencemos con la independencia.

                El conjunto es independiente pues sabiendo que vale $p_1$ no
                puedo deducir que vale $\neg p_1$. A lo sumo puedo deducir 
                que no vale $\neg \neg p_1$.

                Tarea: demostrarlo formalmente.

            \item Veamos si es maximal.

                El conjunto $\Gamma$ es insatisfacible pues las dos fórmulas
                que lo definen son contradictorias entre sí. Recordando
                que cuando un conjunto es insatisfacible la consecuencia del 
                mismo son todas las fórmulas.


                \begin{proof} \phantom{.}
                
                    Supongamos que $\Gamma$ no es maximal respecto a la 
                    independencia.

                    En consecuencia, $\exists \; \alpha \notin \Gamma /
                    \Sigma=\Gamma \cup \{ \alpha \}$ es independiente.

                    Por otra parte, como $\Gamma$ es insatisfacible, entonces 
                    $C(\Gamma)=\mathrm{FORM}$.

                    Por lo tanto, $\alpha \in C(\Gamma) = \mathrm{FORM} \implies
                    \alpha \in C(\Sigma - \{ \alpha \}) = \mathrm{FORM}$. Lo cual es
                    absurdo pues $\Sigma$ es independiente.

                    Como el absurdo viene de suponer que $\Gamma$ no es
                    maximal, probamos que sí lo es respecto de la 
                    independencia.

                \end{proof}

        \end{enumerate}

        Como se cumplen ambas condiciones de la definición, $\Gamma$ es base.

    \item $\Gamma = \mathrm{VAR}$

        Supongamos que $\alpha = p_i$ \nota{$i \in \mathbb{N}$}%

        \begin{enumerate}
            \item Empecemos con la independencia.

            Defino $f_i : \mathrm{VAR} \to \{ 0,1 \} / f_i(p_j) =
            \begin{cases}
                0 & \text{si } j = i \\
                1 & \text{si } j \neq i 
            \end{cases}$

            Sea $v_{f_i}$ la valuación que extiende a $f_i$.
            \begin{gather*}
                v_{f_i} (\alpha) = f_i(p_i) = 0 \\
                v_{f_i}(p_j) = f_i(p_j) = 1 \notamath{$j \neq i$}
            \end{gather*}
            Luego, 
            \begin{gather*}
                v_{f_i}(\Gamma \backslash \{ p_i \}) = 1 
                \text{ y } v_{f_i}(p_i)=0 \\
                \implies p_i \notin C(\Gamma \backslash \{ p_i \})
            \end{gather*}
            \begin{gather*}
                \dashbox{$\therefore ~ \Gamma$ es independiente.}
            \end{gather*}
        
            \item Notemos que $\exists ! v \text{ valuación}/v(\Gamma) = 1$
        
                \begin{enumerate}[%
                                labelindent=*,
                                style=multiline,
                                leftmargin=*,
                                align=left,
                                leftmargin=2\parindent,
                                label=Caso \arabic*)]
                    \item Si $v(\alpha) = 1$

                        Como $\exists !$ valuación que satisface a $\Gamma$ y
                        es $v \implies \alpha \in C(\Gamma)$
                        $\implies \alpha \in 
                        \underbrace{C(\Sigma \backslash \{\alpha\})}_{\Gamma}$

                        \begin{gather*}
                            \implies \Sigma \text{ es dependiente.}
                        \end{gather*}

                    \item Si $v(\alpha) = 0 \implies \Sigma$ es 
                        insatisfacible.

                        Sea $\beta = p_i$, con $p_i \notin \mathrm{VAR}(\alpha)$.
                        Veamos que $\Sigma \backslash \{ \beta \}$ es
                        insatisfacible.

                        Supongamos que $\Sigma \backslash \{ \beta \}$ es
                        satisfacible.
                        \begin{gather*}
                            \implies \exists \; w \text{ valuación} /
                            w(\Sigma \backslash \{\beta\} = 1
                        \end{gather*}

                        Entonces, $w(\alpha) = 1$ pues
                        $\alpha \in \Sigma \backslash \{ \beta \}$

                        Además, $\frest{w}{\mathrm{VAR}(\alpha)} = 1$ y
                        $\frest{v}{\mathrm{VAR}(\alpha)} = 1$

                        Por lo tanto $\underbrace{w(\alpha)}_{=1} 
                            = \underbrace{v(\alpha)}_{= 0}$.
                        Lo cual es un absurdo.

                        \bigskip
                        Otra forma, usando el \nameref{teo:compacidad}

                        Recordemos: Si $v(\alpha) = 0 \implies \Sigma$ es 
                        insatisfacible.

                        $\implies \exists \; \Sigma' \subseteq \Sigma$, 
                        $\Sigma'$ es finito e insatisfacible.

                        Tomamos $\beta \in \underbrace{\Sigma}_{\text{inf.}} 
                        \backslash \underbrace{\Sigma'}_{\text{finito}}$

                        Como $\Sigma' \subseteq \Sigma \backslash \{\beta\}$

                        Entonces
                        \begin{align*}
                            & \underbrace{c(\Sigma')}_{\mathrm{FORM}} 
                            \subseteq c(\Sigma \backslash \{\beta\} \\
                            \implies & c(\Sigma \backslash \{ \beta \}) = F \\
                            \implies & \beta \in 
                            c(\Sigma\backslash \{ \beta \})
                        \end{align*}
                \end{enumerate}

                \begin{gather*}
                    \therefore ~ \Gamma \text{ es independiente.}
                \end{gather*}
        \end{enumerate}
\end{itemize}

\subsection{Conjunto finitamente satisfacible}

\begin{definicion}{Conjunto finitamente satisfacible}{}
    $\Gamma$ es finitamente satisfacible (\Verb+f.s.+) si todo subconjunto
    finito de $\Gamma$ es satisfacible.
\end{definicion}

\medskip

\begin{lema}{}{gamma-pi-no-pi-esfs}
    Sea $\Gamma$ f.s., $p_i \in \mathrm{VAR}$

    \medskip


    \nota{\textit{Noni:} ``Es un ``o'' común. Pueden ocurrir ambas cosas a la 
    vez o solamente una.''}%
    Entonces $\Gamma \cup \{ p_i \}$ es f.s. o $\Gamma \cup \{ \neg p_i \}$
    es f.s.
\end{lema}

\subsubsection{Ejemplos}

Si tengo $\Gamma = \{ p_1 \}$, resulta ser f.s. pues cualquier subconjunto
finito es satisfacible y los subconjuntos de $\Gamma$ son el mismo y el
vacío.

Si tomamos $\Gamma \cup \{ p_2 \}$, también es f.s. (por el mismo motivo) y 
$\Gamma \cup \{ \neg p_2 \}$ también es f.s..


\begin{proof} \phantom{.} \nota{Del Lema \ref{lema:gamma-pi-no-pi-esfs}}%

    Supongo que existe $p_i \in \mathrm{VAR} / 
    \underbrace{\Gamma \cup \{ p_i \} \text{ no es f.s.}}_{\circled{1}} 
    \text{ y } 
    \underbrace{\Gamma \cup \{ \neg p_i \} \text{ no es f.s.}}_{\circled{2}}$

\begin{enumerate}[label=\protect\circled{\arabic*}]
    \item $\implies \exists \; X \subseteq \Gamma \cup \{ p_i \} / X$ 
        es finito e insatisfacible.

        \nota{$\Gamma$ es f.s.}%
        Notemos que $X \nsubseteq \Gamma$ 

        \begin{gather*}
         \therefore ~ X=X' \cup \{ p_i \} \notamath{$X' \subseteq \Gamma$}
        \end{gather*}

    \item $\implies \exists \; Y \subseteq \Gamma \cup \{ \neg p_i \} / Y$ 
        es finito e insatisfacible.

        \nota{$\Gamma$ es f.s.}%
        Notemos que $X \nsubseteq \Gamma$ 

        \begin{gather*}
         \therefore ~ Y=Y' \cup \{ \neg p_i \} \notamath{$Y' \subseteq \Gamma$}
        \end{gather*}

\end{enumerate}

\nota{$X'\subseteq \Gamma$, $Y' \subseteq \Gamma$, 
$\implies X'\cup Y' \subseteq \Gamma$}%
Entonces, $X' \cup Y' \subseteq \Gamma$; $X' \cup Y'$ es finito, pues $X'$ e
$Y'$ lo son, y como $\Gamma$ es f.s. $\implies \exists \; v \text{ valuación}/
v(X' \cup Y') = 1$ 

\begin{enumerate}[%
                labelindent=*,
                style=multiline,
                leftmargin=*,
                align=left,
                leftmargin=2\parindent,
                label=Caso \arabic*)]
    \item $v(p_i) = 1 \implies v(X' \cup \{ p_i \})=1 \implies v(X)=1$

        ¡Absurdo! $X$ es insatisfacible.

    \item $v(p_i) = 0 \implies v(Y' \cup \{ \neg p_i \})=1 \implies v(Y)=1$

        ¡Absurdo! $Y$ es insatisfacible.
\end{enumerate}

El absurdo vino de suponer que existe $p_i \in \mathrm{VAR} / 
\Gamma \cup \{ p_i \} \text{ no es f.s.} 
\text{ y } 
\Gamma \cup \{ \neg p_i \} \text{ no es f.s.}$

\end{proof}


\subsection{Literal}

\begin{definicion}{Literal}{}
    Se denomina literal a las fórmulas que son variables o variables negadas.
\end{definicion}


\subsection{Teorema de Compacidad}

\begin{teorema}{Teorema de Compacidad}{compacidad}
    Sea $\Gamma$ un conjunto.

    \medskip

    \begin{center}
        $\Gamma$ es satisfacible $\iff \Gamma$ es finitamente satisfacible
    \end{center}
\end{teorema}

\begin{itemize}[align=right]
    \item $\implies$)
        \begin{align*}
            \Gamma \text{ es satisfacible} \implies& \exists \; v 
            \text{ valuación} / v(\alpha) = 1 \\
            \implies& v(\alpha)=1 \notamath{$\forall \alpha \in \Gamma$}
        \end{align*}

        Sea $\Sigma = \{ \alpha_1, \dotsc, \alpha_n \} \subseteq \Gamma$

        $v(\alpha_i) = 1$, $1 \leq i \leq n$, pues $\alpha_i \in \Gamma$
        $\implies v(\Sigma)=1$

        \begin{gather*}
            \therefore ~ \Gamma \text{ es f.s.}
        \end{gather*}

    \item $\impliedby$) Tenemos por dato que $\Gamma$ es finitamente 
        satisfacible.

        Defino una sucesión creciente de conjuntos literales.

        \begin{align*}
            \Delta_0 =& \varnothing \\
            \Delta_{n+1} =& 
            \begin{cases}
                \Delta_n \cup \{ p_n \} & \text{si } \Gamma \cup \Delta_n
                                        \cup \{ p_n \} \text{ es f.s.}\\
                \Delta_n \cup \{ \neg p_n \} & \text{sino}
            \end{cases} \\
        \end{align*}    
        \begin{gather*}
            \Delta_0 \subseteq \Delta_1 \subseteq \Delta_2 \subseteq \Delta_3
            \subseteq \dots
        \end{gather*}


        Defino $\Delta = \bigcup_{n \in \mathbb{N}} \Delta_n$

       \begin{enumerate}
           \item $\Gamma \cup \Delta_n$ es f.s. $\forall n$

            \begin{proof}
            Por inducción en $n$.

            \begin{itemize}
                \item CB) $n = 0$

                    \begin{gather*}
                         \Gamma \cup \Delta_0 = \Gamma \text{ es f.s.}
                         \notamath{Dato}
                    \end{gather*}
                \item HI) $\Gamma \cup \Delta_n$ es f.s.
                \item T) $\Gamma \cup \Delta_{n+1}$ es f.s.
            \end{itemize}

            \begin{align*}
                 \Gamma \cup \Delta_{n+1} =
                 \begin{cases}
                     \Gamma \cup \Delta_n \cup \{ p_n \} & \text{si } 
                     \underbrace{\Gamma \cup \Delta_n}_{\substack{
                     \text{Por HI}\\\text{es f.s.}}} 
                     \cup \{ p_n \} \text{ es f.s.}\\
                     \Gamma \cup \Delta_n \cup \{ \neg p_n \} & \text{sino}
                 \end{cases} \\
            \end{align*}

            Utilizando el Lema \ref{lema:gamma-pi-no-pi-esfs}:

            Si $\Gamma \cup \Delta_n \cup \{ p_j \}$ es f.s. 
            $\implies \Gamma \cup \Delta_{n+1} 
            = \Gamma \cup \Delta_n \cup \{ p_n \}$ es f.s.

            Si $\Gamma \cup \Delta_n \cup \{ \neg p_j \}$ es f.s. 
            $\implies \Gamma \cup \Delta_{n+1} 
            = \Gamma \cup \Delta_n \cup \{ \neg p_n \}$ es f.s.

            \end{proof}

            \item Sea $p_j \in \mathrm{VAR} \implies p_j \in \Delta$ o 
                $\neg p_j \in \Delta$

            \begin{proof} \phantom{.}
            
            \begin{itemize}
                \item \begin{align*}
                \text{Si } \Gamma \cup \Delta_j \cup \{ p_j \}
                \text{ es f.s.} \implies& \Delta_{j+1} 
                = \Delta_j \cup \{ p_j \} \\
                \implies& p_j \in \Delta_{j+1} \subseteq \Delta
            \end{align*}

                \item \begin{align*}
                \text{Si } \Gamma \cup \Delta_j \cup \{ p_j \}
                \text{ no es f.s.} \implies& \Delta_{j+1} 
                = \Delta_j \cup \{ \neg p_j \} \\
                \implies& \neg p_j \in \Delta_{j+1} \subseteq \Delta
            \end{align*}            
            
            \end{itemize}
            \end{proof}

            \item Defino $f:\mathrm{VAR} \to \{ 0,1 \} / f(p_j) = \begin{cases}
                    1 & \text{si } p_j \in \Delta \\
                    0 & \text{sino}
                    \end{cases}$

                \begin{gather*}
                    v_f(\Delta)=1
                \end{gather*}

                Porque:
                
                \begin{itemize}
                    \item Si  $p_j \in \Delta \implies v_f(p_j) = f(p_0)=1$
                
                    \item Si 
                    $\neg p_j \in \Delta \implies v_f(\neg p_j) = 1 - f(p_j)=1$
                \end{itemize}

            \item Veamos que $v_f(\Gamma)=1$

                \begin{proof} \phantom{.}
                
                    \nota{\textit{Noni:} ``¡No pueden utilizar la notación
                    $v_f(\Gamma)=0$ ! No es estándar y se presta a confusión:
                    no queda claro si estoy diciendo que no satisface a
                    $\Gamma$ o que no satisface a ninguna fórmula de 
                    $\Gamma$.''}%
                    Supongo que $v_f$ no satisface $\Gamma$
                    $\implies \exists \; \alpha \in \Gamma / v_f(\alpha)=0$

                    Llamemos $M = \max \{i / p_i \text{ aparece en } \alpha\}$


                \begin{enumerate}[%
                                labelindent=*,
                                style=multiline,
                                leftmargin=*,
                                align=left,
                                leftmargin=2\parindent,
                                label=Caso \arabic*)]
                    \item $v_f(p_M)=1$

                        \begin{gather*}
                            v_f(p_M)=1 
                            \underbrace{\implies}_{*}
                            \{ \alpha \} \cup \Delta_M \cup \{ p_M \} 
                            \text{ es insatisfacible}
                        \end{gather*}

                    \bigskip

                    \begin{proof}[Demostración de la implicación $*$.]
                        \phantom{.}

                    Supongo que existe $w \text{ valuación} /
                    w ( \{ \alpha \} \cup 
                    \underbrace{\Delta_M \cup \{ p_M \}}_{\Delta_{M+1}}
                    ) = 1$

                    Como $v_f(p_M)=1 \implies p_M \in \Delta 
                    \implies p_M \in \Delta_{M+1}$
                    \begin{align*}
                        \therefore ~ 
                    \frest{w}{var(\alpha)} =& \frest{v_f}{var(\alpha)} \\
                    \notamath{$w(\alpha)=1$ por la suposición.\\
                    $v_f(\alpha)=0$ por el dato en este item.}
                    \implies \underbrace{w(\alpha)}_{=1} =& 
                    \underbrace{v_f(\alpha)}_{=0} \\
                    \end{align*}

                    Lo cual es un absurdo que vino de suponer que existe una
                    variable que satisface  el conjunto
                    $\{ \alpha \} \cup \Delta_M \cup \{ p_M \}$. 

                    Por lo tanto, el mencionado conjunto es insatisfacible.

                    \end{proof}

                    \bigskip

                    Por otra parte, en este caso tenemos que
                    \begin{gather*}
                        \underbrace{\{ \alpha \} \cup \Delta_M 
                        \cup \{ p_M \}}_{\substack{\text{Finito e} \\
                        \text{insatisfacible}}}
                        \subseteq \underbrace{\Gamma \cup 
                        \Delta_{M+1}}_{\text{Es f.s.}}
                    \end{gather*}

                    Lo cual es un absurdo que vino de suponer que $v_f(p_M)=1$

                    \item $v_f(p_M)=0$

                    \begin{gather*}
                        v_f(p_M)=0 
                            \underbrace{\implies}_{*}
                            \{ \alpha \} \cup \Delta_M \cup \{ \neg p_M \} 
                            \text{ es insatisfacible}
                    \end{gather*}

                    \bigskip

                    \begin{proof}[Demostración de la implicación $*$.]
                        \phantom{.}

                    Supongo que existe $w \text{ valuación} /
                    w ( \{ \alpha \} \cup \Delta_M \cup \{ \neg p_M \}) = 1$

                    Como $v_f(p_M)=0 
                    \implies v_f(\neg p_M)=1 \text{ y } v_f(\Delta)=1
                    \implies \neg p_M \in \Delta 
                    \implies \Delta_M \cup \{ \neg p_M \} = \Delta_{M+1}$
                    \begin{align*}
                        \therefore ~ 
                        \frest{w}{var(\alpha)} =& \frest{v_f}{var(\alpha)} \\
                        \notamath{$w(\alpha)=1$ por la suposición.\\
                        $v_f(\alpha)=0$ por el dato en este item.}
                        \implies \underbrace{w(\alpha)}_{=1} =& 
                        \underbrace{v_f(\alpha)}_{=0} \\
                    \end{align*}

                    Lo cual es un absurdo que vino de suponer que existe una
                    variable que satisface  el conjunto
                    $\{ \alpha \} \cup \Delta_M \cup \{ \neg p_M \}$. 

                    Por lo tanto, el mencionado conjunto es insatisfacible.

                    \end{proof}

                    \bigskip

                    Por otra parte, en este caso tenemos que
                    \begin{gather*}
                        \underbrace{\{ \alpha \} \cup \Delta_M 
                        \cup \{ \neg p_M \}}_{\substack{\text{Finito e} \\
                        \text{insatisfacible}}}
                        \subseteq \underbrace{\Gamma \cup 
                        \Delta_{M+1}}_{\text{Es f.s.}}
                    \end{gather*}

                    Lo cual es un absurdo que vino de suponer que $v_f(p_M)=0$

                \end{enumerate}

                Por lo tanto, probamos que $v_f(\Gamma)= 1 \implies \Gamma$ es
                satisfacible.

                \end{proof}
       \end{enumerate} 
\end{itemize}

\begin{proposicion}{}{}
    \nota{\textit{Santiago:} 
    `` ``Son equivalentes'' quiere decir que si una afirmación es válida,
    todas lo son.''}%
    Son equivalentes:
    \begin{enumerate}[label=\protect\circled{\arabic*}]
        \item \nameref{teo:compacidad}
        \item $\Gamma$ es insatisfacible 
            $\iff$ $\exists \; \Gamma' \subseteq \Gamma$ 
            finito tal que $\Gamma'$ es insatisfacible.
        \item $\alpha \in C(\Gamma) \implies 
            \exists \; \Gamma'$ finito tal que $\Gamma' \subseteq \Gamma$ y
            $\alpha \in C(\Gamma')$
    \end{enumerate}
\end{proposicion}

\begin{proof} \phantom{.}

    Recordemos que el \nameref{teo:compacidad} nos dice que
    \begin{center}
        $\Gamma$ es satisfacible $\iff \Gamma$ es finitamente satisfacible
    \end{center}

    \begin{itemize}
        \item $\circled{1} \iff \circled{2}$) 
            Es el contrarecíproco.

        % \item $\circled{1} \implies \circled{2}$) Sea $\alpha \in C(\Gamma)$
        %     $\underbrace{\implies}_{\text{Ya probado}}
        %     \Gamma \cup \{ \neg \alpha \}$ es insatisfacible.

        %     \begin{gather*}
        %         \therefore ~ \text{ Por \nameref{teo:compacidad} } \exists \;
        %         \Gamma' \text{ finito insatisfacible} / 
        %         \Gamma' \subseteq \Gamma \cup \{ \neg \alpha \}
        %     \end{gather*}

        %     \begin{enumerate}[%
        %                     labelindent=*,
        %                     style=multiline,
        %                     leftmargin=*,
        %                     align=left,
        %                     leftmargin=2\parindent,
        %                     label=Caso \arabic*)]
        %         \item $\Gamma' \subseteq \Gamma$

        %             Entonces $C(\Gamma ') = \mathrm{FORM}$, $\Gamma '$ finito.

        %             Como $\Gamma'$ es insatisfacible 
        %             $\implies \alpha \in C(\Gamma')$

        %         \item $\Gamma' \nsubseteq \Gamma$

        %             \begin{gather*}
        %                 \implies \Gamma ' = \Gamma '' \cup \{ \neg\alpha \} 
        %                 \notamath{$\Gamma '' \subset \Gamma$}
        %             \end{gather*}
                    
        %             \medskip

        %             \nota{Si pruebo esto ya está, pues $\Gamma''$ es finito y
        %             Queremos ver que $\alpha \in C(\Gamma'')$. 
        %             $\Gamma'' \subset \Gamma$}%

        %             Supongo que $\alpha \notin C(\Gamma '')$

        %             \begin{align*}
        %                 \implies& \exists \; v \text{ valuación}/ 
        %                 v(\Gamma'')=1 \text{ y } v(\alpha)=0 \\
        %                 \implies& v(\Gamma '' \cup \{ \neg \alpha \} ) = 1 \\
        %                 \implies& v(\Gamma ') = 1
        %             \end{align*}

        %             Lo cual es un absurdo pues $\Gamma '$ es insatisfacible.

        %             El mismo vino de suponer que $\alpha \notin C(\Gamma'')$.
        %             Por lo tanto, $\alpha \in C(\Gamma'')$ y, entonces,
        %             $\Gamma ''$ es finito y $\Gamma'' \subset \Gamma$.
        %     \end{enumerate}


        % \item $\circled{2} \implies \circled{1}$) Como $\Gamma$ es 
        %     satisfacible $\implies \Gamma$ es finitamente satisfacible.
        %     \nota{Sale igual que cuando probamos el T. de Compacidad.}%

        %     Sea $\Gamma$ satisfacible
        %     \begin{align*}
        %         \implies& \exists v \text{ valuación} / v(\Gamma)= 1 \\
        %         \implies& v(\alpha)=1 \notamath{$\forall \alpha \in \Gamma$}
        %     \end{align*}

        %     Sea $\Sigma = \{ \alpha_1, \dotsc, \alpha_n \} \subseteq \Gamma$
        %     $\implies v(\alpha_i) = 1$, $1 \leq i \leq n$, porque 
        %     $\alpha_i \in \Gamma$
        %     \nota{Observemos que en este caso no usamos \circled{2}.}%
        %     \begin{gather*}
        %         \therefore ~ v(\Sigma) = 1
        %     \end{gather*}

        %     Por lo tanto, $\Gamma$ es f.s.

        % \bigskip

        \item $\circled{2} \implies \circled{3}$) 
            \begin{enumerate}[%
                            labelindent=*,
                            style=multiline,
                            leftmargin=*,
                            align=left,
                            leftmargin=2\parindent,
                            label=Caso \arabic*)]
                \item $\implies$)
                Si $\alpha \in C(\Gamma)$
                \begin{align*}
                    \implies& \Gamma \cup \{ \neg \alpha \}
                    \text{ es insatisfacible.}
                    \notamath{Propoiedad} \\
                    \implies& \text{ Existe } 
                    \Gamma' \subseteq \Gamma \cup \{ \neg \alpha \}
                    \text{ insatisfacible} 
                    \notamath{Por \circled{3}}
                \end{align*}

                Como $\Gamma' \cup \{ \neg \alpha \}$ es finito 
                e insatisfacible
                $\implies$ $\alpha \in C(\Gamma')$ 
                \nota{Por propiedad}%
            \item $\impliedby$)
                Tarea.
            \end{enumerate}


        % \item $\circled{3} \implies \circled{1}$) $\Gamma$ f.s. 
        %     $\implies \Gamma$ satisfacible
        %     \nota{Otra forma, usando \circled{2}}%

        %     Supongamos $\Gamma$ insatisfacible.
        %     \begin{align*}
        %         \implies& C(\Gamma) = \mathrm{FORM} \\
        %         \implies& \alpha = (p_1 \wedge \neg p_1) \in C(\Gamma)
        %     \end{align*}

        %     \begin{center}
        %         $\therefore$ Por \circled{2} existe $\Gamma' \subseteq\Gamma$,
        %         $\Gamma'$ finito, $/ (p_1 \wedge \neg p_1) \in C(\Gamma')$
        %     \end{center}

        %     Veamos que $\Gamma'$ es insatisfacible.

        %     Supongo $\Gamma'$ satisfacible.
        %     \begin{align*}
        %         \implies& \exists \; w \text{ valuación}/ w(\Gamma') = 1\\
        %         \implies& w(p_1 \wedge \neg p_1)=1
        %         \notamath{$p_1 \wedge \neg p_1 \in C(\Gamma')$}
        %     \end{align*}

        %     Lo cual es un absurdo, pues $p_1 \wedge \neg p_1$ es una 
        %     contradicción.

        %     Entonces, probamos que $\Gamma'$ es insatisfacible.

        %     Por lo tanto, $\Gamma'$ es insatisfacible y finito. Además,
        %     $\Gamma ' \subset \Gamma$

        %     \begin{center}
        %         $\therefore ~ \Gamma$ no es f.s.                
        %     \end{center}
        \item $\circled{3} \implies \circled{1}$) 
            \begin{enumerate}[%
                            labelindent=*,
                            style=multiline,
                            leftmargin=*,
                            align=left,
                            leftmargin=2\parindent,
                            label=Caso \arabic*)]
                \item $\implies)$ Tarea. Sale de la definición.
                \item $\impliedby)$ Sea $\Gamma$ f.s., queremos ver que
                    $\Gamma$ es satisfacible.

                    Supongamos que no, entonces $\Gamma$ es insatisfacible.

                    Como $(p_0 \wedge \neg p_0) \in C(\Gamma) = \mathrm{FORM}$
                    entonces, por \circled{3}, existe 
                    $\Gamma' \subseteq \Gamma$ tal que $\Gamma'$ es finito
                    y $p_0 \wedge \neg p_0 \in C(\Gamma')$.

                    Pero $\Gamma'$ no es satisfacible, pues si lo fuera
                    $(p_0 \wedge \neg p_0)$ también lo sería. ¡Absurdo!

                    Por lo tanto, $\Gamma'$ no es satisfacible $\implies$
                    $\Gamma$ no es f.s.: ¡Absurdo! $\implies$ $\Gamma$ es
                    satisfacible.
            \end{enumerate}
    \end{itemize}
\end{proof}

\pagebreak
\section{Teoría axiomática}

\begin{definicion}{Axiomas}{axiomas}
    Sean $\alpha, \beta, \gamma \in F$, se definen los siguiente axiomas:

    \medskip

    \begin{itemize}
        \item Axioma 1) $(\alpha \to (\beta \to \alpha))$
        \item Axioma 2) $((\alpha \to (\beta\to\gamma )) \to 
            ((\alpha\to\beta) \to (\alpha \to \gamma)))$
        \item Axioma 3) $((\neg \alpha \to \neg\beta) \to 
            ((\neg \alpha \to \beta) \to \alpha))$
    \end{itemize}
\end{definicion}

\bigskip
\textit{Observación:}

Los axiomas de la Definición \ref{def:axiomas} son tautologías.

\begin{proof} \phantom{.}
    Tarea.
\end{proof}


\begin{definicion}{Regla MODUS PONENS}{regla-mp}
    \begin{equation*}
    \frac{
        \begin{array}[b]{l}
            \circled{1} ~ \alpha \to \beta \\
            \circled{2} ~ \alpha
        \end{array}
    }{
            \circled{3} ~ \beta \phantom{\to \beta}
    }
    \end{equation*}

    \medskip

    Decimos que \circled{3} se obtiene por MODUS PONENS (MP) a partir
    de \circled{1} y \circled{2}.
\end{definicion}

La regla quiere decir que si tenemos como dato $\alpha\to\beta$ y 
$\alpha$ obtenemos $\beta$.

\medskip

\begin{definicion}{Sistema axiomático}{}
    Está formado por los tres \nameref{def:axiomas} y por la
    \nameref{def:regla-mp}.
\end{definicion}


\medskip

\begin{definicion}{Prueba}{}
    Sea $\alpha \in F$.

    \medskip

    Una prueba para $\alpha$ es una suseción finita de fórmulas 
    $\alpha_1, \alpha_2, \dotsc, \alpha_n$ tal que:
    \begin{enumerate}
        \item $\alpha_n = \alpha$
        \item Cada $\alpha_k$ es un axioma, o se obtiene aplicando MP a 
            $\alpha_i$ y $\alpha_j$.
            \nota{$1 \leq k \leq n$\\
            $i, j < k$}%

            Es decir, $\alpha_k$ es axioma o $\alpha_j = (\alpha_i \to \alpha_k)$.

    \end{enumerate}
\end{definicion}

\medskip

\begin{definicion}{Teorema}{}
    Sea $\alpha \in F$.

    \medskip

    $\alpha$ es demostrable si existe una prueba de $\alpha$.

    \bigskip
    En este caso, $\alpha$ se llama \textit{teorema}.
\end{definicion}

\subsubsection{Ejemplo}

Probar que $(\gamma \to \gamma)$ es demostrable mediante una prueba.

\begin{enumerate}
    \item $\alpha_1=$ $(\gamma \to ((\gamma \to \gamma) \to \gamma )) 
        \to 
        ((\gamma \to (\gamma \to \gamma))\to (\gamma \to \gamma))$
        \nota{Axioma 2}%

    \item $\alpha_2=$ $\gamma \to ( (\gamma \to \gamma) \to \gamma)$
        \nota{Axioma 1}%

    \item Aplicando MP entre 1 y 2, obtenemos:
        $\alpha_3=$ $(\gamma \to (\gamma \to \gamma)) \to (\gamma \to \gamma)$

    \item $\alpha_4=$ $(\gamma \to (\gamma \to \gamma))$
        \nota{Axioma 1}%

    \item Por último, aplicando MP entre 3 y 4, 
        $\alpha_5=$ $(\gamma \to \gamma)$
\end{enumerate}

Entonces, $\alpha_1, \alpha_2, \alpha_3, \alpha_4, \alpha_5=\gamma \to \gamma$

\subsection{Teorema}


\begin{teorema}{}{}
    Sea $\alpha \in F$.

    \medskip

    Si $\alpha$ es demostrable $\implies \alpha$ es tautología.
\end{teorema}

\begin{proof} \phantom{.}
    
    Sea $\alpha \in F$.

    Si $\alpha$ es demostrable, entonces $\exists \; \alpha_1,
    \alpha_2, \dotsc, \alpha_n=\alpha$ prueba de $\alpha$.

    Veamos que $\alpha$ es tautología por inducción en $n$, donde $n$ es la 
    longitud de la prueba.

    \begin{itemize}
        \item CB) $n=1$

            $\alpha_1$ es prueba de $\alpha$, es decir, $\alpha_1=\alpha$.

            Como es una única fórmula (no hay anteriores), la única 
            posibilidad es que $\alpha$ sea un axioma. Luego, $\alpha$ es
            tautología.
            \nota{Por la observación presentada junto a la Definición de
                \nameref{def:axiomas}.}%

        \item HI) $\exists \; \alpha_1, \dotsc, \alpha_k = \alpha$ es prueba 
            de $\alpha$ $\implies \alpha$ es tautología.

            Para escribir menos, a la HI la vamos a llamar $\mathcal{P}(k)$.

            Es decir, la hipótesis inductiva resulta que $\mathcal{P}(k)$ es
            verdadero, con $k \leq n$.

        \item T) $\mathcal{P}(n+1)$
    \end{itemize}
    

    Sea $\alpha_1, \alpha_2, \dotsc, \alpha_{n+1} = \alpha$.

    \begin{enumerate}[%
                    labelindent=*,
                    style=multiline,
                    leftmargin=*,
                    align=left,
                    leftmargin=2\parindent,
                    label=Caso \arabic*)]
        \item $\alpha_{n+1}$ es un axioma $\implies \alpha_{n+1}$  es 
            tautología.
            \nota{Por la observación mencionada en el CB.}%

        \item $\alpha_{n+1}$ se obtiene aplicando MP a $\alpha_i$ y $\alpha_j$
            siendo $i, j \leq n$

            Notemos que vamos a tener lo siguiente:
            \begin{gather*}
                \alpha_1, \alpha_2, \dotsc, \alpha_i \text{ es una prueba}\\
                \alpha_1, \alpha_2, \dotsc, \alpha_j \text{ es una prueba}
            \end{gather*}

            Esto es cierto pues, como 
            $\alpha_1, \alpha_2, \dotsc, \alpha_{n+1}=\alpha$ es una
            prueba, entonces, si tuviera una prueba de longitud 20 y me
            quedo con los primeros 10, esos primeros 10 forman una prueba de
            la última fórmula pues cada uno de ellos es un axioma o se obtiene
            aplicando MP a partir de fórmulas anteriores.

            Entonces, por HI, como $i \leq n$, $\alpha_i$ es una tautología.
            Análogamente, como $j \leq n$, $\alpha_j$ es una tautología.

            Como obtuvimos $\alpha_i$ por MP, entonces 
            $\alpha_i = (\alpha_j \to \alpha_{n+1})$.
            Análogo con $\alpha_j$.

            Quiero ver que $\alpha_{n+1}$ es tautología.

            Sea $v$ valuación.

            \begin{align*}
                1 &= v(\alpha_i) \\
                  &= v(\alpha_j \to \alpha_{n+1}) \\
                  &= \max \{ 1- \underbrace{v(\alpha_j)}_{=1}, 
                  v(\alpha_{n+1}) \} \\
                  &= v(\alpha_{n+1})
            \end{align*}

            Como lo probamos para una valuación cualquiera, probamos que
            $\alpha_{n+1}$ es una tautología.

            De esta manera queda demostrado el teorema que nos interesaba.
    \end{enumerate}
\end{proof}


\subsection{Prueba de una fórmula a partir de un conjunto de fórmulas}
 
\begin{definicion}{Prueba de $\alpha$ a partir de $\Gamma$}{}
    Sea $\Gamma \subseteq \mathrm{FORM}$, $\alpha \in F$.

    \medskip

    Decimos que $\alpha$ se deduce de $\Gamma$ si existe una sucesión finita
    de fórmulas $\alpha_1, \alpha_2, \dotsc, \alpha_n = \alpha$ que verifica
    alguno de los siguientes casos:
    \nota{Es un ``o'' entre cada caso.}%

    \begin{center}
        \begin{enumerate}[%
                        labelindent=*,
                        style=multiline,
                        leftmargin=*,
                        align=left,
                        leftmargin=2\parindent,
                        label=Caso \arabic*)]
            \item $\alpha_i \in \Gamma$ \nota{$1 \leq i \leq n$}%
            \item $\alpha_i$ es un axioma \nota{$1 \leq i \leq n$}%
            \item $\alpha_i$ se obtiene por MP a partir de anteriores:
                \nota{$1 \leq i \leq n$}%
                $\alpha_j = (\alpha_k \to \alpha_i)$ y $\alpha_k$, siendo 
                $j, k \leq i$.
        \end{enumerate}
    \end{center}

    A $\alpha_1, \alpha_2, \dotsc, \alpha_n = \alpha$ la denominamos
    prueba de $\alpha$ a partir de $\Gamma$.

    Cuando decimos que $\alpha_i \in \Gamma$ decimos que $\alpha_i$ es un 
    dato.

    \bigskip
    \textbf{Notación:}
    $\Gamma \vdash \alpha$
    
\end{definicion}


\bigskip
\textit{Observación:}
Notemos el caso particular $\Gamma = \varnothing$.

$\Gamma = \varnothing \vdash \alpha$ es equivalente a decir que $\alpha$ es
demostrable, lo cual a su vez es equivalente a decir que $\alpha$ es un
teorema. 

Este caso también se puede notar como: $\vdash \alpha$


\subsubsection{Ejemplo}

Demostrar que $\Gamma \vdash \beta$, siendo 
$\Gamma = \{\alpha, (\alpha \to \beta)\}$

\begin{enumerate}
    \item $\alpha$  \nota{Dato: $\alpha \in \Gamma$}%
    \item $(\alpha \to \beta)$ \nota{Dato: $(\alpha\to\beta)\in\Gamma$}%
    \item $\beta$ \nota{MP 1 y 2}%
\end{enumerate}

\subsection{Teorema de la deducción}

\begin{teorema}{Teorema de la deducción %
(versión axiomática)}{deduccion-axiomatica}
    Sea $\Gamma \subseteq \mathrm{FORM}$, $\alpha, \beta \in \mathrm{FORM}$

    \medskip

    \begin{gather*}
        \Gamma \vdash (\alpha \to \beta) \iff \Gamma \cup \{\alpha\} \vdash
        \beta
    \end{gather*}
\end{teorema}

\bigskip
\nota{La vamos a usar para demostrar el Teorema 
    \ref{teo:deduccion-axiomatica}.}%
\textit{Observación:}

\begin{gather*}
    \Gamma \vdash \rho 
    \underbrace{\implies}_{\cancel{\impliedby}} 
    \Gamma \cup \Sigma \vdash \rho
\end{gather*}

Si de un conjunto de fórmulas puedo deducir una fórmula, entonces también
puedo deducirla cuando agrando el conjunto de datos.
Al reducirlo no puedo asegurar esto (por eso la vuelta no vale).


\begin{proof} \phantom{.}
    \nota{Del Teorema \ref{teo:deduccion-axiomatica}.}%

    \begin{itemize}
        \item $\implies$) Como $\Gamma \vdash (\alpha\to\beta)$, entonces
            existe una prueba a partir de $\Gamma$
            de $(\alpha\to\beta)$:
            $\gamma_1, \gamma_2, \dotsc, \gamma_n = (\alpha\to\beta)$

            Por la observación anterior, 
            $\gamma_1, \dotsc, \gamma_n=(\alpha\to\beta)$ es una prueba
            a partir de $\Gamma \cup \{ \alpha \}$

            Ahora queremos ver que podemos obtener $\beta$ a partir de 
            $\Gamma \cup \{ \alpha \}$.
            \begin{enumerate}
                \item $\gamma_1$
                \item $\gamma_2$
                \item[$\vdots$]
                \item[$n$.] $\gamma_n = (\alpha \to \beta)$
                \item[$n+1$.] $\alpha$ 
                    \nota{Dato: $\alpha \in \Gamma \cup \{ \alpha \}$}%

                \item[$n+2$.] $\beta$ \nota{MP entre $n$ y $n+1$}%
            \end{enumerate}

        \item $\impliedby$) Tenemos por dato que
            $\Gamma \cup \{ \alpha \}\vdash \beta$ y queremos probar:
            $\Gamma \vdash (\alpha \to \beta)$.

            Como $\Gamma \cup \{ \alpha \} \vdash \beta \implies
            \exists \; \gamma_1, \dotsc, \gamma_n = \beta$ prueba a partir
            de $\Gamma \cup \{ \alpha \}$

            Vamos a probar por inducción en $n$, con $n$ la longitud de la
            prueba, que $\Gamma \vdash (\alpha \to \beta)$.


            \begin{itemize}
                \item CB) $n=1$

                    $\alpha_1 = \beta$ es prueba a partir de 
                    $\Gamma \cup \{ \alpha \}$


                    \begin{enumerate}[%
                                    labelindent=*,
                                    style=multiline,
                                    leftmargin=*,
                                    align=left,
                                    leftmargin=2\parindent,
                                    label=Opción \arabic*)]
                        \item $\beta \in \Gamma\cup \{ \alpha \}$
                            
                        \begin{enumerate}[%
                                        labelindent=*,
                                        style=multiline,
                                        leftmargin=*,
                                        align=left,
                                        leftmargin=2\parindent,
                                        label=Caso \arabic*)]
                        \item \begin{align*}
                            \beta=\alpha & \\
                            \implies& \text{Ya vimos que } (\alpha\to\alpha)
                            \text{ es demostrable.} \\
                            \implies& \varnothing\vdash(\alpha\to\alpha)\\
                            \implies& \Gamma\vdash(\alpha\to\alpha)
                        \end{align*}

                        \item $\beta \in \Gamma$

                        \begin{enumerate}
                            \item $\beta$ \nota{Dato: $\beta \in \Gamma$}%
                            \item $\beta\to(\alpha\to\beta)$\nota{Axioma 1}%
                            \item $\alpha\to\beta$
                                \nota{MP entre 1 y 2}%
                        \end{enumerate}

                        De esto se deduce que $\Gamma\vdash(\alpha\to\beta)$
                        \end{enumerate}

                        \item $\beta$ es un axioma.
                        \begin{enumerate}
                            \item $\beta$ axioma.
                            \item $\beta \to (\alpha\to\beta)$
                                \nota{Axioma 1}%
                            \item $\alpha \to \beta$
                                \nota{MP entre 1 y 2}%
                        \end{enumerate}

                        Así probamos que 
                        $\varnothing\vdash(\alpha\to\beta)
                        \implies \Gamma \vdash (\alpha\to\beta)$

                    \end{enumerate}

                \item HI) $\exists \; \gamma_1, \dotsc, \gamma_k = \beta$ 
                    prueba a partir de $\Gamma \cup \{ \alpha \}$
                    $\implies \Gamma \vdash (\alpha\to\beta)$

                    Llamando a todo esto $\mathcal{P}(k)$, la HI resulta
                    $\mathcal{P}(k)$ con $k \leq n$.

                \item T) $\mathcal{P}(n+1)$
            \end{itemize}


            Sea $\gamma_1, \dotsc, \gamma_{n+1} = \beta$ una prueba a partir
            de $\Gamma \cup \{ \alpha \}$, queremos ver que
            $\Gamma \vdash (\alpha \to \beta)$

            \begin{enumerate}[%
                            labelindent=*,
                            style=multiline,
                            leftmargin=*,
                            align=left,
                            leftmargin=2\parindent,
                            label=Caso \arabic*)]
                \item Si $\beta$ es axioma, caemos en el caso base.

                    \begin{enumerate}
                        \item $\beta$ \nota{Dato: $\beta$ axioma}%
                        \item $\beta\to(\alpha\to\beta)$ \nota{Axioma 1}%
                        \item $\alpha \to \beta$
                            \nota{MP entre 1 y 2}%
                    \end{enumerate}

                \item $\beta \in \Gamma \cup \{\alpha\}$
                    \begin{itemize}
                        \item $\beta=\alpha 
                            \implies \varnothing\vdash(\alpha\to\alpha)
                            \implies \Gamma\vdash(\alpha\to\alpha)$
                        \item $\beta \in \Gamma \implies$ 

                            \begin{enumerate}
                                \item $\beta$ \nota{Dato}%
                                \item $(\beta\to(\alpha\to\beta))$
                                    \nota{Axioma 1}%
                                \item $\alpha\to\beta$
                                    \nota{MP entre 1 y 2}%
                            \end{enumerate}
                    \end{itemize}

                \item $\exists \; \gamma_i, \gamma_j \text{ con } i,j\leq n /
                    \gamma_i = (\gamma_j \to \beta)$

                    Como $\gamma_1, \dotsc, \gamma_i = (\gamma_j\to\beta)$
                    es una prueba a partir de $\Gamma \cup \{ \alpha \}$
                    e $i \leq n$, entonces, por HI, 
                    $\Gamma\vdash(\alpha\to(\gamma_j\to\beta))$

                    Análogamente, como $\gamma_1, \dotsc, \gamma_j$ es una
                    prueba a partir de $\Gamma\cup \{ \alpha \}$, $j \leq n$,
                    entonces, por HI, $\Gamma \vdash(\alpha\to\gamma_j)$
            \end{enumerate}

            En resumen, ya probamos que $\Gamma \vdash (\alpha\to\gamma_j)$ y
            que $\Gamma\vdash(\alpha\to(\gamma_j\to\beta))$.
            Recordemos que queremos ver que $\Gamma\vdash(\gamma\to\beta)$.

            Armemos la siguiente prueba:

            \begin{enumerate}
                \item $(\alpha\to(\alpha_j\to\beta)) \to 
                    ((\alpha\to\alpha_j) \to (\alpha\to\beta))$
                    \nota{Axioma 2}%
                \item $(\alpha \to (\gamma_j\to\beta))$
                    \nota{$\Gamma\vdash(\alpha\to(\gamma_j\to\beta))$}%
                \item $(\alpha \to \gamma_j)\to(\alpha\to\beta)$
                    \nota{MP entre 1 y 2}%
                \item $(\alpha\to\gamma_j)$
                    \nota{Pues $\Gamma \vdash (\alpha\to\gamma_j)$}%
                \item $(\alpha\to\beta)$ \nota{MP entre 3 y 4}%
            \end{enumerate}

            \begin{gather*}
                \therefore ~ \Gamma \vdash (\alpha \to \beta)
            \end{gather*}
    \end{itemize}
\end{proof}

\subsubsection{Ejemplos}

\begin{itemize}
    \item Probar que $(\gamma \to \gamma)$ es demostrable.

    \begin{proof} \phantom{.}
    
        Queremos ver que $\varnothing\vdash(\gamma\to\gamma)$
    
        \begin{align*}
            \varnothing \vdash (\gamma \to \gamma)
            &\iff \varnothing \cup \{ \gamma \} \vdash \gamma 
            \notamath{Teorema \ref{teo:deduccion-axiomatica}}\\
            &\iff \underbrace{\{ \gamma \} \vdash \gamma}_{\text{Verdadero}}
        \end{align*}
    
        Podemos afirmar que esto último es verdadero pues:
        \begin{enumerate}
            \item $\gamma$ (dato)
            \nota{$\gamma \subseteq \{ \gamma \}$}%
        \end{enumerate}
    
    \end{proof}

    \item Dado
        $\Gamma = \{ (p_3 \to (p_4 \to p_3)) \to ((p_1 \to p_6) \to p_7) \}$.

        Dar una prueba de 
        $p_3 \to ((p_1 \to p_6) \to p_7)$
        a partir de $\Gamma$.

        \begin{align*}
            \Gamma \vdash p_3 \to ((p_1 \to p_6) \to p_7) \iff&
            \notamath{Ambos $\iff$ por \nameref{teo:deduccion-axiomatica}} \\
            \iff& \Gamma \cup \{ p_3 \} \vdash (p_1 \to p_6) \to p_7
            \\
            \iff& \Gamma \cup \{ p_3, p1 \to p_6 \} \vdash p_7
        \end{align*}

        Luego

        \begin{enumerate}
            \item $p_3 \to (p_4 \to p_3)$ \nota{Axioma 1}%
            \item $\alpha$ \nota{Hipótesis}%
            \item $(p_1 \to p_6) \to p_7$ \nota{MP 1 y 2}%
            \item $p_1 \to p_6$ \nota{Hipótesis}%
            \item $p_7$ \nota{MP 3 y 4}%
        \end{enumerate}
\end{itemize}
\bigskip
\textit{Observación:}

Podemos probar que $\alpha$ es demostrable a través de una prueba
$\implies$ tienen que hacer una prueba.

\begin{itemize}
    \item Decidir si $\Gamma \vdash \alpha$:
        \begin{itemize}
            \item[Sí:] Entonces hay que exhibir una prueba de $\alpha$ a 
                partir de $\Gamma$.
            \item[No:] Tendría que probar que no existe ninguna prueba. Pero
                esto es ``difícil''. En el parcial probamos que 
                $\alpha \notin C(\Gamma)$.
            
        \end{itemize}
\end{itemize}

\subsection{Teorema de correctitud}

\begin{teorema}{Teorema de correctitud}{correctitud}
    Sea $\Gamma \subseteq \mathrm{FORM}$, $\alpha \in F$.

    \medskip

    \begin{itemize}
        \item Versión débil:
            \begin{gather*}
                \underbrace{\alpha \text{ demostrable}}_{\vdash \gamma} 
                \implies 
                \underbrace{\alpha \text{ tautología}}_{%
                \alpha \in C(\varnothing)}
            \end{gather*}
        \item Versión fuerte:
            \begin{gather*}
                \Gamma \vdash \alpha \implies \alpha \in C(\Gamma)
            \end{gather*}
    \end{itemize}
\end{teorema}


\begin{proof} \phantom{.}

    \begin{enumerate}[%
                    labelindent=*,
                    style=multiline,
                    leftmargin=*,
                    align=left,
                    leftmargin=2\parindent,
                    label=Caso \arabic*)]
        \item $\Gamma = \varnothing$

            Ya lo demostramos.

            \begin{gather*}
                \underbrace{\varnothing \vdash \alpha}_{%
                \substack{\alpha \text{ es} \\ \text{demostrable}}} 
                \implies \underbrace{\alpha \in C(\varnothing)}_{%
                \substack{\alpha \text{ es una}\\\text{tautología}}}
            \end{gather*}
        
        \item Caso general. Tarea.
            \nota{Es similar a la demostración del Caso 1.}%
    \end{enumerate}
\end{proof}


\subsection{Consistencia}
\begin{definicion}{Consistente}{}
    Sea $\Gamma \subseteq \mathrm{FORM}$

    \medskip

    $\Gamma$ es consistente si 
    \begin{gather*}
        \nexists \; \varphi \in \mathrm{FORM}/
        \Gamma\vdash\varphi \text{ y } \Gamma \vdash \neg \varphi
    \end{gather*}
    
    \medskip

    \nota{La diferencia está en la existencia de $\varphi$.}%
    Por otra parte, decimos que $\Gamma$ es inconsistente si
    \begin{gather*}
        \exists \; \varphi \in \mathrm{FORM} /
        \Gamma \vdash \varphi \text{ y } \Gamma \vdash \neg \varphi
    \end{gather*}
\end{definicion}

\bigskip
\textit{Observación:}
\nota{Su contrarrecíproco es: 
``Si $\Gamma$ es inconsistente $\Rightarrow$ $\Gamma$ es insatisfacible.''}%
Si $\Gamma$ es satisfacible $\implies \Gamma$ es consistente.

\begin{proof} \phantom{.}

    Supongo $\Gamma$ inconsistente.

    Entonces
    \begin{align*}
        & \exists \; \varphi \in \mathrm{FORM} / 
        \Gamma \vdash \varphi \text{ y } \Gamma \vdash \neg \varphi \\
        \notamath{Por \nameref{teo:correctitud}}
        \implies& \underbrace{\varphi \in C(\Gamma)}_{\circled{1}} 
        \text{ y } \underbrace{\neg\varphi \in C(\Gamma)}_{\circled{2}}\\
        \notamath{$\Gamma$ es satisfacible}
        \implies& \exists \; v \text{ valuación}/ v(\Gamma)=1 \\
        \implies& \underbrace{v(\varphi)=1}_{\text{Por }\circled{1}}
        \text{ y } \underbrace{v(\neg\varphi)=1}_{\text{Por }\circled{2}} 
    \end{align*}

    Lo cual es un absurdo. Y el absurdo vino de suponer que $\Gamma$ es
    inconsistente.

    Por lo tanto, $\Gamma$ es consistente.

\end{proof}


\subsection{Maximal consistente}

\begin{definicion}{Maximal consistente}{}
    Sea $\Gamma \subseteq \mathrm{FORM}$.

    \medskip

    Decimos que $\Gamma$ es maximal consistente (m.c.) si $\Gamma$ es 
    consistente y 

    $\forall \alpha \in \mathrm{FORM}$:

    \begin{center}
    \begin{enumerate}[%
                    labelindent=*,
                    style=multiline,
                    leftmargin=*,
                    align=left,
                    leftmargin=2\parindent,
                    label=Caso \arabic*)]
        \item $\alpha \in \Gamma$
        \item $\exists\; \varphi \in F / \Gamma \cup \{\alpha\} \vdash \varphi$
        y $\Gamma \cup \{ \alpha \} \vdash \neg \varphi$
        \nota{$\Gamma \cup \{ \alpha \}$ es inconsistente}%
    \end{enumerate}
    \end{center}

\end{definicion}

\bigskip
\textit{Observación:}
El maximal \underline{no} es único.

\subsubsection{Ejemplo}

Dado $\{ p_0 \} = \Gamma$, son maximales consistentes $\{ p_0, \neg p_1 \}$ y
$\{ p_0, p_1 \}$.

\subsection{Lema de Lindenbaum}

\begin{lema}{Lema de Lindenbaum}{lindenbaum}
    Sea $\Gamma$ un conjunto.

    \medskip

    Si $\Gamma$ es consistente $\implies \exists \; \Gamma' \text{ m.c.}/
    \Gamma \subseteq \Gamma'$ 
\end{lema}


Este lema nos está diciendo que, dado cualquier cojunto, siempre existe un
maximal consistente que lo contiene.


\begin{proof} \phantom{.}

    \begin{enumerate}
        \item $\# \mathrm{FORM} = \aleph_0$ \nota{Ejercicio: probarlo.}%

            \begin{gather*}
                \implies \exists F: \mathbb{N}\to \mathrm{FORM} / F(n) = \varphi_n
                \notamath{$\mathrm{FORM}=\{\varphi_n / n \in \mathbb{N}\}$
                $=$ $\{ \varphi_0, \varphi_1, \dots\}$}
            \end{gather*}

        \item Defino \dashbox{$ \Gamma_0 = \Gamma $}

            Defino 
            
            \begin{gather*}
                \dashbox{
                $\Gamma_{n+1} =
                \begin{cases}
                    \Gamma_r \cup \{ \varphi_{n+1} \} & \text{si } \Gamma_n
                    \cup \{ \varphi_{n+1} \}
                                        \text{ es consistente} \\
                    \Gamma_n        & \text{sino}
                \end{cases}$
                }
            \end{gather*}

            Entonces
            \begin{gather*}
                \Gamma_0 \subseteq \Gamma_1 \subseteq \Gamma_2 \subseteq\dots
            \end{gather*}

            Es decir, $\Gamma_n \subseteq \Gamma_{n+1}$

            Defino
            \begin{gather*}
                \dashbox{$
                    \Gamma' = \bigcup_{n \in \mathbb{N}} \Gamma_n
                $}
            \end{gather*}

            Teniendo estas tres cosas definidas, quiero ver que $\Gamma'$ es
            m.c. y $\Gamma \subseteq \Gamma'$.

            \begin{enumerate}
                \item $\Gamma \subseteq \Gamma'$

                    Esto es cierto porque
                    \begin{gather*}
                        \Gamma = \Gamma_0 \subseteq \bigcup_{n\in\mathbb{N}}
                        \Gamma_n = \Gamma'
                    \end{gather*}

                \item $\Gamma_n$ es consistente $\forall n \in \mathbb{N}$.

                    Veámoslo por inducción en $n$.

                    \begin{itemize}
                        \item CB) $n=0$

                            $\Gamma_0 = \Gamma$ es consistente por hipótesis.

                        \item HI) $\Gamma_n$ es consistente.
                        \item T) $\Gamma_{n+1}$ es consistente.
                    \end{itemize}


                    Sea
                    \begin{gather*}
                        \Gamma_{n+1} = \begin{cases}
                            \Gamma_r \cup \{ \varphi_{n+1} \} & \text{si } 
                                            \Gamma_n \cup \{ \varphi_{n+1} \}
                                            \text{ es consistente} \\
                        \underbrace{\Gamma_n}_{\circled{1}}  & \text{sino}
                        \end{cases}
                    \end{gather*}

                    \circled{1} es consistente por HI.

                    Entonces, así queda probado que $\Gamma_{n+1}$ es 
                    consistente.
            \end{enumerate}

        \item $\Gamma'$ es consistente.


            Supongo que no.

            \begin{align*}
                \implies& \exists \; \varphi \in \mathrm{FORM} /
                \Gamma' \vdash\varphi \text{ y } \Gamma' \vdash \neg\varphi \\
                \implies& \exists\; \alpha_1,\alpha_2,\dotsc,\alpha_n=\varphi
                \text{ prueba a partir de $\Gamma'$} \\
                & \exists\; \beta_1,\beta_2,\dotsc,\beta_n=\neg\varphi
                \text{ prueba a partir de $\Gamma'$}
            \end{align*}

            Entonces creamos el conjunto $X$ y lo definimos

            \begin{gather*}
                \notamath{$X \subseteq \Gamma'$}
                X=\{\alpha_j / 
                1 \leq j \leq n \text{ y } \alpha_j \in \Gamma'\} \cup
                \{ \beta_j /
                1 \leq j \leq k \text{ y } \beta_j \in \Gamma' \}
            \end{gather*}

            Sea $M = \max \{ n / \varphi_n \in X \}$.

            \begin{gather*}
                \implies X \subseteq \Gamma_{M+1}
            \end{gather*}

            \begin{gather*}
                \therefore ~ \underbrace{\Gamma_{M+1} \vdash \varphi}_{%
                \substack{\alpha_1, \dotsc, \alpha_n = \varphi \\
                \text{prueba a } \\
                \text{partir de } \Gamma_{M+1}}}
                \text{ y }
                \underbrace{\Gamma_{M+1} \vdash \neg \varphi}_{%
                \substack{\beta_1, \dotsc, \beta_n = \neg \varphi \\
                \text{prueba a } \\
                \text{partir de } \Gamma_{M+1}}}
            \end{gather*}
            
            \smallskip

            Por lo tanto, $\Gamma_{M+1}$ es incosistente.

            ¡Absurdo! Pues en el item anterior (2) probamos que todos los 
            $\Gamma_M$ son consistentes.

        \item $\Gamma'$ es maximal.

            Supongo que $\exists \varphi \in \mathrm{FORM} / \varphi \notin \Gamma'$.

            Como $\varphi \in \mathrm{FORM} \implies \exists\; N \in \mathbb{N}/
            \varphi=\varphi_N$

            Luego,
            \begin{align*}
                \varphi_N \notin \Gamma' & \\
                \implies& \varphi_N \notin \Gamma_{N+1} \\
                \implies& \Gamma_N \cup \{ \varphi_N \} 
                \text{ es consistente} \\
                \implies& \exists \; \psi \in \mathrm{FORM} / 
                \Gamma_N \cup \{ \varphi_N \} \vdash \psi \text{ y }
                \Gamma_N \cup \{ \varphi_N \} \vdash \neg \psi
            \end{align*}

            Como $\Gamma_N \subseteq \Gamma'$:
            \begin{gather*}
                \Gamma' \cup \{ \varphi_N \} \vdash \psi \text{ y }
                \Gamma' \cup \{ \varphi_N \} \vdash \neg \psi
            \end{gather*}

            \nota{Está cumpliendo la definición de maximal consistente.}%
            Por lo tanto, $\Gamma'$ es maximal. 
    \end{enumerate}
\end{proof}


\bigskip
\textit{Observación:}
\begin{enumerate}
    \item $\Gamma \cup \{ \neg \varphi \} \text{  es inconsistente} \iff
        \Gamma \vdash \varphi$
    \item $\Gamma \cup \{\varphi\} \text{  es inconsistente} \iff
        \Gamma \vdash \neg \varphi$
\end{enumerate}

\begin{proof} \phantom{.}
    
    \begin{enumerate}
        \item Tarea.

        \item \phantom{.}
            \begin{itemize}
                \item[$\impliedby$)] Si $\Gamma \vdash \neg \varphi$:
                    \begin{align*}
                        \implies& \Gamma\cup\{\varphi\}\vdash \neg \varphi \\
                        \text{ y }& \Gamma\cup\{\varphi\}\vdash \varphi 
                        \notamath{Por dato.} \\
                        \implies& \Gamma \cup \{ \varphi \} 
                        \text{ es inconsistente}
                    \end{align*}

                \item[$\implies$)] Si $\Gamma \cup \{ \varphi \}$ es 
                    inconsistente:
                    \begin{align*}
                        \implies& \exists\; \psi \in \mathrm{FORM} / 
                        \Gamma \cup \{ \varphi \} \vdash \psi \text{ y }
                        \Gamma \cup \{ \varphi \} \vdash \neg \psi \\
                        \implies& \underbrace{\Gamma \vdash 
                        (\varphi\to\psi)}_{\circled{1}} 
                        \text{ y }
                        \underbrace{\Gamma \vdash (\varphi\to\neg \psi)}_{%
                        \circled{2}}
                        \notamath{Por \nameref{teo:deduccion-axiomatica}}
                    \end{align*}

                    \smallskip

                    Recordemos que estamos intentando probar que 
                    $\Gamma\vdash\neg\varphi$. Hagamos una prueba.

                    \begin{enumerate}
                        \item $(\neg\neg\varphi\to\neg\neg\psi)\to
                            ((\neg\neg\varphi\to\neg\psi)\to\neg\varphi)$
                            \nota{Axioma 3}%
                        \item $(\varphi\to\psi)\to
                            (\neg\neg\varphi\to\neg\neg\psi)$
                            \nota{Tarea.}%
                        \item $\varphi\to\psi$
                            \nota{Por \circled{1}}%
                        \item $\neg\neg\varphi \to \neg\neg \psi$
                            \nota{MP entre b y c}%
                        \item $(\neg\neg\varphi\to\neg\psi)\to\neg\varphi$
                            \nota{MP entre a y d}%
                        \item $\varphi\to\neg\psi$
                            \nota{Por \circled{2}}%
                        \item $(\varphi\to\neg\psi)\to
                            (\neg\neg\varphi\to\neg\psi)$
                            \nota{Tarea.}%
                        \item $\neg\neg\varphi\to\neg\psi$
                            \nota{MP entre f y g}%
                        \item $\neg\varphi$
                            \nota{MP entre e y h}%
                    \end{enumerate}

                    \medskip
                    \textit{Aclaración: en los pasos que quedan como tarea
                        hay que comprobar que la prueba sea demostrable.}
            \end{itemize}
    \end{enumerate}
\end{proof}

\subsection{Proposición}


\begin{proposicion}{}{}
    Sea $\Gamma'$ maximal consistente.

    \medskip

    Entonces 
    \begin{gather*}
        \forall \varphi \in \mathrm{FORM}: ~
        \varphi \in \Gamma' \text{ o } \neg \varphi \in \Gamma'
    \end{gather*}    
\end{proposicion}

\nota{Notemos que el ``\Verb+o+'' de esta proposición es excluyente.
    Si estuvieran las dos el conjunto no sería consistente.}


\begin{proof} \phantom{.}

    Supongo que el consecuente es falso.

    \begin{enumerate}[%
                    labelindent=*,
                    style=multiline,
                    leftmargin=*,
                    align=left,
                    leftmargin=2\parindent,
                    label=Caso \arabic*)]
        \item $\varphi \notin \Gamma'$ y $\neg\varphi \notin \Gamma'$

            Como $\varphi \notin \Gamma'$
            \begin{align*}
                \implies& \Gamma \cup \{\varphi\} \text{ es consistente}\\
                \implies& \Gamma' \vdash \neg\varphi
                \notamath{Por la observación anterior.} \\
            \end{align*}

            Por otra parte, como $\neg\varphi \notin \Gamma'$
            \begin{align*}
                \implies& \Gamma \cup \{\neg\varphi\} 
                \text{ es inconsistente}
                \notamath{$\Gamma'$ m.c.} \\
                \implies& \Gamma' \vdash \varphi
                \notamath{Por la observación anterior.} \\
            \end{align*}

            Por ambas cosas, $\Gamma'$ es inconsistente. Lo cual es un
            absurdo que vino de suponer que ninguna de las dos condiciones
            de pertenencia se cumplían.

        \item $\Gamma \in \Gamma'$ y $\neg \varphi \in \Gamma'$

            Como $\Gamma \in \Gamma'$ $\implies$ $\Gamma \vdash \varphi$

            Y como $\neg \varphi \in \Gamma'$ $\implies$ $\Gamma' \vdash \neg \varphi$

            Por ambas condiciones se obtiene $\Gamma'$ inconsistente.
            Lo cual es un absurdo.
    \end{enumerate}
\end{proof}

\subsection{Propiedad}

\begin{itemize}
    \item Sea $\Gamma$ m.c., entonces 
        $\Gamma \vdash \varphi \iff \varphi \in \Gamma$
\end{itemize}


\begin{proof} \phantom{.}

   \begin{itemize}
       \item $\impliedby$) 
           \begin{gather*}
               \varphi \in \Gamma \implies 
               \underbrace{\Gamma \vdash \varphi}_{\varphi \text{ es dato}}
           \end{gather*}

       \item $\implies$) Supongamos $\varphi \notin \Gamma$.

           \begin{align*}
               \implies& \Gamma \cup \{ \varphi \} \text{ es inconsistente} 
               \notamath{$\Gamma$ m.c.} \\
               \implies& \Gamma \vdash \neg \varphi
               \notamath{Observación anterior}
           \end{align*}

           Lo cual es un absurdo porque $\Gamma \vdash \varphi$ y $\Gamma$ es
           consistente por ser maximal consistente.
   \end{itemize} 

\end{proof}

\subsection{Teorema}

\begin{teorema}{}{}
    \begin{gather*}
        \Gamma \text{ consistente} \implies \Gamma \text{ satisfacible}
    \end{gather*}
\end{teorema}

\begin{proof} \phantom{.}

    Dado $\Gamma$ consistente, entonces por el \nameref{lema:lindenbaum},
    $\exists\; \Gamma'$ m.c. tal que $\Gamma \subseteq \Gamma'$


    Defino $f:\mathrm{VAR} \to \{0,1\}/f(p_j) = \begin{cases}
        1 & \text{si } p_j \in \Gamma' \\
        0 & \text{si } \neg p_j \in \Gamma'
    \end{cases}$

    Esta función está bien definida porque $\Gamma'$ es m.c. 
    $\implies \alpha \in \Gamma'$ 
    $\underbrace{\text{o}}_{\text{excluyente}}$ 
    $\neg \alpha \in \Gamma'$.
    En particular, $\forall \alpha$ $p_j \in \Gamma'$ 
    $\underbrace{\text{o}}_{\text{exc.}}$ 
    $\neg p_j \in \Gamma'$

    Sea $v_f$ la valuación que extiende a $f$. Quiero ver que $v_f(\Gamma')=1$

    Vamos a probar que $v_f(\alpha)=1 \iff \alpha \in \Gamma'$ por inducción
    en $c(\alpha)$.


    \begin{itemize}
        \item CB)
            \begin{gather*}
                c(\alpha)= 0 \implies \alpha \in \mathrm{VAR} \implies \alpha = p_j \\
                v_f(p_j)=1 \iff f(p_j)=1 \iff p_j \in \Gamma'
            \end{gather*}
        \item HI) Sea $\alpha \in F/ c(\alpha) \leq n$
            \begin{gather*}
                \notamath{El contrarrecíproco es:\\ 
                $v_f(\alpha)=0 \Leftrightarrow \alpha\notin\Gamma'$}
                v_f(\alpha)= 1 \iff \alpha \in \Gamma'
            \end{gather*}

        \item T) Sea $\alpha \in F/ c(\alpha) \leq n+1$
            \begin{gather*}
                v_f(\alpha)= 1 \iff \alpha \in \Gamma'
            \end{gather*}
    \end{itemize}

    Sea $\alpha \in F/ c(\alpha) = n+1$. Quiero ver que $v_f(\alpha) = 1
    \iff \alpha \in \Gamma'$.
    
    \begin{enumerate}[%
                    labelindent=*,
                    style=multiline,
                    leftmargin=*,
                    align=left,
                    leftmargin=2\parindent,
                    label=Caso \arabic*)]
        \item $\alpha = \neg \beta \implies c(\beta) = c(\alpha) - 1 =  n$

            \begin{align*}
                \implies v_f(\alpha)=1 & \\
                &\iff 1-v_f(\beta)=1 \\
                &\iff v_f(\beta)=0 \\
                &\iff \beta \notin \Gamma'
                \notamath{HI} \\
                &\iff \neg\beta \in \Gamma'
                \notamath{$\Gamma'$ m.c.} \\
                &\iff \alpha \in \Gamma'
            \end{align*}

        \item Sea $\alpha=(\beta_1\to\beta_2) \implies
                \underbrace{c(\beta_1)}_{\geq 0} + 
                \underbrace{c(\beta_2)}_{\geq 0} = 0 \implies
                c(\beta_i) \leq n$
            \nota{$i \in \{ 1,2 \}$}%


            \begin{itemize}
                \item $v_f(\alpha)=1$
                    \begin{align*}
                        \iff& \max \{ 1-v_f(\beta_1),v_f(\beta_2) \}=1 \\
                        \iff& v_f(\beta_1)=0 \text{ o } v_f(\beta_2)=1
                    \end{align*}

                    \begin{itemize}
                        \item $v_f(\beta_1)=0$
                            \begin{align*}
                                \implies& \beta_1 \notin \Gamma'
                                \notamath{Por HI} \\
                                \implies& \neg\beta_1 \in \Gamma'
                                \notamath{$\Gamma'$ m.c.} \\
                                \implies& \underbrace{\Gamma' \vdash 
                                    \neg \beta_1}_{\circled{1}}
                            \end{align*}

                        \smallskip

                        Tarea: 
                        $\underbrace{\varphi \vdash 
                        (\neg\beta_1\to(\beta_1\to\beta_2))}_{\circled{2}}$

                        Por \circled{1} y \circled{2} y MP:
                        \begin{align*}
                            \Gamma'\vdash(\beta_1\to\beta_2) & \\
                            \implies&\underbrace{(\beta_1\to\beta_2)}_{\alpha}
                            \in \Gamma'
                            \notamath{$\Gamma'$ m.c.}
                        \end{align*}
                    \item $v_f(\beta_2)=1$
                        \begin{align*}
                            \implies& \beta_2 \in \Gamma'
                            \notamath{Por HI} \\
                            \implies& \underbrace{\Gamma' \vdash \beta_2}_{%
                            \circled{3}}
                        \end{align*}

                        \smallskip

                        Sabemos que $\varnothing \vdash 
                        \underbrace{(\beta_2\to(\beta_1\to\beta_2))}_{%
                        \text{Axioma 1 } \circled{4}}$

                    Por \circled{3} y \circled{4} y MP:
                    \begin{align*}
                        \Gamma' \vdash(\beta_1 \to \beta_2) & \\
                        \implies&\underbrace{(\beta_1\to\beta_2)}_{\alpha}
                        \in \Gamma'
                        \notamath{$\Gamma'$ m.c.}
                    \end{align*}
                    \end{itemize}

                    Hasta acá probamos que 
                    $v_f(\alpha)=1 \implies \alpha \in \Gamma'$

                \item $v_f(\alpha)=0$
                    \begin{align*}
                        \iff& \max \{ 1-v_f(\beta_1),v_f(\beta_2) \}=0 \\
                        \iff& v_f(\beta_1)=1 \text{ y } v_f(\beta_2)=0
                    \end{align*}

                    Luego,

                    \begin{align*}
                        & v_f(\beta_1)=1 \implies \beta_1\in\Gamma' \\
                        & v_f(\beta_2)=0 \implies \beta_2\notin\Gamma'
                        \notamath{Por HI} \\
                        \implies& \beta_1 \in \Gamma'
                        \text{ y } \neg \beta_2 \in \Gamma' \\
                        \implies& \underbrace{\Gamma' \vdash \beta_1}_{%
                        \circled{5}}
                        \text{ y } \underbrace{\Gamma' \vdash \neg\beta_2}_{%
                        \circled{6}}
                    \end{align*}

                    Tarea: $\underbrace{\varnothing \vdash 
                    (\beta_1\to(\neg\beta_2\to(\neg(\beta_1\to\beta_2))))}_{%
                    \circled{7}}$


                    Por  \circled{5} y \circled{6} y MP:
                    \begin{gather*}
                        \underbrace{\Gamma' \vdash
                        (\neg\beta_2\to\neg(\beta_1\to\beta_2))}_{%
                        \circled{8}}
                    \end{gather*}

                    Por  \circled{6} y \circled{8} y MP:
                    \begin{gather*}
                        \Gamma' \vdash \neg(\beta_1 \to \beta_2)
                    \end{gather*}

                    Entonces, como $\Gamma'$ es m.c.:
                    \begin{align*}
                        \neg \overbrace{(\beta_1 \to \beta_2)}^{\alpha}
                        \in \Gamma' & \\
                        &\implies \neg \alpha \in \Gamma' \\
                        &\implies \alpha \notin \Gamma'
                    \end{align*}
            \end{itemize}
    \end{enumerate}
\end{proof}

\subsection{Teorema de completitud}

\begin{teorema}{Teorema de completitud}{completitud}
   \begin{gather*}
       \alpha \in C(\Gamma) \implies \Gamma \vdash \alpha
   \end{gather*} 
\end{teorema}


\begin{proof} \phantom{.}

    \begin{align*}
        \text{Si } \alpha \in C(\Gamma) & \\
        &\implies \Gamma \cup \{ \neg\alpha \} \text{ es insatisfacible} \\
        &\implies \Gamma \cup \{\neg\alpha\} \text{ es inconsistente} \\
        &\implies \Gamma \vdash \alpha
    \end{align*}
\end{proof}

\section{Resumen}

\begin{center}
   \renewcommand{\arraystretch}{1.5}
    \begin{tabular}{c c c}
        & Lógica proposicional & \\
        && \\
        Semántica & & Axiomática \\
        $\alpha$ tautología & 
        $\xleftrightarrow[\text{\phantom{Lógica proposicional}}]{}$ & 
        $\alpha$ es demostrable \\
        $\alpha \in C(\Gamma)$ & 
        $\xleftrightarrow[\text{\phantom{Lógica proposicional}}]{}$ & 
        $\Gamma \vdash \alpha$ \\
        $\Gamma$ satisfacible & 
        $\xleftrightarrow[\text{\phantom{Lógica proposicional}}]{}$ & 
        $\Gamma$ es consistente
    \end{tabular}
\end{center}

\section{Consistencia de un sistema axiomático}

\begin{definicion}{Consistencia de un sistema}{}
    Un sistema axiomático $S$ es consistente si
    \begin{gather*}
        \nexists \; \varphi \in \mathrm{FORM} / {\vdash}_{s}\; \varphi 
        \text{ y } {\vdash}_{s}\; \neg\varphi
    \end{gather*}

\end{definicion}

\subsection{Teorema}

\begin{teorema}{}{}
    El sistema axiomático que vimos en \nameref{chap:logica-proposicional}
    es consistente.
\end{teorema}


\begin{proof} \phantom{.}

    Supongo que es inconsistente. Entonces

    \begin{align*}
        \exists \; \varphi / \vdash \varphi \text{ y } \varphi \neg \varphi&\\
        &\implies \varphi \in c(\varnothing) \text{ y } 
        \neg\varphi \in c(\varnothing) \\
        &\implies \varphi \text{ es tautología y } 
        \neg \varphi \text{ es tautología}
    \end{align*}

    Lo cual es un absurdo que vino de suponer que el sistema axiomático
    es inconsistente.
    
    Por lo tanto el sistema axiomático presentado en 
    \nameref{chap:logica-proposicional} es consistente.

\end{proof}

% \textbf{Nota to self:} en el último video teórico de Noni, en 1:30:00 
% empieza a hacer ejercicios de la \textbf{práctica}.

\bigskip

\bigskip

\textbf{Importante:} En este capítulo finaliza la teoría para el primer 
parcial. Queda por estudiar Álgebra de Boole (del apunte de la cátedra) que 
se evalúa en el final.

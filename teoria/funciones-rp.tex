%% Copyright (c) 2022 Martín E. Zahnd
%%
%% This code is licensed under MIT license (see LICENSE.txt for details)
%%
\chapter{Funciones RP}
% \graphicspath{ {./teoria/resources/funciones-rp/} }

\section{Esquemas recursivos}

\nota{ERI}%
\begin{definicion}{Esquema recursivo Tipo I}{}
    Sea $h: \mathbb{N} \to \mathbb{N}$
    
    Sea $g: \mathbb{N}^2 \to \mathbb{N}$

    \medskip

    Decimos que $h$ se obtiene a partir de $g$ por un ERI si se puede escribir
    de la siguiente forma:
    \begin{align*}
        h(0) &= k \in \mathbb{N} \\
        h(n+1) &= g(n, h(n))
    \end{align*}
\end{definicion}

\medskip

\begin{teorema}{}{}
    Si $h$ se obtiene a partir de $g$ por un ERI y $g$ es computable, 
    entonces $h$ es computable.
\end{teorema}

\begin{proof} \phantom{.}
    \begin{align*}
        &Y \gets K \\
        [A_1] \quad &IF ~ X_1 = 0 \quad GOTO ~ E_1 
        \notamath{Def. en ej. de la práctica}\\
        &Y \gets g(Z_1, Y) \\
        &Z_1 \gets Z_1 + 1 \\
        &X_1 \gets X_1 - 1 \\
        &GOTO ~ A_1
    \end{align*}

    Notemos que
    \begin{center}
        \begin{tabular}{c | c | l}
            $X_1$ & $Z_1$ & $Y$ \\
            \hline
            $2$ & $0$ & $0$ \\
            $2$ & $0$ & $k = h(0)$ \\
            $1$ & $1$ & $g(0,k) = h(1)$ \\
            $0$ & $2$ & $g(1, g(0,k))=h(2)$
        \end{tabular}
    \end{center}
\end{proof}

\subsubsection{Ejemplo}
Demostrar mediante un ER que $f: \mathbb{N} \to \mathbb{N}/f(n) = n!$ es
computable.
\nota{$\mathrm{PROD(x,y)}$: Producto de $x$ e $y$ \\
$\mathrm{SUC}(x)$: Sucesor de $x$\\
$\Pi_j$: Proyección de la variable j-ésima}%
\begin{align*}
    f(0) &= 1 \\
    f(n+1) &= (n+1)! = (n+1) n! = (n+1) f(n) \underbrace{=}_{\text{Quiero}} 
    g(n, f(n))
\end{align*}

Defino $g: \mathbb{N}^2 \to \mathbb{N} / g(x,y) = 
\mathrm{PROD}(\mathrm{SUC}(x),y)$

\begin{align*}
    g(x,y) = \mathrm{PROD}(\mathrm{SUC}(\Pi_1 (x,y)), \Pi_2 (x,y)) \\
    g = \mathrm{PROD} \circ (\mathrm{SUC} \circ \Pi_1 \times \Pi_2)
\end{align*}

\smallskip

$g$ es computable por ser composición de funciones computables 
\nota{Ver los ejemplos en la página~\pageref{ej:funcion-suma}}%
(ya vimos que $\mathrm{PROD}$, $\mathrm{SUC}$ y $\Pi_j$ es computable).

\begin{center}
    $\therefore ~ f$ es computable por ser ERI a partir de $g$ computable.
\end{center} 

\bigskip

\nota{ERII}%
\begin{definicion}{Esquema recursivo Tipo II}{}
    Sea $h: \mathbb{N}^{n+1} \to \mathbb{N}$
    
    Sea $g: \mathbb{N}^{n+2} \to \mathbb{N}$

    Sea $q: \mathbb{N}^{n} \to \mathbb{N}$

    \medskip

    Decimos que $h$ se obtiene por ERII a partir de $g$ y $q$ si puede 
    escribirse:
    \begin{align*}
        h(x_1, x_2, \dotsc, x_n, 0) &= q(x_1, \dotsc, x_n) \\
        h(x_1, x_2, \dotsc, x_n, y+1) &= g(x_1, \dotsc, x_n, y, 
        h(x_1,\dotsc, x_n, y))
    \end{align*}
\end{definicion}

\medskip

\begin{teorema}{}{}
    Sea $g: \mathbb{N}^{n+2} \to \mathbb{N}$

    Sea $q: \mathbb{N}^{n} \to \mathbb{N}$

    Sea $h: \mathbb{N}^{n+1} \to \mathbb{N}$ tal que se obtiene por ERII a 
    partir de $q$ y $g$.

    \medskip

    Si $q$ y $g$ son computables $\implies$ $h$ es computable.
\end{teorema}

\begin{proof} \phantom{.}
    \begin{align*}
        &Y \gets q(X_1, \dotsc, X_n) \\
        [A_1] \quad &IF ~ X_{n+1} = 0 \quad GOTO ~ E_1 \\
        &Y \gets g(x_1, \dotsc, x_n, Z_1, Y) \\
        &Z_1 \gets Z_1 + 1 \\
        &X_{n+1} \gets X_{n+1} - 1 \\
        &GOTO ~ A_1
    \end{align*}
\end{proof}

\section{Funciones iniciales} \label{sec:funciones-iniciales}

Las siguientes funciones se denominan \underline{funciones iniciales}.

\begin{definicion}{Función cero}{}
    \begin{gather*}
        \mathrm{CERO}: \mathbb{N} \to \mathbb{N} / \mathrm{CERO}(x) = 0
    \end{gather*}
\end{definicion}

\medskip

\begin{definicion}{Función sucesor}{}
    \begin{gather*}
        \mathrm{SUC}: \mathbb{N} \to \mathbb{N} / \mathrm{SUC}(x) = x+1
    \end{gather*}
\end{definicion}

\medskip

\begin{definicion}{Función proyección}{}
    \begin{gather*}
        \Pi_{j}^{n}: \mathbb{N}^{n} \to \mathbb{N} / 
        \Pi_{j}^{n}(x_1, \dotsc, x_n) = x_j
        \notamath{$1 \leq j \leq n$}
    \end{gather*}
\end{definicion}

\bigskip
\textit{Observación:}
Las funciones iniciales son computables.

\begin{proof}
    \nota{En los ejemplos de programas en la sección del lenguaje S.
    Página~\pageref{ej:funcion-suma}}%
    Ya la hicimos.
\end{proof}

\section{Recursividad primitiva}

\nota{RP}%
\begin{definicion}{Recursividad primitiva}{rp}
    Una función es RP si es inicial o se obtiene aplicando finitas 
    ``operaciones válidas'' a las \nameref{sec:funciones-iniciales},
    siendo las operaciones válidas: composición, ERI y ERII.
\end{definicion}


\subsubsection{Ejemplo}

Probar que $\mathrm{SUMA}$ es RP, siendo
\begin{gather*}
    \mathrm{SUMA}: \mathbb{N}^2 \to \mathbb{N} / \mathrm{SUMA}(x,y) = x+y
\end{gather*}

\begin{proof} \phantom{.}

    \begin{align*}
        \mathrm{SUMA}(x,0) &= x \underbrace{=}_{\text{Quiero}} g(x) \\
        \mathrm{SUMA} (x,y+1) &= x + (y+1) = (x+y) + 1 \\ 
        &= \mathrm{SUMA}(x,y)+1
        \underbrace{=}_{\text{Quiero}} f(x,y,\mathrm{SUMA}(x,y))
    \end{align*}

    Defino $g: \mathbb{N} \to \mathbb{N} / g(x) = \Pi_1 (x)$
    \nota{$g=\Pi_1$ es inicial}%

    Defino $f: \mathbb{N}^3 \to \mathbb{N} / f(x,y,z) = z+1 
    = \mathrm{SUC}(\Pi_3 (x,y,z))$


    Es decir, $f = \mathrm{SUC} \circ \Pi_3$ es composición de iniciales.

    Con lo cual me queda que $\mathrm{SUMA}$ es RP pues es un ERII de $g$
    inicial y de $f$ composición de iniciales.

\end{proof}

\textbf{Cuidado:} lo siguiente está mal.
\begin{align*}
    \mathrm{SUMA}(x,2) &= \mathrm{SUC}( \mathrm{SUC} (x)) \\
    \mathrm{SUMA}(x,3) &= \mathrm{SUC}( \mathrm{SUC}( \mathrm{SUC} (x))) \\
    \implies \mathrm{SUMA}(x,y) &= \underbrace{\mathrm{SUC} \circ \dotsc 
    \circ \mathrm{SUC}}_{y \text{ veces}}(x) \\
    \implies \mathrm{SUMA} &=  \mathrm{SUC} \circ \dotsc \circ \mathrm{SUC} 
    \circ \Pi_1
\end{align*}

El error es que no podemos decir cuántas veces estamos haciendo la composición
porque depende de dónde evaluamos $\mathrm{SUM}$, y no puede depender de dónde
la estamos evaluando.

\bigskip

\begin{teorema}{}{}
    Si $f: \mathbb{N}^k \to \mathbb{N}$ es RP $\implies$ $f$ es computable.
\end{teorema}

\begin{proof} \phantom{.}

    \begin{enumerate}
        \item Las funciones iniciales son computables (ya lo vimos).
        \item Composición de computables es computable (ya lo vimos).
        \item ERI y ERII de computables es computable (ya lo vimos).
    \end{enumerate}

    Como una función $f$ RP se obtiene aplicando finitas composiciones, ERI
    y ERII a iniciales que son funciones computables, entonces $f$ es 
    computable.

\end{proof}

\bigskip
\textit{Observación:}
Si $f$ no es total $\implies$ no es RP.

\bigskip
\textit{Observación:}
Existen funciones computables que no son RP.

\begin{proof}
    No la vemos.
\end{proof}

\subsubsection{Ejemplo}

\nota{\textit{Noni}: En alguna materia de programación la implementaron, así
que ya saben que es computable.}%
\textbf{Función de Ackermann}: Sea $A: \mathbb{N}^2 \to \mathbb{N}$ tal que
\begin{align*}
     &A(0,y) = y+1 \\
     &A(x+1,0) = A(x,1) \\
     &A(x+1,y+1) = A(x, A(x+1,y))
\end{align*}

Faltaría ver que la función no es RP.

\bigskip

\begin{teorema}{}{}
   \begin{enumerate}
       \item $f$ es composición de funciones RP $\implies$ $f$ es RP.
       \item $f$ se obtiene a partir de un ERI o un ERII de funciones RP
           $\implies$ $f$ es RP.
   \end{enumerate} 
\end{teorema}

\begin{proof} \phantom{.}

    \begin{enumerate}
        \item Defino $f: \mathbb{N}^k \to \mathbb{N} / 
        f = h \circ (g_1 \times \dotsc \times g_k)$

        Donde 
        $g_1, \dotsc, g_n: \mathbb{N}^k \to \mathbb{N}$ RP y 
        $h: \mathbb{N}^n \to \mathbb{N}$ RP.

        \begin{itemize}
        \item Como $g_j$ es RP, se obtuvo aplicando $m_j$ operaciones 
            \nota{``\hyperref[def:rp]{Operaciones permitidas}'': 
                Composiciones, ERI y ERII.\\
        $1 \leq j \leq n$}%
        permitidas a funciones iniciales.

        \item Como $h$ es RP, se obtuvo aplicando $m$ operaciones permitidas a
            funciones iniciales.
            
        \end{itemize}

        $\implies f$ se obtiene aplicando $1+m+m_1+\dots + m_n \in \mathbb{N}$
        operaciones a funciones iniciales.
        \begin{center}
            $\implies f$ es RP
        \end{center}

    \item 
        \begin{enumerate}[%
                        labelindent=*,
                        style=multiline,
                        leftmargin=*,
                        align=left,
                        leftmargin=2\parindent,
                        label=Caso \arabic*)]
            \item Tenemos
                $f: \mathbb{N} \to \mathbb{N}/ f(0) = k, f(n+1)=h(n,f(n))$,
                siendo $h: \mathbb{N}^2 \to \mathbb{N}$ RP.

                Como $h$ es RP, se obtuvo aplicando $k$ operaciones permitidas
                a funciones iniciales.

                $\implies f$ se obtiene aplicando $1+k$ operaciones permitidas
                a funciones iniciales.
                \begin{center}
                    $\implies f$ es RP
                \end{center}
            \item Tenemos $f: \mathbb{N}^{k+1} \to \mathbb{N} / \substack{%
                    f(x_1, \dotsc, x_k, 0) = g(x_1, \dotsc, x_k)\\
                    f(x_1, \dotsc, x_k, y+1) = h(x_1, \dotsc, x_k,y, 
                    f(x_1, \dotsc, x_k, y))}$

                Siendo $g: \mathbb{N}^k \to \mathbb{N}$ RP y 
                $h: \mathbb{N}^{k+2} \to \mathbb{N}$ RP.

                \begin{itemize}
                \item Como $g$ es RP, se obtuvo aplicando $m_g$ operaciones 
                permitidas a funciones iniciales.

                \item Como $h$ es RP, se obtuvo aplicando $m_h$ operaciones 
                permitidas a funciones iniciales.
                    
                \end{itemize}

                $\implies f$ se obtiene aplicando $m_g+m_h+1 \in \mathbb{N}$
                operaciones a funciones iniciales.
                \begin{center}
                    $\implies f$ es RP
                \end{center}
        \end{enumerate}
    \end{enumerate}
\end{proof}

\section{Predicados RP}

\begin{definicion}{Predicado}{}
    Un \textit{predicado} $\mathcal{P}^k$ de $k$ variables es una relación $k$-aria de
    números naturales, es decir, $\mathcal{P}^k \subseteq \mathbb{N}^k$.
\end{definicion}

\medskip

\begin{definicion}{Función característica}{}
    Se asocia a $P^k$ una función $C_{P^k}$, denominada
    \textit{función característica}, y definida como:
    \begin{gather*}
        C_{P^k} : \mathbb{N}^k \to \{ 0,1 \} /
        \notamath{$\overrightarrow{x} = (x_1, \dotsc, x_n)$, con 
                    $n \in \mathbb{N}$}
        C_{P^k} (\overrightarrow{x}) = \begin{cases}
            1 & \text{si } \overrightarrow{x} \in \mathcal{P}^k\\
            0 & \text{si } \overrightarrow{x} \notin \mathcal{P}^k
        \end{cases}
    \end{gather*}

    \bigskip
    \textbf{Notación:}
    $C_{P^k} = \Chi_{P^k}$
\end{definicion}

\medskip

\begin{definicion}{Predicado computable y RP}{}
    \begin{itemize}
        \item Decimos que ``$P^k$ es computable'' si 
            $C_{P^k}$ es computable.
        \item Decimos que ``$P^k$ es RP'' si 
            $C_{P^k}$ es RP.
    \end{itemize}

\end{definicion}

\subsubsection{Ejemplo}

Dada $\mathcal{P} = \{ (x,y) \in \mathbb{N}^2 / x \leq y \} \subsetneq \mathbb{N}^2$.
Probar que $\mathcal{P}$ es RP.

Definimos

\begin{gather*}
    C_{\mathcal{P}}: \mathbb{N}^2 \to \mathbb{N} / C_{\mathcal{P}} (x,y) =
    \begin{cases}
        1 & \text{si } x \leq y \\
        0 & \text{si } x > y
    \end{cases} =
    \begin{cases}
        1 & \text{si } x \restatruncada y  = 0\\
        0 & \text{si } x \restatruncada y \neq 0
    \end{cases} = \alpha(x \restatruncada y)
\end{gather*}

\nota{$\restatruncada$ significa ``resta truncada''\\
$\alpha$ es una función de la práctica\\
Ver ecuaciones \ref{eq:anexo-restatruncada} y \ref{eq:anexo-alpha}}%
Entonces $C_{\mathcal{P}} 
= \alpha \circ \restatruncada \circ (\Pi_1 \times \Pi_2)$.

Por lo tanto, $C_{\mathcal{P}}$ es composición de $\alpha$, $\restatruncada$,
$\Pi_j$, que son RP $\implies C_{\mathcal{P}}$ es RP $\implies \mathcal{P}$ es
RP.

\subsection{Teorema}

\begin{teorema}{}{}
    \begin{enumerate}
        \item Sean $P^k$ y ${Q}^k$ predicados RP $\implies$
            $(P \wedge Q)$ y $\neg P$ son predicados RP.
        \item Sean ${P}^k$ y ${Q}^k$ predicados computables
            $\implies$ $(P \wedge Q)$ y $\neg P$ son predicados computables.
    \end{enumerate}
\end{teorema}

\begin{proof} \phantom{.}
    \begin{enumerate}
        \item $\mathcal{P}^k$ es RP $\implies$ 
            $C_{{P}}: \mathbb{N}^k \to \mathbb{N}$ es RP.

            $\mathcal{Q}^k$ es RP $\implies$ 
            $C_{Q}: \mathbb{N}^k \to \mathbb{N}$ es RP.

            $C_{\neg P}: \mathbb{N}^k \to \mathbb{N} /
            C_{\neg P} (\overrightarrow{x}) = 
            \begin{cases}
                1 & \text{si } \overrightarrow{x} \notin {P} \\
                0 & \text{si } \overrightarrow{x} \in {P}
            \end{cases}$

            \begin{align*}
                C_{\neg P} (\overrightarrow{x}) =& \, 
                1 \restatruncada C_{P}(\overrightarrow{x})\\
                \text{ó}& \\
                C_{\neg P} (\overrightarrow{x}) =& \,
                \alpha(C_{P}(\overrightarrow{x})) \implies C_{\neg P} 
                = \underbrace{\alpha}_{\text{RP}} \circ 
            \underbrace{C_P}_{\substack{\text{RP}\\ \text{(dato)}}}
            \end{align*}

            $\implies C_{\neg P}$ es RP por composición de RP.

            $\implies \neg P$ es un predicado RP.


            $C_{P \wedge Q}: \mathbb{N}^k \to \mathbb{N} /
            C_{P \wedge Q} (\overrightarrow{x}) = 
            \begin{cases}
                1 & \text{si } \overrightarrow{x} \in 
                P \cap Q \\
                0 & \text{si } \overrightarrow{x} \notin
                P \cap Q \\
            \end{cases}$

            \begin{align*}
                C_{P \wedge Q}(\overrightarrow{x}) &= 
                C_{{P}}(\overrightarrow{x}) \, . \,
                C_{{Q}}(\overrightarrow{x}) \\
                &= \mathrm{PROD}(C_{{P}}(\overrightarrow{x}), 
                    C_{{Q}}(\overrightarrow{x})) \\
                &= \mathrm{PROD}( (C_{{P}}  \times
                    C_{{Q}}) (\overrightarrow{x})) \\
                &= \underbrace{\mathrm{PROD}}_{\text{RP}} 
                \, \circ \,
                (\underbrace{C_{{P}}}_{\substack{\text{RP}\\
                \text{(dato)}}} \times 
                    \underbrace{C_{{Q}}}_{\substack{\text{RP}\\
                    \text{(dato)}}})
            \end{align*}

            $\implies C_{P \wedge Q}$ es RP por composición de funciones RP.

            $\implies P \wedge Q$ es un predicado RP.

        \item ${P}^k$ es computable $\implies$ 
            $C_{{P}}: \mathbb{N}^k \to \mathbb{N}$ es computable.

            ${Q}^k$ es computable $\implies$ 
            $C_{{Q}}: \mathbb{N}^k \to \mathbb{N}$ es computable.

            $C_{\neg P}: \mathbb{N}^k \to \mathbb{N} /
            C_{\neg P} (\overrightarrow{x}) = 
            \begin{cases}
                1 & \text{si } \overrightarrow{x} \notin {P} \\
                0 & \text{si } \overrightarrow{x} \in {P}
            \end{cases}$

            \begin{align*}
                C_{\neg P} (\overrightarrow{x}) &= 
                1 \restatruncada C_{P}(\overrightarrow{x})\\
                &\text{ó} \\
                C_{\neg P} (\overrightarrow{x}) &= 
                \alpha(C_{P}(\overrightarrow{x})) \implies C_{\neg P} 
                = \underbrace{\alpha}_{\text{Computable}} \circ 
                \underbrace{C_{{P}}}_{\substack{\text{Computable}\\
                    \text{(dato)}}}
            \end{align*}


            $\implies C_{\neg P}$ es computable por composición de funciones
            computables.

            $\implies \neg {P}$ es un predicado computable.

            \smallskip

            $C_{P \wedge Q}$ computable: Tarea.
    \end{enumerate}
\end{proof}

\begin{corolario}{}{}
    \begin{enumerate}
        \item ${P}^k$, ${Q}^k$ son predicados RP $\implies$
            $({P}^k \cup {Q}^k)$ y 
            $({P}^k \to {Q}^k)$ son RP.
        \item ${P}^k$, ${Q}^k$ son computables $\implies$
            $({P}^k \cup {Q}^k)$ y 
            $({P}^k \to {Q}^k)$ son computables.
    \end{enumerate}
\end{corolario}

\begin{proof}
    Tarea.
\end{proof}

\subsection{Teorema}
\nota{Son dos teoremas en uno. Aplica a funciones RP o a computables.}%
\begin{teorema}{}{rp-computable-funcionpartida}
    Sean $h,g: \mathbb{N}^n \to \mathbb{N}$ funciones RP (computables).

    Sea ${P}^n$ un predicado RP (computable).

    Sea $f: \mathbb{N}^n \to \mathbb{N} / f(\overrightarrow{x}) = \begin{cases}
        h(\overrightarrow{x}) & \text{si } \overrightarrow{x} \in {P}\\
        g(\overrightarrow{x}) & \text{si } \overrightarrow{x}\notin{P}
    \end{cases}$

    \medskip

    Entonces $f$ es RP (computable).
\end{teorema}

\begin{proof} \phantom{.}

    \begin{enumerate}
        \item $f(\overrightarrow{x}) = 
            h(\overrightarrow{x}) \, . \, C_{{P}}(\overrightarrow{x}) 
            + g(\overrightarrow{x}) \, . \, C_{\neg P} (\overrightarrow{x})$

            $f = \mathrm{SUMA} \circ (\mathrm{PROD}( h \times C_{{P}} ) \times  \mathrm{PROD} (g \times C_{\neg P}))$

            $f$ es composición de 
            $\underbrace{\text{suma, producto}}_{\text{RP}}$,
            $\underbrace{h, g, C_{{P}}}_{\text{RP (dato)}}$
            $\underbrace{C_{\neg {P}}}_{%
                \substack{\text{RP pues}\\ C_{{P}} \text{ es RP}}}$
            \begin{center}
                $\implies f$ es RP
            \end{center}

        \item Tarea.
    \end{enumerate}
\end{proof}


\subsubsection{Ejemplo}

Sea $F: \mathbb{N}^2 \to \mathbb{N}$ tal que $F(x,y) = \begin{cases}
    x+y & \text{si } x \leq y \\
    x\, . \, y & \text{si } x > y 
\end{cases}$

Probar que $F$ es RP.

\begin{proof} \phantom{.}

    Escribimos $F$ como:
    \begin{gather*}
        F(x,y) = 
        \begin{cases}
            g(x,y) & \text{si } (x,y) \in {P}\\
            h(x,y) & \text{si } (x,y) \notin {P}
        \end{cases}
    \end{gather*}

    $g(x,y) = x+y$, ya vimos que $g$ es RP.

    $h(x,y) = x \, . \, y$, ya vimos que $h$ es RP.

    Luego $P = \{ (x,y) \in \mathbb{N}^2 / x \leq y \}$, ya vimos que es RP.

    $\implies$ Por el Teorema \ref{teo:rp-computable-funcionpartida} $F$ es RP.
\end{proof}

\subsection{Teorema}

\begin{teorema}{}{}
    Sean $g_1, \dotsc, g_n,h : \mathbb{N}^n \to \mathbb{N}$ funciones RP
    (computables).

    Sean $P_1, \dotsc, P_m$ predicados n-arios RP (computables) tales que
    $P_i \cap P_j = \varnothing$ si $i \neq j$.

    Sea $f: \mathbb{N}^n \to \mathbb{N} / f(\overrightarrow{x}) =
    \begin{cases}
        g_1(\overrightarrow{x}) & \text{si } \overrightarrow{x} \in P_1 \\
        g_2(\overrightarrow{x}) & \text{si } \overrightarrow{x} \in P_2 \\
        \quad \vdots & \\
        g_m(\overrightarrow{x}) & \text{si } \overrightarrow{x} \in P_m \\
        h(\overrightarrow{x}) & \text{sino}
    \end{cases}$

    \medskip

    Entonces $f$ es RP (computable).
\end{teorema}

\begin{proof} \phantom{.}

    \begin{align*}
        f(\overrightarrow{x}) =& \phantom{+}
        g_1(\overrightarrow{x}) \, . \, C_{P_1} (\overrightarrow{x}) +
        g_2(\overrightarrow{x}) \, . \, C_{P_2} (\overrightarrow{x}) +
        \cdots
        +
        g_m(\overrightarrow{x}) \, . \, C_{P_m} (\overrightarrow{x}) +\\
        &+ h(\overrightarrow{x}) 
        \underbrace{\alpha(C_{P_1}(\overrightarrow{x})) \, . \, 
        \alpha(C_{P_2}(\overrightarrow{x}))
        \, . \; 
        \dotsc
        \; . \, 
        \alpha (C_{P_m}(\overrightarrow{x}))}_{1 \iff C_{P_j}=0 \; \forall j} 
        = \\
        &= \sum_{k=1}^{m} g_{k}(\overrightarrow{x}) C_{P_k}(\overrightarrow{x})
        + h(\overrightarrow{x}) \, . \, \prod_{k=1}^{m} 
        \alpha(C_{P_k}(\overrightarrow{x}))
    \end{align*}

    Entonces como $f$ es composición de 
    $\underbrace{\text{suma, producto, } \alpha}_{\text{RP}}$,
    $\underbrace{h, g_j, P_j}_{\text{RP (dato)}}$

    \begin{center}
        $\implies f$ es RP
    \end{center}
\end{proof}

\section{Predicados acotados}

\subsection{Sumatoria y productoria acotada}

\begin{teorema}{}{}
    Sea $f: \mathbb{N}^{n+1} \to \mathbb{N}$ RP (computable).

    \nota{$\mathcal{SA}_f$: Suma acotada\\
    $\mathcal{PA}_f$: Productoria acotada}%
    Sean $\mathcal{SA}_f, \mathcal{PA}_f: \mathbb{N}^{n+1} \to \mathbb{N}$ 
    tal que
    \begin{gather*}
        \mathcal{SA}_f (\overrightarrow{x}, y) = 
        \sum_{k=0}^{y} f (\overrightarrow{x}, k) 
        \quad \text{y} \quad
        \mathcal{PA}_f (\overrightarrow{x},y) 
        = \prod_{k=0}^{y} f(\overrightarrow{x},k)
    \end{gather*}

    \nota{$\overrightarrow{x} = (x_1, \dotsc, x_n)$, con 
    $n \in \mathbb{N}$}%
    
    \medskip

    Entonces $\mathcal{SA}_f$ y $\mathcal{PA}_f$ son RP (computables).

    \bigskip
    \textit{Observación:}
    Si $n=0$

    \begin{gather*}
        \mathcal{SA}_f(y) = \sum_{k=0}^y f(k)
        \quad \text{y} \quad
        \mathcal{PA}_f (y) = \prod_{k=0}^y f(k)
    \end{gather*}
\end{teorema}

\begin{proof} \phantom{.}
    \begin{enumerate}
        \item Sumatoria acotada: $n \neq 0$
            \begin{align*}
                \mathcal{SA}_f (\overrightarrow{x},0) &= \sum_{k=0}^0 f(
                \overrightarrow{x},k) = f(\overrightarrow{x}, 0)
                \overbrace{=}^{\text{Quiero}} g(\overrightarrow{x}) \\
                \mathcal{SA}_f (\overrightarrow{x},y+1) &= \sum_{k=0}^{y+1} f(
                \overrightarrow{x},k) = \sum_{k=0}^{y} f(
                \overrightarrow{x},k) + f(\overrightarrow{x}, y+1) \\
                &= \mathcal{SA}_f (\overrightarrow{x},y) + 
                f(\overrightarrow{x}, y+1) \overbrace{=}^{\text{Quiero}}
                h(\overrightarrow{x}, y, \mathcal{SA}_f(\overrightarrow{x},y))
            \end{align*}

            Siendo $g$:
            \begin{align*}
                g: \mathbb{N}^n \to \mathbb{N} / g(\overrightarrow{x}) &=
                f(\overrightarrow{x}, 0) \\
                g &= \underbrace{f}_{\substack{\text{RP}\\ \text{(Dato)}}}
                \circ \;
                \big(\underbrace{\Pi_1 \times \dots \times \Pi_n \times \circ 
                \mathrm{CERO} \circ \Pi_1}_{\text{Iniciales}} \big)
            \end{align*}

            Entonces $g$ es RP por composición de RP.

            Luego, $h$:
            \begin{align*}
                h: \mathbb{N}^{n+1} \to \mathbb{N} / h(\overrightarrow{x},y,z)
                &= z + f(\overrightarrow{x},y+1) \\
                &= \mathrm{SUMA}(f(\overrightarrow{x}, \mathrm{SUC}(y),z))\\
            \end{align*}
            \begin{gather*}
                h = \underbrace{\mathrm{SUMA}}_{\text{RP}} \circ \,
                \Big( \underbrace{f}_{\substack{\text{RP}\\ \text{(Dato)}}} 
                \circ \;
                \big( \underbrace{\Pi_1 \times \Pi_2 \times \dots \times 
                \Pi_n \times 
                (\mathrm{SUC} \circ \Pi_{n+1})}_{\text{Iniciales}} \big)
                \times \underbrace{\Pi_{n+2}}_{\text{Inicial}} \Big) 
            \end{gather*}

            $h$ es RP por composición de funciones RP.
            \begin{center}
                $\therefore ~ \mathcal{SA}_f$ es RP por escribirse mediante un
                ERII a partir de $g$ y $h$ que son RP.
            \end{center}

        \item Sumatoria acotada: $n=0$
            
            Quiero ver que $\mathcal{SA}_f (y) = \sum_{k=0}^y f(k)$

            \begin{align*}
                \mathcal{SA}_f(0) &= \sum_{k=0}^0 f(k) = f(0) 
                \overbrace{\in \mathbb{N}}^{\text{Quiero}} \\
                \mathcal{SA}_f(y+1) &= \sum_{k=0}^{y+1} f(k) =
                \sum_{k=0}^{y} f(k) + f(y+1) \\
                &= \mathcal{SA}_f(y) + f(y+1) \overbrace{=}^{\text{Quiero}}
                H(y, \mathcal{SA}_f (y))
            \end{align*}

            Defino $H: \mathbb{N}^2 \to \mathbb{N} / H(u,w) 
            = \overbrace{w + f(w+1)}^{\substack{\mathrm{SUMA}(\Pi_2(u,w),
            f(\mathrm{SUC}(\Pi_2 (w))))}}$
            \begin{gather*}
                H = \mathrm{SUMA} \circ \, \Big(
                \Pi_2 \times (f \circ ( \mathrm{SUC} \circ \Pi_2 ) )
                \Big)
            \end{gather*}

            Entonces $H$ es RP por ser composición de 
            $\underbrace{\mathrm{SUMA}}_{\text{RP}}$,
            $\underbrace{\Pi_j, \mathrm{SUC}}_{Iniciales}$, 
            $\underbrace{f}_{\substack{\text{RP}\\ \text{(Dato)}}}$
            \begin{center}
                $\implies H$ es RP
            \end{center}
            \begin{center}
                $\therefore ~ \mathcal{SA}_f$ es RP por escribirse como un ERI
                a partir de $H$ RP.
            \end{center}
    \end{enumerate}

    Tarea: Demostrar los casos $n=0$ y $n \neq 0$ para productoria y para
    computable.

\end{proof}

\subsubsection{Ejemplo}
Sea $q: \mathbb{N} \to \mathbb{N} / q(n) =$ suma de los primeros $n$ números 
pares.

Probar que $q$ es RP.

\medskip

\begin{center}
    \begin{tabular}{c c c}
        $q(1)=0$ & $q(2)=0 + 2$ & $q(3)=0 + 2 + 4$ \\
    \end{tabular}

    $q(0) = ? \rightarrow$ No tiene sentido.
\end{center}
Como el enunciado no nos dice nada, definimos $q(0)=0$
\begin{gather*}
    q(n) = \underbrace{2 \, . \, 0 + 2 \, . \, 1 + \dots + 2 \, . \, (n-1)}_{%
    \text{Los } n \text{ primeros pares}} \\
    q(n) = \sum_{k=0}^{n-1} 2 k 
    = \sum_{k=0}^{n-1} \overbrace{\mathrm{PROD}(2,k)}^{f(k)} 
    \stackrel{?}{=} \sum_{k=0}^{n \restatruncada 1} f(k) \\
    q(n) = \mathcal{SA}_f (n \restatruncada 1)
\end{gather*}

Para responder a la pregunta de la ecuación anterior, notemos que si 
$n \geq 1$, entonces la igualdad se cumple; cuando $n = 0$, del lado
izquierdo de la igualdad, la suma desde 0 hasta $-1$ es cero, entonces
resulta que estamos evaluando si $0 \stackrel{?}{=} f(0)$.
Como  $f(0) = \mathrm{PROD}(2,0) = 0$, entonces $0 = f(0)$ y la igualdad
se cumple.

Entonces $f: \mathbb{N} \to \mathbb{N} / f(k) = \mathrm{PROD}(h_2(k), k)$

$f$ es RP por ser composición de 
$\underbrace{\mathrm{PROD}, \overbrace{h_2}^{\text{cte.}}}_{\text{RP}}$
\begin{gather*}
    q(n) = \mathcal{SA}_f (\restatruncada (n, \underbrace{h_1(n)}_{1})
\end{gather*}

$q$ es composición de $\underbrace{\mathcal{SA}_f}_{\substack{%
\text{RP pues }\\ f \text{ es RP}}}$,
$\underbrace{\restatruncada, h_1}_{\substack{\text{Ya vimos}\\%
\text{que son RP}}}$
\begin{center}
    $\implies q$ es RP
\end{center}

\textit{Noni}: En general el ejercicio de la práctica 5 y 6 lo hacen mal.
Sería bueno que lo muestren para ver si está bien hecho.

Muchas veces escriben como que es composición pero se olvidan de fijarse de
si la composición depende de dónde están evaluando. Cuando depende de dónde
se está evaluando \underline{no es} composición de funciones.

Notemos que acá armamos la ``suma acotada'' y no utilizamos composición
de sumas porque hasta dónde llega la suma depende de dónde evaluaba la 
función. Lo mismo con la productoria acotada.

\subsection{Cuantificadores acotados}

\begin{teorema}{Cuantificadores acotados}{}
    Sea ${P}^{k+1}$ un predicado RP (computable).

    Dadas
    \begin{enumerate}
        \item \nota{$\mathrm{EA}$: ``Existencial acotado''\\
            $\overrightarrow{x} = (x_1, \dotsc, x_k)$}%
            $\mathrm{EA}_P : \mathbb{N}^{k+1} \to \{0,1\} / 
            \mathrm{EA}_P (\overrightarrow{x},y) = \exists \, t \leq y
            \quad C_P (\overrightarrow{x},t)$

            Este predicado es verdadero $\iff$
            $C_P(\overrightarrow{x},0) = 1$ ó 
            $C_P(\overrightarrow{x},1) = 1$ ó 
            \dots ó
            $C_P(\overrightarrow{x},y) = 1$
        \item \nota{$\mathrm{UA}$: ``Universal acotado''}%
            $\mathrm{UA}_P : \mathbb{N}^{k+1} \to \{ 0,1 \} /
            \mathrm{UA}_P (\overrightarrow{x},y) = \forall t \leq y \quad
            C_P(\overrightarrow{x},t)$

            Este predicado es verdadero $\iff$
            $C_P(\overrightarrow{x},0) = 1$ y 
            $C_P(\overrightarrow{x},1) = 1$ y 
            \dots y
            $C_P(\overrightarrow{x},y) = 1$
    \end{enumerate}

    \medskip

    Entonces $\mathrm{EA}_P$ es RP (computable) y $\mathrm{UA}_P$ es RP
    (computable).

    \bigskip
    \textit{Observación:}
    En el caso $k=0$
    \begin{itemize}
        \item $\mathrm{EA}_P (y) = \exists \, t \leq y \quad C_P(t)$
        \item $\mathrm{UA}_P (y) = \forall t \leq y \quad C_P(t)$
    \end{itemize}
\end{teorema}

\begin{proof} \phantom{.}
    \begin{itemize}
        \item $\mathrm{EA}_P$
        \begin{itemize}
        \item $k \neq 0$
            \begin{gather*}
                \mathrm{EA}_P (\overrightarrow{x}, y) = 
                \alpha \Bigg( \alpha \bigg(
                \sum_{t=0}^y C_P (\overrightarrow{x},t)
                \notamath{Si la sumatoria es
                    $= 0 \overbrace{\rightarrow}^{\substack{\text{Quiero que}\\ 
                    \text{devuelva}}} 0$\\
                    Si es 
                $> 0 \rightarrow 1$\\
                Entonces
                $\phantom{=} 3 \rightarrow 1$\\
                $\phantom{=} 1 \rightarrow 1$, etc.
            Por ello aplicamos el doble $\alpha$}
                \bigg) \Bigg)
            \end{gather*}

            $\mathrm{EA}_P$ es composición de $\underbrace{\alpha}_{\text{RP}}$
            y $\underbrace{\text{suma acotada de un predicado RP}}_{\text{RP}}$

        \item $k=0$
            \begin{gather*}
                \mathrm{EA}_P (y) =
                \alpha \Bigg( \alpha \bigg(
                        \sum_{t=0}^y C_P (t)
                \bigg) \Bigg)
            \end{gather*}

            Misma justificación que el caso $k \neq 0$.
        \end{itemize}
        \begin{center}
            $\therefore ~ \mathrm{EA}_P$ es RP
        \end{center}

        \item $\mathrm{UA}_P$
        \begin{itemize}
        \item $k \neq 0$
            \begin{align*}
                \mathrm{UA}_P(\overrightarrow{x},y) &= \forall t \leq y \quad
                C_P(\overrightarrow{x},t) \\
                \mathrm{UA}_P(\overrightarrow{x},y) &= 1 \iff
                C_P(\overrightarrow{x}, 0) = 1 \text{ y }
                C_P(\overrightarrow{x}, 1) = 1 \text{ y }
                \dots \text{ y } \\
                &\phantom{= 1 \iff} ~
                C_P(\overrightarrow{x}, y) = 1 \\
                \mathrm{UA}_P(\overrightarrow{x},y) &=
                \prod_{t=0}^y C_P(\overrightarrow{x}, y)
            \end{align*}

            $\mathrm{UA}_P$ es una productoria acotada de predicados RP
            $\implies \mathrm{UA}_P$ es RP para $k \neq 0$.

        \item $k=0$

            \begin{gather*}
                \mathrm{UA}_P (y) = \prod_{t=0}^y C_P (t)
            \end{gather*}

            Idem caso $k \neq 0$.
        \end{itemize}

        \begin{center}
            $\therefore ~ \mathrm{UA}_P$ es RP
        \end{center}
    \end{itemize}
\end{proof}


\subsubsection{Ejemplo}

Probar que $\mathrm{DIV}: \mathbb{N}^2 \to \{ 0,1 \} / 
\mathrm{DIV}(x,y) = \begin{cases}
    1 & \text{si } \divides{x}{y}\\
    0 & \text{sino}
\end{cases} ~$
es RP.

Notemos que $\mathrm{DIV} (0,y) = 0$ pues la división por cero no está 
definida.

\begin{align*}
    \mathrm{DIV}(x,y) = 1 
    &\iff \exists \, t \text{ tal que } y = x \, . \, t \text{ y } x \neq 0 \\
    &\iff \exists \, t \leq y \quad \mathrm{EQ}(\mathrm{PROD}(x,t),y)
    \wedge \alpha(\alpha(x))
\end{align*}

Hasta acá tenemos un $\mathrm{AND}$ de funciones RP, pero con un pequeño
\nota{Recordemos: \\
    $\mathrm{EA}(x,y)$ $=$ \\
$=$ $\exists \; t \leq y \quad C_P (x,t)$}%
problema: lo que tenemos definido \textit{no} es el existencial acotado pues,
por la definición que dimos, el predicado $C_P$ no puede tener la $y$, que
sí tenemos en $\mathrm{EQ}$.

Entonces, a partir del predicado 
$\exists \, t \leq y \quad \mathrm{EQ}(\mathrm{PROD}(x,t),y)$ 

Definamos
\begin{align*}
    \mathrm{DIV}(x,y) &= \mathrm{EA}_P (x,y,y) \wedge \alpha(\alpha(x))\\
    C_P(x_1, x_2, y) &= \mathrm{EQ}(\mathrm{PROD}(y, x_1), x_2) \\
    \mathrm{EA}_P (x_1, x_2, y) &= \exists \, z \leq y \quad C_P(x_1, x_2, z) \\
    \mathrm{EA}_P (\underbrace{x,y}_{\circled{1}},
    \underbrace{y}_{\circled{2}}) 
    &= \exists \, z \leq y \quad C_P(\underbrace{x,y}_{\circled{1}}, z) 
    = \exists \, z \leq y \quad \mathrm{EQ}(\mathrm{PROD}(z,x),y)
\end{align*}

Notemos que para usar el existencial acotado tuvimos que emplear un truco: 
hicimos que $\mathrm{DIV}$ de dos variables (\circled{1})
sea igual a un existencial acotado con 3 variables, repitiendo la última de 
ellas (\circled{2}).
Así las variables $\circled{1}$ son las que aparecen en $C_P$.

\medskip

Entonces
\begin{gather*}
    \mathrm{DIV} = \mathrm{AND} \big(
    \mathrm{EA}_P(\Pi_1 \times \Pi_2 \times \Pi_2)
    \times \alpha \circ \alpha \circ \Pi_1 \big)
\end{gather*}

$\mathrm{DIV}$ es composición de funciones RP y, por lo tanto, $\mathrm{DIV}$ 
es RP.

\subsection{Cuantificador acotado con menor estricto}

\begin{teorema}{Cuantificador acotado con $<$}{cuantificador-acotado-menor}
    Sea $P^{n+1}$ un predicado RP (computable).

    Sean 
    \begin{enumerate}
        \item $\mathrm{EAE}_P (\overrightarrow{x},y) = \exists \, t < y \quad
            C_P (\overrightarrow{x},t)$
        \item $\mathrm{UAE}_P (\overrightarrow{x},y) = \forall t < y \quad
            C_P (\overrightarrow{x},t)$
    \end{enumerate}

    Con $\mathrm{EAE}_P, \mathrm{UAE}_P: \mathbb{N}^{n+1} \to \{ 0,1 \}$

    \medskip

    \nota{En el Teorema \ref{teo:cuantificador-acotado-menor}, 
    $\mathrm{EAE}_P$ y $\mathrm{UAE}_P$ también se pueden definir 
    de una variable.}%

    Entonces $\mathrm{EAE}_P$ y $\mathrm{UAE}_P$ son RP (computables).
\end{teorema}

\begin{proof} \phantom{.}

    Sea $\overrightarrow{x} = (x_1, \dotsc, x_n)$.

    \begin{enumerate}
    \item Defino
        \begin{align*}
            \mathrm{EAE}_P(\overrightarrow{x},y) &= \exists \, t \leq y \quad
            (\underbrace{C_P (\overrightarrow{x},t) \wedge t \neq y}_{%
            C_Q(\overrightarrow{x},y,t)}) \\
            &= \exists \, t \leq y \quad C_Q (\overrightarrow{x},y,t)
        \end{align*}

        Entonces, para deshacernos de $y$ escribimos
        \begin{align*}
            \mathrm{EAE}_P(\overrightarrow{x},y) &= 
            \mathrm{EA}_Q(\overbrace{\overrightarrow{x}}^{\circled{1}},
            \overbrace{y}^{\circled{2}}, \overbrace{y}^{\circled{3}}) = \\
            &= \exists \, t \leq \underbrace{y}_{\circled{3}} \quad 
                (C_P(\underbrace{\overrightarrow{x}}_{\circled{1}}, t) \wedge
            \alpha(\mathrm{EQ}(t, \underbrace{y}_{\circled{2}})))
        \end{align*}

        Entonces $\mathrm{EAE}_Q$ es $\mathrm{EA}_Q$ de $Q$ que es RP.
        \begin{center}
            $\implies \mathrm{EAE}_P$ es RP
        \end{center}

        \medskip

        \underline{Otra manera:}

        \begin{align*}
            \mathrm{EAE}_P (\vec{x}, y) &\underbrace{=}_{\text{Si } y\geq 1} 
            \mathrm{EA}_P (\vec{x}, y \restatruncada 1) 
            \, . \, \alpha(\alpha(y)) \\
            \mathrm{EAE}_P (\vec{x}, 0) &= 
            \exists \; t < 0 \quad C_P (\vec{x}, t) = 0 \\
            \mathrm{EA}_P (\vec{x}, 0) &= 
            \exists \; t \leq 0 \quad C_P (\vec{x}, t) = C_P (x,0) \\
        \end{align*}

        Entonces $\mathrm{EAE}_P$ es RP por ser composición de $\mathrm{EA}_P$,
        que es RP,
        $\Pi_1$, $\Pi_2$, $\alpha$ y $\mathrm{PROD}$, que son RP.

        \item Defino
        \begin{gather*}
            \mathrm{UAE}_P(\overrightarrow{x},y) = \forall t \leq y \quad
            (\underbrace{C_P (\overrightarrow{x},t) \vee t = y}_{%
                C_Q(\overrightarrow{x},y,t)}) \\
        \end{gather*}

        Entonces, para deshacernos de $y$ escribimos

        \begin{align*}
            \mathrm{UAE}_P(\overrightarrow{x},y) &= 
            \mathrm{UA}_Q(\overbrace{\overrightarrow{x}, y}^{\circled{4}},
              \overbrace{y}^{\circled{5}}) = \\
            &= \forall t \leq \underbrace{y}_{\circled{5}} \quad 
                (C_Q(\underbrace{\overrightarrow{x}, y}_{\circled{4}}, t)
        \end{align*}

        Donde $C_Q(\overrightarrow{x}, y, t) = C_P(\overrightarrow{x},t)
        \vee \mathrm{EQ}(t,y)$

        Entonces $\mathrm{UAE}_Q$ es RP por ser $\mathrm{UA}_Q$, siendo $Q$
        RP
    \end{enumerate}
\end{proof}

\subsection{Minimización acotada}

\begin{teorema}{Minimización acotada}{}
    Sea $P^{n+1}$ un predicado RP (computable).

    Sea $\mathrm{MA}_P: \mathbb{N}^{n+1} \to \mathbb{N}$ tal que
    \begin{gather*}
        \mathrm{MA}_P (\overrightarrow{x}, y) = \begin{cases}
            \min_{t \leq y}{\{ C_P (\overrightarrow{x},t) \}} & \text{si 
            existe } t \leq y / C_P (\overrightarrow{x},t) = 1 \\
            0 & \text{sino}
        \end{cases}
    \end{gather*}

    \medskip

    Entonces $\mathrm{MA}_P$ es RP (computable).
\end{teorema}

\begin{proof} \phantom{.}

    \begin{align*}
        g(\overrightarrow{x},y) =& \alpha(C_P(\overrightarrow{x},0)) 
        + \alpha(C_P(\overrightarrow{x}, 0)) \alpha(C_P(\overrightarrow{x},1))
        \\
        &+ \alpha(C_P(\overrightarrow{x},0)) \alpha(C_P(\overrightarrow{x},1))
          \alpha(C_P(\overrightarrow{x},2)) \\
        &+ \dots
        + \alpha(C_P(\overrightarrow{x},0)) \, \cdots \, 
          \alpha(C_P(\overrightarrow{x},y)) \\
        =& \sum_{j=0}^{y} \prod_{k=0}^{j} \alpha(C_P(\overrightarrow{x},k))
        = \begin{cases}
            \min_{t \leq y}{\{ C_P (\overrightarrow{x},t) \}} & \text{si 
            existe} \\
            y+1 & \text{sino}
        \end{cases}
    \end{align*}

    Entonces
    \begin{align*}
        \mathrm{MA}_P (\overrightarrow{x},y) 
        &= g(\overrightarrow{x},y) \, . \, (g(\overrightarrow{x},y) \leq y) \\
        \mathrm{MA}_P (\overrightarrow{x},y) 
        &= g(\overrightarrow{x},y) \, . \, \mathrm{MENOR}(g(\overrightarrow{x},y) , y+1) \\
    \end{align*}

    Donde $\mathrm{MENOR}: \mathbb{N}^2 \to \mathbb{N}$ se define como
    \begin{gather*}
        \mathrm{MENOR} (x,y) =
        \begin{cases}
            1 & x < y \\
            0 & \text{sino}
        \end{cases}
    \end{gather*}

    Por lo tanto $\mathrm{MA}_P$ es RP por ser composición de $g$ (que es $\mathcal{SA}_P$, $\mathcal{PA}_P$,
    $\alpha$), $\mathrm{MENOR}$, $\mathrm{SUMA}$, $\Pi_1$, $\Pi_2$.
\end{proof}

\bigskip

Tarea: Probar que el máximo acotado es RP.

\textit{Noni}: Háganlo de dos maneras, usando mínimo acotado y sin usar
mínimo acotado. Si ambas les salen significa que etendieron bien mínimo 
acotado.

\subsubsection{Ejemplo}

Probar que $h: \mathbb{N} \to \mathbb{N} / 
h(x) = \left[\sqrt{\frac{3}{2} x}\right]$ es RP.

Recordemos la función \textit{parte entera}:

Sea $x \in \mathbb{R}$, entonces
\begin{gather*}
    [x] = \begin{cases}
        \ceil{x} & x < 0 \\
        \floor{x} & x \geq 0
    \end{cases}
\end{gather*}

Otra manera de escribirlo es:
\begin{gather*}
    [x] = t \in \mathbb{Z} / t \leq x < t + 1
\end{gather*}


Es decir,
\begin{gather*}
    [x] = \min_{t \leq x}{~ x < t + 1} = \max_{t \leq x}{~ x < t+1}
\end{gather*}

¿Por qué podemos decir que es igual al mínimo y al máximo a la vez?
Esto se debe a que $\exists ! \; t \in \mathbb{Z} / t \leq x < t + 1$,
entonces, como el elemento es único, el mínimo y el máximo es el mismo.

\begin{align*}
    h(x) =& \min_{t \leq \sqrt{\nicefrac{3}{2} x}}{~ \sqrt{\frac{3}{2} x}} 
    < t + 1 \\
    \notamath{Necesito una cota. $\nicefrac{3}{2} = 1,5 < 2 < 10$\\
    Notemos que al tomar una cota mayor el conjunto podría resultar más grande,
    entonces nos vemos obligados a trabajar con el mínimo.\\
    Recordemos que \\$\mathrm{POT}(x,y) = {(x+1)}^y$, por eso ponemos $t$}
    =& \min_{t \leq 10x}{~ \sqrt{\frac{3}{2} x}} < t + 1 \\
    =& \min_{t \leq 10x}{~ \underbrace{\mathrm{MENOR}(\mathrm{PROD}(h_3(x),x),
            \mathrm{PROD}(h_2(x), \mathrm{POT}(t,h_2(x))
    ))}_{C_P(x,t)}}
\end{align*}
\nota{}%

\begin{tcolorbox}[%
    colback=white,%
    colbacktitle=black!75!white,%
    colframe=black!25!white,%
    title=Cuenta auxiliar%
    ]

Para la última igualdad:
\begin{align*}
    0 \leq \sqrt{\frac{3}{2} x} < t + 1 &\iff \\
    &\iff \frac{3}{2} x < (t+1)^2 \qquad 
    (\sqrt{y} \geq 0 ~ \forall y \in \mathbb{R}) \\
    &\iff 3x < 2 (t+1)^2 \qquad \qquad
\end{align*}

El último paso se debe a que necesitamos que la función constante sea RP,
por lo cual tiene que devolver un natural.
\end{tcolorbox}

\medskip

Tenemos entonces que $h(x) = \mathrm{MA}_P (10 x)
= \mathrm{MA}_P (\mathrm{PROD}(h_{10}(x), x))$


Entonces $h$ es RP por ser composición de funciones RP.

\textit{Noni}: Notemos que podríamos haber elegido $h_2$ ó $h_{100}$ como cota (en lugar
de $h_{10}$) y habría estado bien igual.

Por otra parte, si bien en el resultado nos quedó que $h$ depende de $x$ y $t$,
como la función se complica demasiado en el parcial ``les vamos a permitir''
que dejen la $x$. Deberían saber, de todos modos, que se puede salvar como
vimos en los ejemplos anteriores, resultando:

\begin{gather*}
    h(x) = f(x,10 \, . \, x), f(x,y) = \mathrm{MA}_P(x,y)
\end{gather*}

Ahora sí está ``perfectamente escrito''.

Para que máximo acotado, universal acotado, existencial acotado, etc, sean RP
tiene que ser RP lo que está adentro de la función \textbf{y} la cota.


\subsection{Minimización no acotada}

\begin{teorema}{Minimización no acotada}{}
    Sea $P^{n+1}$ un predicado computable.

    Sea $A = \{ t \in \mathbb{N}/ C_P(\overrightarrow{x},t)=1 \}$

    \medskip

    Entonces $H(\vec{x}) = \min_{t}{~ C_P(\overrightarrow{x},t)} = 
    \begin{cases}
        \min{~ A} & \text{si } A \neq \varnothing \\
        \uparrow & \text{si } A = \varnothing
    \end{cases}$

    es parcialmente computable
\end{teorema}

\begin{proof} \phantom{.}

    \begin{align*}
        [A_1] \quad &IF ~ C_P(\overrightarrow{x},y) \neq 0 \quad GOTO ~ E_1\\
                    &Y \gets Y + 1 \\
                    &GOTO ~ A_1
    \end{align*} 

    si $C_P(x_0, y)$ es falso, ejecuta la segunda instrucción; 
    si $C_P(x_1, y)$ es falso, ejecuta la segunda instrucción;
    y así sucesivamente.

   Si para ningún $Y$ es verdadero, este programa entra en un ciclo infinito; 
   sino el programa devuelve el $Y$ por más grande que sea.

   Por esto, es una función parcialmente computable.
\end{proof}

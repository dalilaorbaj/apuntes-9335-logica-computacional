%% Copyright (c) 2022 Martín E. Zahnd
%%
%% This code is licensed under MIT license (see LICENSE.txt for details)
%%
\chapter{Funciones Computables}
% \graphicspath{ {./teoria/resources/funciones-computables/} }

\section{Función par}

\begin{teorema}{}{}
    Sea $\langle , \rangle : \mathbb{N}^2 \to \mathbb{N} / 
    \langle x, y \rangle = 2^x (2y+1) - 1$

    \medskip

    Entonces $\langle , \rangle$ es biyectiva

    \bigskip
    \textbf{Notación:}
    $\langle,\rangle$ se llama \textit{función par}.
\end{teorema}


\begin{proof} \phantom{.}

    \begin{itemize}
        \item Iny.)
            \begin{align*}
                \langle x, y \rangle = \langle a, b \rangle 
                \implies& 2^x(2y + 1) - 1 = 2^a(2b+1-1)\\
                \implies& \underbrace{2^x \vphantom{y} % Para que 'Par' quede a
                                                % la misma altura que los 
                                                % 'Impar'
                }_{\text{Par}} 
                \underbrace{2y+1}_{\text{Impar}} = 2^a
                \underbrace{2b+1}_{\text{Impar}} \\
                \implies& \notamath{TFA} x = a \text{ y } 2y+1=2b+1 \\
                \implies& x=a \text{ y } y=b \\
                \implies& (x,y) = (a,b)
            \end{align*}
        \item Sobrey.)
            Sea $z \in \mathbb{N}$. Quiero ver que existe 
            $(x,y) \in \mathbb{N}^2 / \langle x, y \rangle = z$
            \begin{align*}
                \iff& 2^x (2y+1) -1 = z \\
                \iff& 2^x(2y+1) = z+1 \\
            \end{align*}
            \nota{La función $V_p$ está definida en la 
            página \pageref{def:funcion-vp} del Anexo}%
            \begin{gather*}
                x = \max_{t} {~\divides{2^t}{(z+1)}}  = V_2(z+1) 
                \; \in \mathbb{N}
                \quad \text{ y } \quad 
                y = \left(\frac{z+1}{2^x} - 1 \right) \frac{1}{2} 
                \; \in \mathbb{N}
            \end{gather*}

            Pues $\left(\frac{z+1}{2^x} \right)$ es impar ya que le sacamos 
            todos los $2$ que había en su factorización 
            (cuando definimos $x$).

            Por lo tanto, $\langle , \rangle$ es una función biyectiva.
    \end{itemize}
\end{proof}

\subsubsection{Ejemplo}

Hallar $(x,y) / \langle x, y \rangle = 11$
\begin{align*}
    \langle x, y \rangle = 11 \iff& 2^x (2y + 1) - 1 = 11\\
    \iff& 2^x(2y+1) = 12 = 2^2 \, . \, 3 \\
    \implies& \notamath{Por TFA} x= 2 \quad \text{ y } \quad 2y+1=3 \\
    \therefore & ~ \boxed{x=2 \quad \text{ y } \quad y = 1}
\end{align*}

\section{Teorema}

\begin{teorema}{}{}
    Sean $l: \mathbb{N} \to \mathbb{N} / l(z) = x$ siendo
    $z=\langle x, y \rangle$

    \phantom{Sean }$r: \mathbb{N}\to \mathbb{N} / r(z) = y$ siendo
    $z=\langle x, y \rangle$
    
    \medskip

    $l, r$ y $\langle , \rangle$ son funciones RP.
\end{teorema}


\begin{proof} \phantom{.}

    \begin{itemize}
        \item 
            \begin{align*}
                \langle x,y \rangle =& \underbrace{2^x (2y+1)}_{\geq 1} - 1 \\
                =& 2^x (2y+1) \restatruncada 1 \\
                =& \mathrm{PROD}( \mathrm{POT} (1,x), \mathrm{SUC}(y+y))
                \restatruncada 1
            \end{align*}

            $\langle , \rangle$ es RP por ser composición de 
            $\mathrm{PROD}$, $\mathrm{POT}$,
            $\mathrm{SUC}$, $+, \restatruncada, h_1$, todas funciones RP.

        \item $l(z) = \min_{x\leq z}{~ \exists \, y \leq z \quad
                \overbrace{\langle x, y \rangle=z}^{%
            \mathrm{EQ}(\langle x, y \rangle,z)}}$

                Las cotas están bien elegidas pues:
                \begin{itemize}
                    \item Veamos que $x \leq z$:

                    Supongo que $x > z \implies x \geq z + 1 
                    \implies 2^x \geq 2^{z+1} 
                    \underbrace{>}_{\substack{\text{Por}\\%
                    \text{inducción}}} z+1$

                    Como $2y+1 \geq 1$
                    \begin{align*}
                        \implies& 2^x (2y+1) \geq 2^x > z+1 \\
                        \implies& 2^x(2y+1) - 1 > z \\
                        \implies& z > z
                    \end{align*}
                    ¡Absurdo! Por lo tanto es cierto que $x \leq z$.

                    \item Veamos que $y \leq z$:

                    Supongo que $y\geq z + 1$
                    \begin{align*}
                        \implies& 2y+1 \geq 2z+2+1 \\
                        \implies& 2y+1 \geq 2z+3 > z+1
                    \end{align*}

                    Como $2^x \geq 1$
                    \begin{align*}
                        \implies& 2^x (2y+1)  \geq 2y+1 > z+1 \\
                        \implies& 2^x(2y+1)-1 > z \\
                        \implies& z > z
                    \end{align*}
                    ¡Absurdo! Por lo tanto es cierto que $y \leq z$.
                \end{itemize}

                Entonces $l$ es una función RP por composición de mínimo 
                acotado con existencial acotado con $\langle , \rangle$,
                $\mathrm{EQ}$, que son RP.

            \item $r(z) = \min_{y\geq z}{ ~\exists \, x \leq z
                \quad \langle x, y \rangle = z}$

                Tarea.
    \end{itemize}
\end{proof}


\section{Numeración de Gödel}

\begin{definicion}{Numeración de Gödel}{}
    Sea $k \in \mathbb{N}$.

    \medskip

    Para cada $k$ se define una función
    \begin{gather*}
        [\;,\;,\;]_k: \mathbb{N}^{k+1} \to \mathbb{N} / 
        [(a_0, \dotsc, a_k)] = p_0^{a_0} \cdots p_k^{a_k}
    \end{gather*}

    \nota{$p_0 = 2$, $p_1 = 3$, $p_2 = 5$}%
    Donde $p_i$ representa el $(i+1)$-primo

    \bigskip
    \textbf{Notación:}
    $[(a_0, \dotsc, a_k)]$ se llama \textit{Número de Gödel} de 
    $(a_0, \dotsc, a_k)$.
\end{definicion}

\subsubsection{Ejemplo}

\begin{itemize}
    \item $[(3,4)] = 2^3 3^4$
    \item $[(3,0,8)] = 2^3 3^0 5^8$
    \item $[(3,1,0,4)] = 2^3 \, . \, 3^1 \, . \, 5^0 \, . \, 7^4$
    \item $[(3,1,0,4,0)] = 
        2^3 \, . \, 3^1 \, . \, 5^0 \, . \, 7^4 \, . \, 11^0$
\end{itemize}

Observemos que el mismo número de Gödel corresponde a la 4-tupla 
y a la 5-tupla.

\subsection{Teorema}

\begin{teorema}{}{}
    Las funciones que calculan los números de Gödel son inyectivas, RP, pero
    no sobreyectivas.
\end{teorema}

\begin{proof} \phantom{.}

    \begin{itemize}
        \item Sobrey.) Notemos dos cosas: 
            \begin{enumerate}
                \item $0 \notin Im [\;,\;,\;]_k$
                \item $p_{k+1} \notin Im [\;,\;,\;]_k$ 
            \end{enumerate}

            Entonces la función no es sobreyectiva.
        \item Iny.) $[(a_0, \dotsc, a_k)] = [(b_0, \dotsc, b_k)]$
            \begin{align*}
                \implies& p_0^{a_0}, p_1^{a_1}, \dotsc, p_k^{a_k} =
                p_0^{b_0}, p_1^{b_1}, \dotsc, p_k^{b_k}\\
                \implies& \notamath{por TFA. $0 \leq j \leq k$} 
                a_j = b_j
            \end{align*}
        \item RP) En la práctica vimos que la siguiente función es RP:
            $f: \mathbb{N} \to \mathbb{N} / f(n) = P_n$, siendo $P_n$ el
            $(n+1)$-ésimo primo.

            \begin{gather*}
                [(a_0, \dotsc, a_k)] = \prod_{j=0}^{k} \mathrm{POT}(
                \underbrace{f(j)}_{\geq 1} \restatruncada 1, a_j)
            \end{gather*}

            Entonces la función de Gödel es RP por ser composición de 
            funciones RP:
            la potencia; la resta truncada; la función $f$; la función 
            constante 1; el producto; y las proyecciones, que por ser 
            iniciales son RP.
    \end{itemize}
\end{proof}

\subsection{Indicadores de números de Gödel}

\begin{definicion}{Indicadores de números de Gödel}{indicadores-num-godel}
    \begin{enumerate}
        \item \nota{La función $V_p$ está definida en la página 
            \pageref{def:funcion-vp} del Anexo}%
            $\cdot \, [ \, \cdot \, ]: \mathbb{N}^2 \to \mathbb{N} /
            x[i] = V_{p_i}(x)$
        \item $|\, \cdot \,|: \mathbb{N} \to \mathbb{N} /
            |n| = \begin{cases}
                long(n) & n \neq 0\\
                0 & n = 0
            \end{cases}$

            \nota{$\alpha_k \neq 0$}%
            Donde si $n \neq 0 \implies n = p_0^{\alpha_0} p_1^{\alpha_1}
            p_2^{\alpha_2} \cdots p_k^{\alpha_k}$ 

            \begin{gather*}
                long(n) = long[(\alpha_0, \dotsc, \alpha_k)]  = k + 1
            \end{gather*}
    \end{enumerate}
\end{definicion}

\subsubsection{Ejemplo}

\begin{itemize}
    \item $long(44)$ 
        \begin{align*}
            =& long(2^2 \, . \, 11) \\
            =& long([2,0,0,0,1]) = 5 \\
            =& long([2,0,0,0,1,0]) = 5
        \end{align*}
\end{itemize}

\subsection{Teorema}

\begin{teorema}{}{}
    Las funciones denominadas ``\nameref{def:indicadores-num-godel}''
    son RP.
\end{teorema}

\begin{proof} \phantom{.}

    \begin{enumerate}
        \item $x[i] = 
            \overbrace{\min_{t \leq x}{~ \notdivides{p_{i}^{t+1}}{x}}}^{%
            \circled{1}} 
            = \max_{t \leq x}{~ \divides{p_i^t}{x}}$
            
            $t \leq x \text{ porque } p^x > x \; \forall p \text{ primo } 
                \implies \notdivides{p^x}{x}$

            A $\notdivides{p_{i}^{t+1}}{x}$ lo puedo pensar como

            \begin{gather*}
                \underbrace{\alpha( \mathrm{DIV}(p_i^{t+1}, x))}_{%
                C_P(i,t,x)}
                = \alpha(\mathrm{DIV}(\mathrm{POT}(p_i \restatruncada 1, t+1),
                x))
            \end{gather*}

            Que es un predicado RP (por composición de funciones RP).

            Por lo tanto el mínimo en \circled{1} es el mínimo acotado de
            un predicado RP y, en consecuencia, $x[i]$ es RP.

        \item $|n| = \max_{t \leq n}{~ (\divides{p_t}{n})}$
            $= \left(\max_{t \leq n}{~ \mathrm{DIV}(h(k+1), n)}\right) \,
            \alpha(\alpha(n)) + \mathrm{EQ}(n,1)$

            Tarea: formalizar.
    \end{enumerate}
\end{proof}

\section{Codificación de programas}

\begin{definicion}{Codificación de programas}{}
    Cada instrucción en un programa queda determinada por tres números:
    \begin{enumerate}[label=\protect\circled{\arabic*}]
        \item \textbf{Variables}: $Y, X_1, Z_1, X_2, Z_2, \dotsc$
    
        \item \textbf{Etiquetas}: 
            $A_1, B_1, C_1, D_1, E_1, A_2, B_2, C_2, D_2, \dotsc$
    
        \item \textbf{Instrucciones}:
        \begin{align*}
            &V \gets V + 1 \\
            &V \gets V - 1 \\
            &IF ~ V \neq 0 \quad GOTO ~ L \\
            &V \gets V
        \end{align*}
    \end{enumerate}

    \bigskip

    Definimos $\#I : \{ \text{Instrucciones} \} \to \mathbb{N}$ como:

    \begin{gather*}
        \# I_{a,b,c} = \langle a, \langle b, c \rangle \rangle
    \end{gather*}

    Donde
    \begin{gather*}
        a = \begin{cases}
            0 & \text{si } I \text{ no tiene etiqueta} \\
            \# L  \notamath{$\# L$: Pos. de la etiqueta en la lista 
                            \circled{2}}
              & \text{si tiene etiqueta}
        \end{cases}
    \end{gather*}
    
    \begin{gather*}
        b = \begin{cases}
            0 & V \gets V = I\\
            1 & V \gets V + 1 = I\\
            2 & V \gets V - 1 = I\\
            \# L + 2
              & \text{si } I = IF ~ V \neq 0 \quad GOTO ~ L
        \end{cases}
    \end{gather*}

    \begin{gather*}
        \notamath{$\#V$: Pos. de la variable en la lista \circled{1}}
        c = \# V - 1
    \end{gather*}
    
    \medskip

    Además, tomamos por válido que 
    \begin{gather*}
        I \leftrightarrow (a,b,c)
        \notamath{Dada una instrucción $I$, tenemos la terna $(a,b,c)$}
    \end{gather*}
\end{definicion}

\bigskip
\textit{Observación:}
La función $\# I$ es biyectiva.

\begin{proof} \phantom{.}

    \begin{itemize}
        \item Iny.)
            \begin{align*}
                \langle a, \langle b, c \rangle \rangle 
                = \langle x, \langle y, z \rangle \rangle& \\
                &\implies a = x ~ \text{ y } ~ \langle b, c \rangle 
                = \langle y, z \rangle 
                \notamath{$\langle , \rangle$ es inyectiva}\\
                &\implies a = x, \; b = y, \; c = z 
                \notamath{$\langle , \rangle$ es inyectiva} \\
                &\implies (a,b,c) = (x,y,z) \\
                &\implies I_{(a,b,c)} = I_{(x,y,z)}
            \end{align*}
        \item Sobrey.)
            Sea $z \in \mathbb{N}$.

            Como $\langle , \rangle$ es sobreyectiva, existen 
            $x, \; b \in \mathbb{N}/ \langle x, b \rangle = z$.

            Como $\langle , \rangle$ es sobreyectiva, existen 
            $u, \; w \in \mathbb{N}/ \langle u, w \rangle = b$
            \begin{gather*}
                \implies \langle x , \langle u, w \rangle \rangle = z
            \end{gather*}
    \end{itemize}
\end{proof}

\subsubsection{Ejemplos}

\begin{itemize}
    \item Halla la instrucción de código 6

    \begin{gather*}
        \langle a, \langle b, c \rangle \rangle = 6 
        = 2^a (2 \langle b, c \rangle + 1) - 1\\
    \end{gather*}
    
    \begin{align*}
        7 = 2^a (2 \langle b, c \rangle + 1) 
        \implies& \boxed{a = 0} \text{ y } 7 
        = 2 \langle b, c \rangle + 1 \notamath{Por TFA} \\
        \implies& 3 = \langle b, c \rangle\\
    \end{align*}
    
    Luego
    \begin{align*}
        3 = 2^b (2c + 1) - 1 \implies& 4 = 2^b (2c +1) \\
        \implies& \boxed{b = 2} \text{ y } 1 = 2c + 1 \notamath{Por TFA} \\
        \implies& \boxed{c = 0}
    \end{align*}
    
    Entonces $a = 0$, $b = 2$, $c= 0$.
    
    Por otra parte:
    \begin{itemize}
        \item $a=0$: Me está diciendo que no tiene etiqueta
        \item $b=2$: $V \gets V-1$
        \item $c=0$ : $V = Y$
    \end{itemize}
    
    Por lo tanto
    \begin{align*}
        I: \quad Y \gets Y -1
    \end{align*}
    
    \item Hallar el código de
    \begin{gather*}
        [B_1] \quad IF ~ Z_3 \neq 0 \quad GOTO ~ A_1
    \end{gather*}
    
    \begin{gather*}
        \# I = \langle a, \langle b, c \rangle \rangle \\
        a = 2, \quad c = 7 - 1 = 6 \quad b = 2+1=3 \\
        \# I = \langle 2, \langle 3, 6 \rangle \rangle \\
        \langle 3, 6 \rangle = 2^3 (2 \, . \, 6 + 1) - 1 = 8 \, . \, 13 - 1 = 103\\
        \# I = \langle 2, 103 \rangle = 2^2 (2 \, . \, 103 + 1 ) - 1 
        = 4(207) - 1 = 828-1=827
    \end{gather*}
\end{itemize}

\subsection{Codificación de programas}

\begin{definicion}{Codificación de programas}{codif-programas}
    Sea $\mathcal{P}$ un programa con $I_1, I_2, \dotsc, I_k$ instrucciones.

    \medskip

    Definimos
    \begin{gather*}
        \# \mathcal{P}: \{ \text{Programas} \} \to \mathbb{N} / 
        \# \mathcal{P} = [(\#I_1, \dotsc, \#I_k)] - 1
    \end{gather*}
\end{definicion}

\subsubsection{Ejemplo}

Hallar $\# \mathcal{P}$ con $\mathcal{P}$:
\begin{align*}
    [B_1] \quad &X_1 \gets X_1 - 1 \\
    &IF ~ X_1 \neq 0 \quad GOTO ~ B_1
\end{align*}

\begin{align*}
    \#I_1 =& \langle 2, \langle 2, 1 \rangle \rangle 
    = \langle 2, 11 \rangle = 2^2 (2 \, . \, 11 + 1) - 1 = 91
    \notamath{$\langle 2, 1 \rangle = 2^2 (1 \, . \, 1 + 1)-1 = 11$ \\
    $\langle 4, 1 \rangle = 2^4 (2 \, . \, 1 + 1) -1 = 47$}\\
    \#I_2 =& \langle 0, \langle 2+2,1 \rangle \rangle 
    = \langle 0, 47 \rangle = 2^0 (2 \, . \, 41 + 1) - 1 = 94
\end{align*}

Entonces
\begin{gather*}
    \# \mathcal{P} = 2^{91} \, . \, 3^{94} - 1
\end{gather*}

Notemos que hay un problema:

Sea $\mathcal{P}_2$
\begin{align*}
    [B_1] \quad &X_1 \gets X_1 - 1 \\
    &IF ~ X_1 \neq 0 \quad GOTO ~ B_1\\
    &Y \gets Y
\end{align*}

Como
\begin{align*}
    \notamath{$\langle 0,0 \rangle = 2^0 (2 \, . \, 0 + 1) - 1 = 0$}
    \#(Y \gets Y) = \langle 0, \langle 0, 0 \rangle \rangle = 0
\end{align*}

Entonces
\begin{align*}
    \# \mathcal{P}_2 &= 2^{91} \, . \, 3^{94} \, . \, 5^0 - 1 \\
           &= 2^{91} \, . \, 3^{94} - 1 \\
           &= \# \mathcal{P}
\end{align*}

Y notamos que la función no es inyectiva, pues tenemos dos programas distintos 
con el mismo número.

Para eliminar esta ambigüedad, la última instrucción no puede ser $Y \gets Y$
excepto que sea la única.

\begin{proof} \textit{La función propuesta en \nameref{def:codif-programas} 
    es biyectiva.}

    Queremos probar que $\#: \{ \text{Programas} \} \to \mathbb{N} / 
        \# \mathcal{P} = [(\#I_1, \dotsc, \#I_k)] - 1$ es biyectiva.

    \begin{itemize}
        \item Sobrey.) 
            \begin{itemize}
                \item $0 \in Im \#$

                    El programa $Y \gets Y$ tiene código 0.

                \item $n \in \mathbb{N}$

                    \begin{gather*}
                        \underbrace{n + 1}_{\geq \, 1} =
                        p_0^{\alpha_0} \cdots p_k^{\alpha_k} =
                        [(\alpha_0, \dotsc, \alpha_k)]
                    \end{gather*}

                    Como $\# \{ \text{Instrucciones} \} \to \mathbb{N}$ es
                    \nota{$0 \leq j \leq k$}%
                    sobreyectiva, existe $I_j / \# I_j = \alpha_j$

                    \begin{align*}
                        \implies& n+1 = [(\# I_0, \dotsc, \#I_k)] \\
                        \implies& n = [(\# I_0, \dotsc, \# I_k)] - 1
                    \end{align*}

                    \begin{gather*}
                        \therefore ~ \# \mathcal{P}_{I_0, \dotsc, I_k} = n
                    \end{gather*}
            \end{itemize}

            De esta manera queda probada la sobreyectividad.

        \item Iny.)
            $\# \underbrace{\mathcal{P}_{I_0, \dotsc, I_k}}_{\mathcal{P}_1} = 
            \# \underbrace{\mathcal{P}_{\widetilde{I}_0, \dotsc, 
            \widetilde{I}_t}}_{\mathcal{P}_2}$
            \begin{gather*}
                    \implies [(\# I_0, \dotsc, \# I_k)] - 1 =
                    [(\# \widetilde{I}_0 \dotsc, \# \widetilde{I}_t)] - 1 \\
            \end{gather*}

            \begin{itemize}
                \item $k \neq 0 \implies \mathcal{P}_1$ tiene al menos 2
                    instrucciones.
                \begin{align*}
                    \implies& \# I_k \neq 0 \\
                    \implies& \divides{p_k}{%
                        [(\# \widetilde{I}_0, \dotsc, 
                    \# \widetilde{I}_t)]} \\
                    \text{Análogamente } &
                    \divides{p_t}{[(\# I_0, \dotsc, \# I_k)]} \\
                    \implies& k = t
                \end{align*}

                Como las funciones de Gödel son inyectivas
                \begin{gather*}
                    \# I_j = \# \widetilde{I}_j \implies I_j = \widetilde{I}_j
                    \notamath{$0 \leq j \leq k$}
                \end{gather*}
                \item $k = 0$

                    Si $\mathcal{P}_2$ tiene más de una instrucción va a haber 
                    un primo que divide a 
                    $[(\# \widetilde{I}_0 \dotsc, \# \widetilde{I}_t)] - 1$ 
                    pero no a  $[(\# I_0, \dotsc, \# I_k)] - 1$, entonces 
                    $\mathcal{P}_2$ también tiene que tener una única
                    instrucción.

                    Luego, al tener ambos una única instrucción, como la 
                    función de Gödel es inyectiva, tienen la misma instrucción.
            \end{itemize}
    \end{itemize}
\end{proof}


\subsubsection{Ejemplo}

Hallar el programa de código $71$

\bigskip

Tenemos que $71 = [(\#I_1, \dotsc, \#I_k)] - 1$ $\implies$ 
$72 = [(\#I_1, \dotsc, \#I_k)]$

Además, $72 = 36 \, . \, 2 = 3^2 \, . \, 2^3$ por lo que 
$[(\#I_1, \dotsc, \#I_k)] = [(3,2)]$

\begin{gather*}
    \implies \# I_1 = 3 \text{ y } \# I_2 = 2
\end{gather*}

Luego $\langle a, \langle b, c \rangle \rangle = 3 
= 2^a (2 \, . \, \langle b, c \rangle + 1) - 1$

\begin{align*}
    \implies& 4 = 2^a (2 \, . \, \langle b, c \rangle + 1)  \\
    \implies& \notamath{TFA} \boxed{a = 2} 
    \text{ y } 1 = 2 \, . \, \langle b, c \rangle + 1 \\
    \implies& \langle b, c \rangle = 0 \\
    \implies& \boxed{b = 0} \quad \boxed{c = 0}
    \notamath{Pues $\langle , \rangle$ es inyectiva}
\end{align*}

Resultando
\begin{gather*}
    I_1 : ~ [B_1] \quad Y \gets Y
\end{gather*}


Para la segunda instrucción:

\begin{gather*}
    \# I_2 = 2 = \langle a, \langle b, c \rangle \rangle
\end{gather*}

\begin{align*}
    2^a (2 \, . \, \langle b, c \rangle + 1) - 1 = 2 & \\
    \implies& 2^a (2 \, . \, \langle b, c \rangle + 1) = 3 \\
    \implies& \boxed{a=0} \notamath{TFA} \\
    \text{ y }& 2 \, . \, \langle b, c \rangle + 1 = 3 \\
    \implies& \langle b, c \rangle = 1 \\
    \implies& 2^b (2c + 1) - 1 = 1 \\
    \implies& 2^b (2c + 1 ) = 2 \\
    \implies& \notamath{TFA} \boxed{b = 1} \text{ y } 2c + 1 = 1 \\
    \implies& \boxed{c=0}
\end{align*}

Resultando
\begin{gather*}
    I_2 : ~ Y \gets Y + 1
\end{gather*}


Por lo tanto, el programa $P$ es:

\begin{align*}
    [B_1] \quad &Y \gets Y \\
    &Y \gets Y + 1
\end{align*}

\subsection{Teorema}

\begin{teorema}{}{}
    Existen funciones no computables
\end{teorema}

\begin{proof} \phantom{.}

    Sabemos que $\# \{ \text{Programas} \} = \aleph_0$
    porque encontramos una función biyectiva entre el conjunto de los 
    programas y los números naturales.

    Pero
    \begin{gather*}
        \# \{ f: \mathbb{N} \to \mathbb{N} / f \text{ es función} \} 
        = {\aleph_0}^{\aleph_0} = \mathcal{C}
    \end{gather*}

    \begin{gather*}
        \mathcal{C} > \aleph_0
    \end{gather*}

    Entonces ``no me alcanzan'' los programas para implementar todas las 
    funciones: existen funciones que no son computables.

\end{proof}

\section{Problema de la parada}

Dados un programa $\mathcal{P}$, $u_1, u_2, \dotsc, u_m$ números naturales y
$\phi_1$ un estado inicial, recordemos que si hay un cómputo 
$d_1=(1,\sigma), d_2, \dotsc, d_k$, $\Psi_p^m$ devuelve el valor de $Y$ en
$d_k$; en caso contrario, $\Psi_p^m = \uparrow$.

\medskip

\begin{definicion}{Problema de la parada}{}
    \begin{gather*}
        \mathrm{Halt}: \mathbb{N}^2 \to \mathbb{N} /
        \mathrm{Halt}(x,y) = 
        \begin{cases}
            1 & \text{si el programa de código } y \\
              & \text{termina ante la entrada } x \\
            0 & \text{sino} 
            \notamath{Es decir, devuelve $0$ si\\ 
            $\Psi_P(x) = \uparrow / \#P=y$}
        \end{cases}
    \end{gather*}
\end{definicion}

\subsubsection{Ejemplos}

\begin{enumerate}
    \item Sea $f: \mathbb{N} \to \mathbb{N} / f(x) = \mathrm{Halt}(x,0) = 1$, 
    $f = \mathrm{SUC} \circ \mathrm{cero}$
    
    \begin{gather*}
        \# P = 0 \qquad P: ~ Y \gets Y
    \end{gather*}

    \item $\mathrm{Halt}(x,71) = 1$
    \begin{align*}
        \#P = 71 \quad [B_1] \quad &Y \gets Y\\
                                   &Y \gets Y+1
    \end{align*}

     \item $P: \quad [A_1] \quad IF ~ X_1 \neq 0 \quad GOTO ~ A_1$
         \begin{gather*}
             \mathrm{Halt}(x,\#P) = 
             \begin{cases}
                 0 & x \neq 0\\
                 1 & x = 0
             \end{cases}
         \end{gather*}
\end{enumerate}

\begin{teorema}{}{}

 $\mathrm{Halt}: \mathbb{N}^2 \to \mathbb{N}$ es no computable.
    
\end{teorema}

\begin{proof} \phantom{.}

    Supongo que $\mathrm{Halt}$ es computable
    $\implies f: \mathbb{N} \to \mathbb{N} / f(x) = \mathrm{Halt}(x,x)$ es
    computable pues $f = \mathrm{Halt} \circ (\Pi_1 \times \Pi_1)$

    Armo el siguiente programa:
    \begin{gather*}
        \notamath{$\# \mathcal{P} = n_0$}
        \mathcal{P}: \quad [A_1] \quad IF ~ f(x) \neq 0 \quad GOTO ~ A_1
    \end{gather*}

    \begin{align*}
        f(n_0) = 1 \iff& \mathrm{Halt}(n_0, n_0) = 1 \\
        \iff& \mathcal{P} \text{ termina si } x = n_0 
        \notamath{Definición de $\mathrm{Halt}$} \\
        \iff& f(n_0)=0 \notamath{Mirando el programa}
    \end{align*}

    ¡Absurdo! Vino de suponer que $\mathrm{Halt}$ es computable.

    \begin{center}
        $\therefore ~ \mathrm{Halt}$ no es computable
    \end{center}

\end{proof}

\subsubsection{Ejemplos}

Decidir si $f$ es computable.

\begin{enumerate}
    \item $f: \mathbb{N}^2 \to \mathbb{N}/f(x,y) = \mathrm{Halt}(x,y)^2 + xy$
        
        Supongo $f$ computable $\implies$ defino
        \begin{align*}
            g: \mathbb{N} \to \mathbb{N} / g(x) =& f(x,x) \restatruncada x^2\\
            =& \underbrace{%
            \underbrace{\mathrm{Halt}(x,x)^2}_{\geq \, 0} + x^2}_{\geq \, x^2}
            \restatruncada x^2 \\
            =& \mathrm{Halt}(x,x)^2 \\
            =& \mathrm{Halt}(x,x) \notamath{\circled{1}}
        \end{align*}

        Siendo válido \circled{1} porque:
        \nota{\textit{Noni:} Pueden probar por inducción que 
        $\mathrm{Halt}(x,y)^n = \mathrm{Halt}(x,y)$}%
        \begin{align*}
            \mathrm{Halt}(x,x) = 1 \implies& \mathrm{Halt}(x,x)^2 = 1 \\
            \mathrm{Halt}(x,x) = 0 \implies& \mathrm{Halt}(x,x)^2 = 0
        \end{align*}

        Entonces $g$ es computable por ser composición de $f$, que
        estamos suponiendo computable, $\restatruncada$, $\mathrm{PROD}$, y 
        $\Pi_1$, que son RP y entonces, por teorema, computables.

        Por lo tanto, como $g$ es computable, $\mathrm{Halt}$ es computable.
        
        ¡Absurdo! Vino de suponer que $f$ es computable.

        \begin{center}
        Conclusión: $f$ no es computable.
        \end{center}

    \item $f: \mathbb{N}^2 \to \mathbb{N} / 
        f(x,y) = \langle \mathrm{Halt}(x,y), xy \rangle$

        Supongo que $f$ es computable, entonces defino
        \begin{align*}
            g: \mathbb{N}^2 \to \mathbb{N} / g(x,y) =& l(f(x,y)) \\
            =& \mathrm{Halt}(x,y)
        \end{align*}

        \nota{$l$ es RP $\implies$ computable\\$f$ computable 
        (lo estamos suponiendo)}%
        Como $g = l \circ f$ es computable por composición de computables,
        $\mathrm{Halt}$ es computable. ¡Absurdo!
        \begin{center}
            Conclusión: $f$ no es computable.
        \end{center}

    \item $f: \mathbb{N}^2 \to \mathbb{N} / 
        f(x,y) = \mathrm{Halt}(x,y)^2 - \mathrm{Halt}(x,y)$

        Ya vimos que $\mathrm{Halt}(x,y)^2 = \mathrm{Halt}(x,y)$

        Por lo tanto $f(x,y) = \mathrm{Halt}(x,y) - \mathrm{Halt}(x,y)=0$

        Entonces $f = \underbrace{\mathrm{cero} \circ \Pi_1}_{%
        \text{RP} \implies \text{computable}}$

        Por lo tanto $f$ es computable por composición de funciones 
        computables.

        Notemos que esta función es computable a pesar de estar originalmente 
        escrita como composición de no computables. 
        
        \textit{Noni}: Es un error grave afirmar que una función no es 
        computable por ser composición de funciones no computables. 
\end{enumerate}

\bigskip

\textit{Noni}: La estrategia para resolver este tipo de ejercicios es siempre
la misma.

Si la función es computable, intentar escribirla de una manera diferente con
funciones computables (sin usar $\mathrm{Halt}$); si no lo es, suponemos que
sí, la componemos con funciones que sabemos son computables y tratamos de que
nos quede $\mathrm{Halt}$.


\section{Tesis de Church y Turing}

\begin{tesis}{Tesis de Church}{church}
    Todos los algoritmos para computar funciones 
    $f: A \subseteq \mathbb{N}^k \to \mathbb{N}$ se pueden programar en
    lenguaje S.
\end{tesis}

Es decir, si el algoritmo no se puede programar en lenguaje S, entonces en 
ningún otro lenguaje se puede.

\medskip

\begin{tesis}{Tesis de Turing}{}
    Lo que no se puede resolver con una Máquina de Turing, no se puede resolver
    con otra máquina.
\end{tesis}

Las tesis no tienen demostración, se consideran como verdaderas por la 
comunidad matemática.

\section{Programas universales}

\begin{definicion}{Programas universales}{}
    Para cada $n > 0$ se define
    \begin{gather*}
        \phi^n : \mathbb{N}^{n+1} \to \mathbb{N} /
        \phi^n (x_1, \dotsc, x_n, e) = \Psi_P^n (x_1, \dotsc, x_n) 
    \end{gather*}

    Con $\# \mathcal{P}= e$.
\end{definicion}

Notemos que esta función solamente está definida si el programa $\mathcal{P}$ 
termina ante la entrada $(x_1, \dotsc, x_n)$, en cuyo caso devuelve
la salida del programa.

\medskip

\begin{teorema}{}{}
    $\phi^n$ es parcialmente computable $\forall n > 0$.
\end{teorema}

\begin{proof} 
    En el apunte de Programas Universales del Anexo.
\end{proof}


